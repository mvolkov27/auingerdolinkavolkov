%A correction suggested by Don Beyer [www-admin at arXiv.org]
%implemented on 16/02/2009 (it doesn't affect the output ps or
%pdf file)
%
%A new subsection on symplectic transpose and a corresponding
%subsection inserted on 24/04/2009
%
%A few typos in Section 5 corrected on 25/04/2009
%
\documentclass[11pt,reqno]{amsart}
\usepackage{cite}
\usepackage{url}
\usepackage{amssymb}
\usepackage{amscd}
\usepackage{amsfonts}
\usepackage{amsmath}
\usepackage{eepic}
\usepackage{gastex}
\usepackage{rotating}
\usepackage{amsthm}
\usepackage{amsgen}
\usepackage{amsmath}
\usepackage[dvips]{color}


\DeclareMathOperator{\dom}{dom} \DeclareMathOperator{\ran}{ran}
\DeclareMathOperator{\rk}{rk} \DeclareMathOperator{\tr}{tr}
\DeclareMathOperator{\var}{\mathsf{var}}
\DeclareMathOperator{\Id}{Eq}

\DeclareSymbolFont{rsfscript}{OMS}{rsfs}{m}{n}
\DeclareSymbolFontAlphabet{\mathrsfs}{rsfscript}

\def\softd{{\leavevmode\setbox1=\hbox{d}%
    \hbox to 1.05\wd1{d\kern-0.4ex{\char039}\hss}}}

\numberwithin{equation}{section}

\def\bb{\mathbb}

\newtheorem{Thm}{Theorem}[section]
\newtheorem{Prop}[Thm]{Proposition}
\newtheorem{Lemma}[Thm]{Lemma}
\newtheorem{Def}{Definition}[section]
\newtheorem{Res}[Thm]{Result}
\newtheorem{Cor}[Thm]{Corollary}
\newtheorem{Example}[Thm]{Example}
\newtheorem{Examples}[Thm]{Examples}
\newtheorem{Conjecture}[Thm]{Conjecture}
\newtheorem{Not}[Thm]{Notation}

\theoremstyle{remark}
\newtheorem{Rmk}{Remark}[section]
\newtheorem{Problem}{Problem}[section]

\def\eq{\simeq}
\def\La{\Lambda}
\def\om{\omega}
\def\cd{\cdot}
\def\wh{\widehat}
\def\pv#1{{\bf #1}}
\def\cal{\mathcal}
\def\Ac{{\cal A}}
\def\Bc{{\cal B}}
\def\Cc{{\cal C}}
\def\Dc{\mathrsfs{D}}
\def\Ec{{\cal E}}
\def\Fc{{\cal F}}
\def\Gc{{\cal G}}
\def\Hc{\mathrsfs{H}}
\def\Ic{{\cal I}}
\def\Jc{{\cal J}}
\def\Kc{{\cal K}}
\def\Lc{\mathrsfs{L}}
\def\Mc{{\cal M}}
\def\Nc{{\cal N}}
\def\Oc{{\cal O}}
\def\Pc{{\cal P}}
\def\Qc{{\cal Q}}
\def\Rc{\mathrsfs{R}}
\def\Sc{{\cal S}}
\def\Tc{{\cal T}}
\def\Uc{{\cal U}}
\def\Vc{\mathbf{V}}
\def\Wc{\mathbf{W}}
\def\Xc{{\cal X}}
\def\Yc{{\cal Y}}
\def\Zc{{\cal Z}}
\def\es{{\Ec\Sc}}
\def\cs{{\Cc\Sc}}
\def\si{\sigma}
\def\Si{\Sigma}
\def\al{\alpha}
\def\ga{\gamma}
\def\de{\delta}
\def\ep{\varepsilon}
\def\be{\beta}
\def\la{\lambda}
\def\te{\theta}
\def\ka{\kappa}
\def\rh{\rho}
\def\ta{\tau_\alpha}

\def\m{\mathrel}
\def\ol{\overline}
\def\mo{\models}


\def\s{\sigma}
\def\De{\Delta}

\def\ze{\zeta}
\def\io{\iota}
\def\fx{F(X)}
\def\ka{\kappa}
\def\we{\wedge}

\def\bp{\bar\phi}
\def\bps{\bar\psi}
\def\Co{{\mathrm C}}
\def\Ko{\Cal K}
\def\H{\mathrm H}
\def\P{\mathrm P\!}

\def\he#1{#1#1^*}
\def\re{regular $*$-}
\def\wi{weakly invertible}
\def\ig{idempotent-generated}
\def\Rm{Rees matrix}
\def\sm{semi\-group}
\def\va{variet}
\def\evar{\mathsf{evar}}
\def\A{\mathfrak{A}}
\def\C{\mathfrak{C}}
\def\B{\mathfrak{B}}
\def\J{\mathfrak{J}}
\def\Sim{\mathfrak{S}}
\def\ov{\overline}
\def\inv{^{-1}}
\def\wt{\widetilde}
\def\id{identit}
\def\fb{finitely based}
\def\TB{\ensuremath{\mathcal{T\kern-1pt B}_2^1}}
\def\TA{\ensuremath{\mathcal{T\kern-1pt A}_2^1}}

\title[Equational Theories]{Equational Theories of Involution Semigroups}

\author{K.~Auinger}
\address{Fakult\"at f\"ur Mathematik, Universit\"at Wien,
Nordbergstrasse 15,  A-1090 Wien, Austria}
\email{karl.auinger@univie.ac.at}
\author{I.~Dolinka}
\address{Department of Mathematics and Informatics, University of Novi Sad,
Trg Dositeja Obradovi\'ca 4, 21000 Novi Sad, Serbia}
\email{dockie@im.ns.ac.yu}
\author {M.~V.~Volkov}
\address{Faculty of Mathematics and Mechanics, Ural State University,
Lenina 51, 620083 Ekaterinburg, Russia}
\email{mikhail.volkov@usu.ru}

\begin{document}

\begin{abstract}
We employ a technique developed in an earlier paper to show that involution
semigroups arising in various contexts do not have a finite basis for their identities. Among these are
partition semigroups endowed with their natural involution,
including the full partition semigroup $\C_n$ for $n\ge 2$, the
Brauer semigroup $\B_n$ for $n\ge 4$ and the annular semigroup
$\A_n$ for $n\ge 4$, $n$ even or a prime power. We also show that
similar techniques apply to the finite basis problem for existence
varieties of locally inverse semigroups.
\end{abstract}

\maketitle %\tableofcontents

\section{Introduction and Preliminaries}
A fundamental and widely studied question connected with an
algebraic structure is whether its equational theory is finitely
axiomatizable. Being very natural by itself, this question, which
is usually referred to as the \emph{finite basis problem}, has
also revealed a number of interesting and unexpected relations to
many issues of theoretical and practical importance ranging from
feasible algorithms for membership in certain classes of formal
languages (see \cite{Alm95}) to classical number-theoretic
conjectures (such as the Twin Prime, Goldbach, existence of odd
perfect numbers and the infinitude of even perfect numbers: see
\cite{Per89} where it is shown that each of these conjectures is
equivalent to the finite axiomatizability of the equational theory
of a particular groupoid).

For the class of semigroups, an overview of results (up to the
year 2000) concerning the finite basis problem can be found in the
survey article~\cite{volkovjaponicae} by the third author. In that
article, the reader will find several \textbf{methods} to treat
the problem.  In \cite{adv} the authors developed `unary' versions
of two classical approaches to the finite basis problem, namely, the method of
inherently non-finitely based semigroups and `the critical
semigroup method' of the survey \cite{volkovjaponicae}. This lead
to two fairly general and easy-to-verify conditions on a unary
semigroup variety $\mathbf{V}$ guaranteeing that $\mathbf{V}$ is
not finitely based. The main field of application presented in \cite{adv}
was matrix semigroups over finite fields endowed with matrix transposition as unary operation. The authors were able to show that no such involutory semigroup does have a finite identity basis and, moreover, they completely classified the cases when these are inherently non-finitely based. In the present paper we present further applications of the second of the above-mentioned methods. The main theme of application will be partition semigroups which will be presented in  Section~2. Included are the Brauer
semigroup, see \cite{brauer}, the full partition semigroup and the annular monoid all of which arose in representation theory and gained much attention recently among semigroup theorists. Section~3 contains further applications, including joins of unary semigroup varieties. Finally, in Section~4, we demonstrate how the same approach applies to so-called existence
varieties of locally inverse semigroups. In the rest of this section we collect some terminology and results from \cite{adv}.

We assume the reader's acquaintance with basic concepts of the
theory of varieties such as the HSP-theorem, see, e.g.,
\cite[Chapter~II]{BuSa81}. As far as \sm\ notions are concerned,
we adopt the standard terminology and notation from~\cite{CP}. 

By a \emph{unary semigroup} we mean an algebraic
structure
 $\mathcal{S}=\langle S,\cdot,{}^*\rangle$ of type $(2,1)$ such that
the binary operation $\cdot$ is associative, i.\,e.\ $\langle
S,\cdot\rangle$ is a semigroup. In general, we do not assume any
additional identities involving the unary operation ${}^*$. If the
identities $(xy)^* = y^*x^*$ and $(x^*)^* = x$ happen to hold in
$\mathcal{S}$, in other words, if the unary operation $x\mapsto
x^*$ is an involutory anti-automorphism of the semigroup $\langle
S,\cdot\rangle$, we call $\mathcal{S}$ an \emph{involutory
semigroup}. If, in addition, the identity $x=xx^*x$ holds,
$\mathcal{S}$ is said to be a \emph{regular $*$-semigroup}. Each
group, subject to its inverse operation $x\mapsto x^{-1}$ is an
involutory semigroup, even a regular $*$-semigroup; throughout the
paper, any group is considered as a unary semigroup with respect
to this inverse unary operation.

A wealth of examples of involutory semigroups and regular
$*$-semigroups can be obtained via the following `unary' version
of the well known \Rm\ construction (see \cite[Section~3.1]{CP}
for a description of the construction in the plain semigroup
case). Let $\mathcal{G}=\langle G,\cdot,{}^{-1}\rangle$ be a
group, $0$ a symbol beyond $G$, and $I$ a non-empty set. We
formally set $0^{-1}=0$. Given an $I\times I$-matrix $P=(p_{ij})$
over $G\cup\{0\}$ such that $p_{ij}=p_{ji}^{-1}$ for all $i,j\in
I$, we define a multiplication~$\cdot$ and a unary operation
${}^*$ on the set $(I\times G\times I)\cup\{0\}$ by the following
rules:
\begin{gather*}
a\cdot 0=0\cdot a=0\ \text{ for all $a\in (I\times G\times I)\cup \{0\}$},\\
(i,g,j)\cdot(k,h,\ell)=\left\{\begin{array}{cl}
(i,gp_{jk}h,\ell)&\ \text{if}\ p_{jk}\ne0,\\
0 &\ \text{if}\ p_{jk}=0;
\end{array}\right.\\
(i,g,j)^* = (j,g^{-1},i),\ 0^* = 0.
\end{gather*}
It can be easily checked that $\langle(I\times G\times I)\cup
\{0\},\cdot,{}^*\rangle$ becomes  an involutory semigroup; it will
be a regular $*$-semigroup precisely when $p_{ii}=e$ (the identity
element of the group $\mathcal{G}$) for all $i\in I$. We denote
this unary semigroup  by $\Mc^0(I,\mathcal{G},I;P)$ and call it
the \emph{unary \Rm\ \sm\ over $\mathcal{G}$ with the sandwich
matrix $P$}. If the involved group $\mathcal G$ happens to be the
trivial group $\mathcal{E}=\{e\}$ then we usually shall ignore the
group entry and represent the non-zero elements of such a Rees
matrix semigroup by the pairs $(i,j)$ with $i,j\in I$.

In this paper, the 10-element unary \Rm\ semigroup over the
trivial group $\mathcal{E}=\{e\}$ with the sandwich matrix
$$\begin{pmatrix}
e & e & e\\
e & e & 0\\
e & 0 & e
\end{pmatrix}$$
plays a key role; we denote this \sm\ by $\mathcal{K}_3$. Thus,
subject to the convention mentioned above, $\mathcal{K}_3$
consists of the nine pairs $(i,j)$, $i,j\in\{1,2,3\}$, and the
element $0$, and the operations restricted to its non-zero
elements can be described as follows:
\begin{gather}
\label{operations in C3}
(i,j)\cdot(k,\ell)=\left\{\begin{array}{cl}
(i,\ell)&\ \text{if}\ (j,k)\ne(2,3),(3,2),\\
0 &\ \text{otherwise};
\end{array}\right.\\
(i,j)^* = (j,i)\notag.
\end{gather}


The formulation of our main tool involves a simple
operator on unary semigroup varieties. For any unary semigroup
$\mathcal{S}=\langle S,\cdot,{}^*\rangle$ we denote by
$\H(\mathcal{S})$ the unary subsemigroup of $\mathcal{S}$ which is
generated by all elements of the form $xx^*$, where $x\in S$. We
call $\H(\mathcal{S})$ the \emph{Hermitian subsemigroup} of
$\mathcal{S}$. For any variety $\Vc$ of unary semigroups, let
$\H(\Vc)$ be the subvariety of $\Vc$ generated by all Hermitian
subsemigroups of members of $\Vc$.
Denote by $\var\Sc$ the variety generated by a given unary
semigroup $\Sc$. The following observation is easy and is taken from \cite[Lemma 2.1]{adv}.

\begin{Lemma}\label{Lemma 3.1} For every unary semigroup $\Sc$ and each $n\in \bb N$ we have
$\H(\var\Sc)=\var\H(\Sc)$.
\end{Lemma}

The next result is (part of) Theorem 2.2  in \cite{adv}.

\begin{Thm}\label{Theorem 2.1}
Let $\Vc$ be any unary semigroup variety such that
$\mathcal{K}_3\in\Vc$. If  $\Vc$ contains a group which is not in $\H(\Vc)$
then $\Vc$ has no finite basis of identities.
\end{Thm}



\section{Applications to partition semigroups}
\subsection{Partition semigroups}
For each positive integer $n$ we are going to define
\begin{itemize}
\item the \emph{partition monoid} $\C_n$,
\item the \emph{Brauer monoid} $\B_n$,
\item the \emph{partial} Brauer monoid $P\B_n$,
\item the \emph{annular monoid} $\A_n$,
\item the \emph{partial} annular monoid $P\A_n$.
\end{itemize}
The monoids $\C_n$, $\B_n$ and $\A_n$ arise as vector space bases
of certain associative algebras which are relevant in
representation theory \cite{brauer, xi, jones, grahamlehrer}. The
semigroup structure and related questions for $\C_n$, $P\B_n$ and
$\B_n$ have been studied recently by Mazorchuk et al., see, for
example, \cite{Maz1, Maz2, KMM, KM2,malcevmazorchuk}. For each $n$
there are natural monoid embeddings
$$\A_n\hookrightarrow\B_n\hookrightarrow P\B_n\hookrightarrow \C_n.$$
Moreover, for each $n$ there are monoid embeddings
$$\C_n\hookrightarrow\C_{n+1},\quad P\B_n\hookrightarrow
P\B_{n+1},\quad \B_n\hookrightarrow\B_{n+1}.$$ For the annular
monoid there is no obvious monoid embedding
$$\A_n\hookrightarrow\A_{n+k}$$
for any $n\ge 2,k\ge 1$, but $\A_n$ appears as a
sub\emph{semigroup} of $\A_{n+2}$ for each $n$.

We start with the definition of $\C_n$. For each positive integer
$n$ let
$$[n]=\{1,\dots,n\},\quad [n]'=\{1',\dots,n'\},\quad [n]''=\{1'',\dots,n''\}$$
be three pairwise disjoint copies of the set of the first $n$
positive integers and put
$$\widetilde{[n]}=[n]\cup [n]'.$$
The base set of the partition monoid $\C_n$ is the set of all
partitions of the set $\widetilde{[n]}$; throughout, we consider a
partition of a set and the corresponding equivalence relation on
that set as two different views of the same thing and without
further mention we freely switch between these views, whenever it
seems to be convenient. For $\xi,\eta\in \C_n$, the product
$\xi\eta$ is defined (and computed) in four steps:
\begin{enumerate}
\item Consider the $'$-analogue of $\eta$: that is, define $\eta'$
on ${[n]'}\cup {[n]''}$ by
$${x'}\mathrel{\eta'}{y'}:\Leftrightarrow x\mathrel{\eta}
y\text{ for all } x,y\in \widetilde{[n]}.$$
\item Let $\left<\xi,\eta\right>$ be the equivalence relation on
$\widetilde{[n]}\cup {[n]''}$ generated by $\xi\cup {\eta'}$, that
is, set $\left<\xi,\eta\right>:=(\xi\cup {\eta'})^t$ where $^t$
denotes the transitive closure.
\item Forget all elements having a single prime $'$: that is, set
$$\left<\xi,\eta\right>^\circ:=\left<\xi,\eta\right>|_{[n]\cup{[n]''}}.$$
\item Replace  double primes with single primes to
obtain the product $\xi\eta$: that is, set
$$x\mathrel{\xi\eta}y:\Leftrightarrow f(x)\mathrel{\left<\xi,\eta\right>^\circ}f(y)
\text{ for all }x,y\in \widetilde{[n]}$$ where
$f:\widetilde{[n]}\to [n]\cup{[n]''}$ is the bijection
$$x\mapsto x, x'\mapsto x'' \text{ for all } x\in [n].$$
\end{enumerate}

For example, let $n=5$ and
$$\xi\rule[-22.5mm]{0pt}{45mm}=
\begin{picture}(40,0)(0,22.5)
\unitlength=0.8mm \drawrect(0,15,9,54) \put(3,8){5} \put(3,18){4}
\put(3,28){3} \put(3,38){2} \put(3,48){1} \drawrect(37,5,46,14)
\put(40,8){$5'$} \drawrect(37,15,46,24) \put(40,18){$4'$}
\put(40,28){$3'$} \drawrect(37,35,46,44) \put(40,38){$2'$}
\put(40,48){$1'$}
\drawline[AHnb=0](0,5)(0,14)(25,14)(25,54)(46,54)(46,45)(34,45)(34,34)(46,34)(46,25)(34,25)(34,5)(0,5)
\end{picture}\ ,\qquad
\eta=
\begin{picture}(40,0)(0,22.5)
\unitlength=0.8mm \drawrect(0,5,9,14) \put(3,8){5} \put(3,18){4}
\put(3,28){3} \put(3,38){2} \put(3,48){1} \drawrect(37,5,46,14)
\put(40,8){$5'$} \drawrect(0,15,46,24) \put(40,18){$4'$}
\put(40,28){$3'$} \put(40,38){$2'$} \put(40,48){$1'$}
\drawline[AHnb=0](0,45)(0,54)(46,54)(46,35)(37,35)(37,45)(0,45)
\drawline[AHnb=0](0,25)(0,44)(9,44)(9,34)(46,34)(46,25)(0,25)
\end{picture}\ .
$$
Then
$$\langle\xi,\eta\rangle\rule[-20mm]{0pt}{40mm}=
\begin{picture}(80,0)(0,20)
\unitlength=0.8mm \drawrect(0,15,9,54) \put(3,8){5} \put(3,18){4}
\put(3,28){3} \put(3,38){2} \put(3,48){1} \drawrect(37,5,46,14)
\put(40,8){$5'$} \drawrect(37,15,83,24) \put(40,18){$4'$}
\put(40,28){$3'$} \put(40,38){$2'$} \put(40,48){$1'$}
\drawrect(74,5,83,14) \put(77,8){$5''$} \put(77,18){$4''$}
\put(77,28){$3''$} \put(77,38){$2''$} \put(77,48){$1''$}
\drawline[AHnb=0](0,5)(0,14)(25,14)(25,54)(83,54)(83,25)(34,25)(34,5)(0,5)
\end{picture}
$$
and
$$\xi\eta\rule[-20mm]{0pt}{40mm}=
\begin{picture}(40,0)(0,20)
\unitlength=0.8mm \drawrect(0,15,9,54) \put(3,8){5} \put(3,18){4}
\put(3,28){3} \put(3,38){2} \put(3,48){1} \drawrect(37,5,46,14)
\put(40,8){$5'$} \drawrect(37,15,46,24) \put(40,18){$4'$}
\put(40,28){$3'$} \put(40,38){$2'$} \put(40,48){$1'$}
\drawline[AHnb=0](0,5)(0,14)(25,14)(25,54)(46,54)(46,25)(34,25)(34,5)(0,5)
\end{picture}\ .$$

This multiplication is associative making $\C_n$ a monoid with
identity $1$ where
$$1=\{\{k,k'\}\mid k\in [n]\}.$$
The group of units of $\C_n$ is the symmetric group $\Sim_n$
(acting on $[n]$ on the right) with canonical embedding
$\Sim_n\hookrightarrow \C_n$ given by
$$\sigma\mapsto \{\{k,(k\si)'\}\mid k\in [n]\} \text{ for all } \si\in
\Sim_n.$$ More generally, the monoid of all (partial)
transformations of $[n]$ acting on the right is also naturally
embedded in $\C_n$ by
\begin{equation}
\label{partialtransformations} \phi\mapsto \{\{k'\}\cup
k\phi^{-1}\mid k\in [n]\}\cup\{\{k\}\mid k\in [n],k\notin\dom
\phi\}
\end{equation}
where $\dom \phi$ is the domain of $\phi$. The equivalence classes
of some $\xi\in \C_n$ are usually referred to as \emph{blocks};
the \emph{rank} $\rk\xi$ is the number of blocks of $\xi$ whose
intersection with $[n]$ as well as with $[n]'$ is not empty ---
this coincides with the usual notion of rank of a partial mapping
on $[n]$ in case $\xi$ is in the image of the embedding
\eqref{partialtransformations}. It is  known that the rank
characterizes the $\Dc$-relation in $\C_n$ \cite{Maz1,KM2}: for
any $\xi,\eta\in\C_n$, one has $\xi\mathrel{\Dc}\eta$ if and only
if $\rk\xi=\rk\eta$.

The monoid $\C_n$ admits a natural involution making it a regular
$*$-semi\-group: consider first the permutation $^*$ on
$\widetilde{[n]}$ that swaps primed with unprimed elements, that
is, set
\[
k^*=k',\ (k')^*=k\text{ for all } k\in [n].
\]
Then define, for $\xi\in \C_n$,
\[
x\mathrel{\xi^*}y:\Leftrightarrow {x^*}\mathrel{\xi}{y^*} \text{
for all }x,y\in \widetilde{[n]}.
\]
That is, $\xi^*$ is obtained from $\xi$ by interchanging in $\xi$
the primed with the unprimed elements. It is easy to see that
\begin{equation}
\label{projections} \xi^{**}=\xi,\ (\xi\eta)^*=\eta^*\xi^*\ \text{
and }\ \xi\xi^*\xi=\xi\ \text{ for all }\ \xi,\eta\in \C_n.
\end{equation}
The elements of the form $\xi\xi^*$ are called \emph{projections}.
They are idempotents (as one readily sees from the last equality
in \eqref{projections}) and have the following transparent
structure. If $k$ is the rank of $\xi\xi^*$ (equal to the rank of
$\xi$), then there is some $t\in \{0,1,\dots,n-k\}$ and a
partition of $[n]$ into $k+t$ blocks:
$$[n]=A_1\cup\cdots\cup A_k\cup B_1\cup\cdots \cup B_t$$ such that
$$\xi\xi^*=\{A_1\cup A_1',\dots, A_k\cup
A_k',B_1,B_1',\dots,B_t,B_t'\}.$$ Fig.\,\ref{projection} shows a
typical projection in $\C_8$; here $k=3$, $t=2$ and $A_1=\{3,4\}$,
$A_2=\{5,8\}$, $A_3=\{7\}$ while $B_1=\{1,2\}$, $B_2=\{6\}$.
\begin{figure}[hb]
\unitlength=1.5mm
\begin{picture}(80,22)
\put(73,3){8} \put(63,3){7} \put(53,3){6} \put(43,3){5}
\put(33,3){4} \put(23,3){3} \put(13,3){2} \put(3,3){1}
\put(73,18){$8'$} \put(63,18){$7'$} \put(53,18){$6'$}
\put(43,18){$5'$} \put(33,18){$4'$} \put(23,18){$3'$}
\put(13,18){$2'$} \put(3,18){$1'$} \drawrect(2,17,16,21)
\drawrect(2,2,16,6) \drawrect(22,2,36,21) \drawrect(52,17,56,21)
\drawrect(52,2,56,6)
\drawline[AHnb=0](42,2)(42,21)(46,21)(46,13.5)(60,13.5)(62,14.5)(66,14.5)(68,13.5)%
(72,13.5)(72,21)(76,21)(76,2)(72,2)(72,9.5)(68,9.5)(66,10.5)(62,10.5)(60,9.5)(46,9.5)(46,2)(42,2)
\drawline[AHnb=0](62,9.7)(62,2)(66,2)(66,9.7)
\drawline[AHnb=0](62,15.3)(62,21)(66,21)(66,15.3)
\end{picture}
\caption{A rank 3 projection in $\C_8$}\label{projection}
\end{figure}

From this it follows easily that the maximal subgroup of $\C_n$
with identity $\xi\xi^*$ is isomorphic to the symmetric group
$\Sim_k$ where the isomorphism between $\Sim_k$ and the group
$\Hc$-class of $\xi\xi^*$ is given by
\begin{equation}\label{embeddingofSk}
\si\mapsto\{A_1\cup A'_{1\si},\dots A_k\cup
A'_{k\si},B_1,B'_1,\dots B_t,B'_t\},\ \si\in\Sim_k.
\end{equation}
We note that in the group $\Hc$-class of any projection, the
involution $^*$ coincides with the inverse operation in that
group. Since $\xi\mathrel{\Rc}\xi\xi^*$, \emph{each} maximal
subgroup of $\C_n$ is isomorphic to the symmetric group $\Sim_k$
for some $k\le n$.

The \emph{Brauer monoid} and the \emph{partial} Brauer monoid can
be conveniently defined as submonoids of $\C_n$: namely, $\B_n$
[respectively $P\B_n$] consists of all elements of $\C_n$ all of
whose blocks have size~2 [at most~2]. Both monoids are closed
under the involution $^*$ and in both cases, the group $\Hc$-class
of a projection of rank $k$ is isomorphic (as a regular
$*$-semigroup) with the symmetric group $\Sim_k$. Now let
$\mathfrak{K}_n$ be any of $\C_n$, $P\B_n$ or $\B_n$. Each
projection different from the identity of $\mathfrak{K}_n$ has
rank less than $n$, whence the monoid $\H(\mathfrak{K}_n)$
contains, apart from the identity element, only elements of rank
strictly less than $n$. This implies

\begin{Cor}
\label{groups in the gap} For each $n\ge 2$ and each
$\mathfrak{K}_n\in \{\C_n,P\B_n,\B_n\}$ there exists a group in
$\var\mathfrak{K}_n$ that is not in $\var\H(\mathfrak{K}_n)$.
\end{Cor}
\begin{proof}
The group in question is the symmetric group $\Sim_n$ that is the
group of units in $\mathfrak{K}_n$ and thus belongs to
$\var\mathfrak{K}_n$. By the argument of Kim and Roush~\cite{KR},
each group in the variety $\var\H(\mathfrak{K}_n)$ belongs to the
group variety generated by the subgroups of the monoid
$\H(\mathfrak{K}_n)$. As observed above, each subgroup of
$\H(\mathfrak{K}_n)$ embeds into the symmetric group $\Sim_{n-1}$
whence it remains to check that $\Sim_n$ does not belong to the
group variety generated by $\Sim_{n-1}$. This follows from
\cite[Theorem 51.2]{Ne} because the group $\Sim_n$ always has a
chief factor of order larger than the maximum order of chief
factors in $\Sim_{n-1}$.
\end{proof}

For completeness we note that $\B_1$ is the trivial monoid and
$P\B_1\cong \C_1$ is isomorphic to the $2$-element semilattice
monoid $\{0,1\}$ (endowed with trivial involution).

Next we define the \emph{annular monoid} $\A_n$ \cite{jones}. It
will be realized as a certain submonoid of the Brauer monoid. For
this purpose it is convenient to first represent the elements of
$\B_n$ as \emph{annular diagrams}. Consider an annulus $A$ in the
complex plane, say $A=\{z\mid 1<|z|<2\}$ and identify the elements
of $\wt{[n]}$ with certain points of the boundary of $A$ via
$$k\mapsto 2e^{\frac{2\pi i(k-1)}{n}}\ \text{ and }\ k'\mapsto
e^{\frac{2\pi i(k-1)}{n}}\ \text{ for all }\ k\in [n].$$ For
$\xi\in\B_n$ take a copy of $A$ and link any $x,y\in\wt{[n]}$ with
$\{x,y\}\in \xi$ by a path (called \emph{string}) running entirely
in $A$ (except for its endpoints). For example, the element
$\xi\in \B_4$ given by
$$\xi=\{\{1,1'\},\{2,4\},\{3,2'\},\{3',4'\}\}$$
is then represented by the annular diagram in Fig.\,\ref{diagram}.
\begin{figure}[ht]
\unitlength=0.8mm
\begin{picture}(50,64)(0,-32)
\drawcircle(25,0,25) \drawcircle(25,0,50)
\gasset{AHnb=0,ExtNL=y,Nfill=y,Nw=2,Nh=2,Nmr=1,linewidth=.4}
\node[NLangle=-90,NLdist=2](D1)(25,-25){4}
\node[NLangle=180,NLdist=2](C1)(0,0){3}
\node[NLdist=2](B1)(25,25){2}
\node[NLangle=0,NLdist=2](A1)(50,0){1}
\node[NLdist=2](D2)(25,-12.5){$4'$}
\node[NLangle=0,NLdist=2](C2)(12.5,0){$3'$}
\node[NLangle=-90,NLdist=2](B2)(25,12.5){$2'$}
\node[NLangle=180,NLdist=2](A2)(37.5,0){$1'$} \drawedge(A1,A2){}
\drawedge[curvedepth=-20](D1,B1){}
\drawedge[curvedepth=5](C1,B2){}
\drawedge[curvedepth=7.5](D2,C2){}
\end{picture}
\caption{Annular diagram representation of
 a member of $\B_4$}\label{diagram}
\end{figure}
Paths representing blocks of the form $\{x,y'\}$ [$\{x,y\}$ and
$\{x',y'\}$, respectively] for some $x,y\in[n]$ are called
\emph{through strings} [\emph{outer} and \emph{inner strings},
respectively]. The \emph{annular monoid} $\A_n$ by definition
consists of all elements of $\B_n$ that have a representation as
an annular diagram any two of whose strings have empty
intersection. One can compose annular diagrams in an obvious way,
modelling the multiplication in $\B_n$ --- from this it follows
easily that $\A_n$ is closed under the multiplication of $\B_n$.
Clearly, $\A_n$ is closed under the involution~$^*$, as well.

Although we shall not need it, we remark that the rank
characterizes the $\Dc$-relation in $\A_n$, as well. Let
$\xi\in\B_n$ be of rank $t$ with through strings
$$\{k_1,l'_1\},\dots,\{k_t,l'_t\},\ \mbox{for some } k_i,l_i\in [n].$$
Then $\{k_1,\dots, k_t\}$ respectively $\{l'_1,\dots,l'_t\}$ is
the \emph{domain} $\dom \xi$ respectively \emph{range} $\ran \xi$
of $\xi$. For any projection $\ep$ we obviously have $\ran
\ep=(\dom \ep)'$.

In order to show that the rank function characterizes the
$\Dc$-relation it is sufficient to show that any two projections
$\ep, \eta$ of the same rank $t$ are $\Dc$-related. Let $\ep$ and
$\eta$ be arbitrary projections of rank $t$ with
$a_1<a_2<\cdots<a_t$  the domain of $\ep$ and
$b'_1<b'_2<\cdots<b'_t$ the range of $\eta$; define $\alpha$ to be
the element having the same outer strings as $\ep$, the same inner
strings as $\eta$ and the through strings
$$\{a_1,b'_1\},\dots,\{a_t,b'_t\}.$$
Then $\alpha\in \A_n$, $\ep=\al\al^*$ and $\eta=\al^*\al$.

\subsection{Subgroups in annular monoids}

It was observed by Jones \cite{jones} that the maximal subgroup of
$\A_n$ whose identity is a projection $\ep$ of rank $t$ is a
cyclic group of order $t$. Indeed, suppose that
$k_1<k_2<\dots<k_t$ are the elements of $\dom \ep$ and let $\xi\in
\A_n$ be $\Hc$-related with $\ep$. In $\xi$ there exists a unique
through string of the form $\{k_1,k'_\ell\}$. From the annular
condition it follows that the remaining $t-1$ through strings of
$\xi$ are precisely
$$\{k_2,k'_{\ell+1}\},\dots,\{k_{t-\ell+1},k'_t\},\,\{k_{t-\ell+2},k'_1\}\dots,
\{k_t,k'_{\ell-1}\}$$ while the inner and outer strings of $\xi$
are those of $\ep$. Taking into account~\eqref{embeddingofSk}, we
observe that $\xi=\tau^{\ell-1}$, where $\tau$ consists of the
through strings
$$\{k_1,k'_2\},\{k_2,k'_3\},\dots,\{k_t,k'_1\}$$
together with the inner and outer strings of $\ep$. Obviously,
 $\tau^t=\ep$ and the order of $\tau$ is $t$.

In the following we shall obtain some facts about $\H(\A_n)$ in
case $n$ is even.
\begin{Lemma}\label{differenceininnerstrings}
If $\{i,j\}$ $[$respectively $\{i',j'\}]$ is an outer $[$inner$]$
string of an element in $\A_{2m}$, then $j-i$ is odd.
\end{Lemma}
\begin{proof} Suppose that $j>i$. On the circuit, only points either between
$i$ and $j$ or between $j$ and $i$ (in, say, counterclockwise
orientation) can be involved in through strings. In the first
case, all points between $j$ and $i$ must be involved in outer
strings hence the number of points between $j$ and $i$ is even;
since the total number of points is even, this implies that $j-i$
is odd. In the second case, by the same reason there must be an
even number of points between $i$ and $j$ whence $j-i$ is odd,
anyway.
\end{proof}

This enables us to obtain the next result.

\begin{Lemma}\label{differenceinthroughstrings}
Let $\al=\ep_1\cdots \ep_k\in\A_{2m}$ be a product of projections
$\ep_s$. Then for each through string $\{i,j'\}$ of $\al$, the
difference $j-i$ is even.
\end{Lemma}

\begin{proof}
The proof is by induction on $k$, the case $k=1$ being trivial.
So, suppose that $k>1$ and set $\be=\ep_1\dots\ep_{k-1}$ and
$\ep=\ep_k$. Let $\{i,j'\}$ be a through string in $\be\ep$. By
the definition of the multiplication, there exist
$k_0,k_1,\dots,k_{2t}\in [n]$ such that
$$i\mathrel{\be}k'_0,\ k_0\mathrel{\ep} k_1,\ k'_1\mathrel{\be}
k'_2,\dots ,k'_{2t-1}\mathrel{\be}k'_{2t},\
k_{2t}\mathrel{\ep}j'.$$ Since $\ep$ is a projection and
$\{k_{2t},j'\}$ is a through string, we have $k_{2t}=j$. Hence
$$j-i=\sum_{s=1}^{2t}(k_s-k_{s-1})+(k_0-i).$$
By the induction assumption, $k_0-i$ is even since $\{i,k'_0\}$ is
a through string in $\be$. By Lemma
\ref{differenceininnerstrings}, each $k_{s+1}-k_s$ is odd.
Consequently, the sum $\sum_{s=1}^{2t}(k_{s}-k_{s-1})$, containing
an even number of odd summands, is even.
\end{proof}

For the next result note that for any $\xi\in\A_{2m}$ of rank $t$,
if $k_1<k_2<\dots<k_t$ are the elements of the domain of $\xi$,
then $k_{i+1}-k_i$ is odd for each $i<t$ --- the argument is
similar to that in Lemma \ref{differenceininnerstrings}. Since $t$
necessarily is even, $k_t-k_1$ is also odd.

\begin{Cor}\label{differenceinindices}
Let $\al\in\A_{2m}$ be a product of projections, $\al$ having
domain $k_1<k_2<\cdots<k_t$ and range $k'_1<k'_2<\cdots<k'_t$. If
$\{k_i,k'_j\}$ is a through string of $\al$, then $j-i$ is even.
\end{Cor}

\begin{proof}
Suppose that $j>i$; by Lemma~\ref{differenceinthroughstrings},
$k_j-k_i$ is even. Since
\begin{equation}\label{evensummands}
k_j-k_i=(k_j-k_{j-1})+\cdots+(k_{i+1}-k_i)
\end{equation}
and each summand on the right hand side of (\ref{evensummands}) is
odd by the above remark, there must be an even number of summands
in that sum and that number coincides with $j-i$.
\end{proof}

This allows us to obtain the principal result in this context.

\begin{Cor} \label{maximalsubgroups}
For each even number $n$, each maximal subgroup of $\H(\A_n)$ has
order less than $\frac{n}2$.
\end{Cor}

\begin{proof}
Each non-trivial maximal subgroup of $\H(\A_n)$ is isomorphic to
the group $\Hc$-class (in $\H(\A_n)$) of some projection $\ep\ne
1$. Let $t$ be the rank of $\ep$ (which is even) and
$k_1<k_2<\cdots<k_t$ be the elements of the domain of $\ep$. Let
$\tau\in \A_n$ be defined by having the through strings
$\{k_1,k'_2\}, \dots,\{k_t,k'_1\}$ and all inner and outer strings
the same as $\ep$ (that is, $\tau$ is $\Hc$-related in $\A_n$ to
$\ep$). Clearly, $\tau$ has order $t$ and generates the group
$\Hc$-class of $\ep$ in $\A_n$. Now let $\al\in \H(\A_n)$ be a
generating element of the group $\Hc$-class of $\ep$ in
$\H(\A_n)$; then $\al$ and $\ep$ have the same domain and range.
There exists a unique $s$ such that $\{k_1,k'_{s+1}\}$ is a
through string in $\al$. By Corollary~\ref{differenceinindices},
$s$ is even. By the annular condition, all the remaining through
strings of $\al$ are of the form
$$\{k_2,k'_{s+2}\},\dots,\{k_{t-s},k'_t\},\,\{k_{t-s+1},k'_1\}\dots,
\{k_t,k'_s\}.$$ Then
$$\al=\tau^s=(\tau^2)^{\frac s 2}$$ whence $\al$ lies in the
subgroup generated by $\tau^2$. The order of $\tau^2$ is $\frac t
2$, hence the order of $\al$ is at most $\frac t2$ which is less
than $\frac n 2$.
\end{proof}

Altogether we are able to detect a group in $\var\A_n$ that is not
in $\var\H(\A_n)$:

\begin{Cor}
\label{differenceofanandhan} For each even number $n$ there exists
a group in $\var\A_n$ that is not in $\var\H(\A_n)$.
\end{Cor}

\begin{proof} We have seen that all cyclic groups in $\var\H(\A_n)$
have even order less than $\frac n2$. On the other hand, each
cyclic group of even order up to $n$ belongs to $\var\A_n$. In
order to find a (cyclic) group in $\var\A_n$ that is not in
$\var\H(\A_n)$ it suffices to find an even number $k\le n$ that
does not divide the least common multiple of all even numbers less
than $\frac n 2$. For such $k$ we may take the largest power of
$2$ which is less than or equal to $n$.
\end{proof}

Since, for any $n$, all maximal subgroups of $\H(\A_n)$ have order
at most $n-2$, by the same reasoning as in
Corollary~\ref{differenceofanandhan}, the next result is
immediate.

\begin{Cor}
\label{differenceofanandhanodd} Let $n$ be a prime power; then the
cyclic group of order $n$ belongs to $\var\A_n$ but not to
$\var\H(\A_n)$.
\end{Cor}

Finally, we show that the cases $n$ even or a prime power are the
only ones for which there is a group in $\var\A_n$ that is not in
$\var\H(\A_n)$. Therefore, our methods are applicable precisely in
these cases.

\begin{figure}[ht]
\begin{picture}(100,100)(-10,5)
\gasset{AHnb=0,ExtNL=y,Nfill=y,Nw=1,Nh=1,Nmr=.5,NLdist=1}
\node[NLangle=180](A1)(-5,10){$n$}
\node[NLangle=180](B1)(-5,20){$n-1$} \put(-5.5,28){$\vdots$}
\node[NLangle=180](C1)(-5,40){$t+2$}
\node[NLangle=180](D1)(-5,50){$t+1$}
\node[NLangle=180](E1)(-5,60){$t$}
\node[NLangle=180](F1)(-5,70){$t-1$} \put(-5.5,78){$\vdots$}
\node[NLangle=180](G1)(-5,90){$2$}
\node[NLangle=180](H1)(-5,100){$1$} \node(A2)(5,10){}
\node(B2)(5,20){} \put(4.5,28){$\vdots$} \node(C2)(5,40){}
\node(D2)(5,50){} \node(E2)(5,60){} \node(F2)(5,70){}
\put(4.5,78){$\vdots$} \node(G2)(5,90){} \node(H2)(5,100){}
\drawedge[curvedepth=-2](A1,B1){} \drawedge[curvedepth=2](A2,B2){}
\drawedge[curvedepth=-2](C1,D1){} \drawedge[curvedepth=2](C2,D2){}
\drawedge[curvedepth=2](E2,D2){}
\drawedge[linegray=1,ELside=r](A1,A2){$\varepsilon_{t+1}$}
\drawedge(E1,E2){} \drawedge(F1,F2){} \drawedge(G1,G2){}
\drawedge(H1,H2){} \node(A3)(15,10){} \node(B3)(15,20){}
\put(14.5,28){$\vdots$} \node(C3)(15,40){} \node(D3)(15,50){}
\node(E3)(15,60){} \node(F3)(15,70){} \put(14.5,78){$\vdots$}
\node(G3)(15,90){} \node(H3)(15,100){}
\drawedge[curvedepth=2](A3,B3){} \drawedge[curvedepth=-2](A2,B2){}
\drawedge[curvedepth=2](D3,E3){} \drawedge[curvedepth=-2](E3,F3){}
\drawedge[linegray=1,ELside=r](A2,A3){$\varepsilon_{t}$}
\drawedge(C3,C2){} \drawedge(F3,F2){} \drawedge(G3,G2){}
\drawedge(H3,H2){} \node(A3)(15,10){} \node(B3)(15,20){}
\put(14.5,28){$\vdots$} \node(C3)(15,40){} \node(D3)(15,50){}
\node(E3)(15,60){} \node(F3)(15,70){} \put(14.5,78){$\vdots$}
\node(G3)(15,90){} \node(H3)(15,100){}
\drawedge[curvedepth=2](A3,B3){} \drawedge[curvedepth=-2](A2,B2){}
\drawedge[curvedepth=2](D3,E3){} \drawedge[curvedepth=-2](E3,F3){}
\drawedge[linegray=1,ELside=r](A2,A3){$\varepsilon_{t}$}
\node(A4)(25,10){} \node(B4)(25,20){} \put(24.5,28){$\vdots$}
\node(C4)(25,40){} \node(D4)(25,50){} \node(E4)(25,60){}
\node(F4)(25,70){} \put(24.5,78){$\vdots$} \node(G4)(25,90){}
\node(H4)(25,100){} \drawedge[curvedepth=-2](A3,B3){}
\drawedge[curvedepth=2](A4,B4){} \drawedge[curvedepth=2](E4,F4){}
\drawedge[linegray=1,ELside=r](A3,A4){$\varepsilon_{t-1}$}
\drawedge(C3,C4){} \drawedge(D3,D4){} \drawedge(G3,G4){}
\drawedge(H3,H4){} \put(29,10){$\dots$} \put(29,20){$\dots$}
\put(29,40){$\dots$} \put(29,50){$\dots$} \put(29,60){$\dots$}
\put(29,90){$\dots$} \put(29,100){$\dots$} \node(A5)(37,10){}
\node(B5)(37,20){} \put(36.5,28){$\vdots$} \node(C5)(37,40){}
\node(D5)(37,50){} \node(E5)(37,60){} \put(36.5,68){$\vdots$}
\node(F5)(37,80){} \node(G5)(37,90){} \node(H5)(37,100){}
\drawedge[curvedepth=2](G5,F5){} \node(A6)(47,10){}
\node(B6)(47,20){} \put(46.5,28){$\vdots$} \node(C6)(47,40){}
\node(D6)(47,50){} \node(E6)(47,60){} \put(46.5,68){$\vdots$}
\node(F6)(47,80){} \node(G6)(47,90){} \node(H6)(47,100){}
\drawedge[curvedepth=-2](G6,F6){}
\drawedge[curvedepth=-2](G6,H6){}
\drawedge[curvedepth=-2](A5,B5){} \drawedge[curvedepth=2](A6,B6){}
\drawedge[linegray=1,ELside=r](A5,A6){$\varepsilon_2$}
\drawedge(C5,C6){} \drawedge(D5,D6){} \drawedge(E5,E6){}
\drawedge(H5,H6){} \node(A7)(57,10){} \node(B7)(57,20){}
\put(56.5,28){$\vdots$} \node(C7)(57,40){} \node(D7)(57,50){}
\node(E7)(57,60){} \put(56.5,68){$\vdots$} \node(F7)(57,80){}
\node(G7)(57,90){} \node(H7)(57,100){}
\drawedge[curvedepth=2](G7,H7){} \drawedge[curvedepth=-2](A6,B6){}
\drawedge[curvedepth=2](A7,B7){} \drawedge[curvedepth=-2](A7,B7){}
\drawedge[linegray=1,ELside=r](A6,A7){$\varepsilon_1$}
\drawedge(C7,C6){} \drawedge(D7,D6){} \drawedge(E7,E6){}
\drawedge(F6,F7){} \drawbcedge(H7,67,130,C7,67,10){}
\node(A8)(77,10){} \node(B8)(77,20){} \put(76.5,28){$\vdots$}
\node(C8)(77,40){} \node(D8)(77,50){} \node(E8)(77,60){}
\put(76.5,68){$\vdots$} \node(F8)(77,80){} \node(G8)(77,90){}
\node(H8)(77,100){} \drawedge[curvedepth=2](A8,B8){}
\drawedge[curvedepth=-2](A8,B8){}
\drawedge[linegray=1,ELside=r](A7,A8){$\varepsilon_0$}
\drawedge(D7,D8){} \drawedge(E7,E8){} \drawedge(F7,F8){}
\drawedge(G7,G8){} \drawbcedge(C8,67,10,H8,67,130){}
\node(A9)(87,10){} \node(B9)(87,20){} \put(86.5,28){$\vdots$}
\node(C9)(87,40){} \node(D9)(87,50){} \node(E9)(87,60){}
\put(86.5,68){$\vdots$} \node(F9)(87,80){} \node(G9)(87,90){}
\node(H9)(87,100){} \drawedge[curvedepth=2](A9,B9){}
\drawedge[linegray=1,ELside=r](A8,A9){$\varepsilon_{t+1}$}
\drawedge[curvedepth=-2](C8,D8){} \drawedge[curvedepth=2](C9,D9){}
\drawedge(E9,E8){} \drawedge(F9,F8){} \drawedge(G9,G8){}
\drawedge(H9,H8){}
\end{picture}
\caption{A cycle of order $t$ as a product of projections in
$\A_n$}\label{shortdecomposition}
\end{figure}

\begin{Prop}
\label{noprimepowerodd} If $n$ is odd and not a prime power then
the groups in $\var\A_n$ and $\var\H(\A_n)$ are the same.
\end{Prop}
\begin{proof} Every element of $\A_n$ has odd rank. Let $t<n$ be odd.
We define projections $\ep_0,\ep_1,\dots,\ep_{t+1}$ as follows.
Each $\ep_i$, $i=0,1,\dots,t+1$, has the outer strings
$\{t+3,t+4\},\dots,\{n-1,n\}$ and the corresponding inner strings;
besides those, the projection $\ep_0$ has the outer string
$\{1,t+2\}$ and the inner string $\{1',(t+2)'\}$ and each of the
projections $\ep_i$, $i=1,\dots,t+1$, has the outer string
$\{i,i+1\}$ and the outer string $\{i',(i+1)'\}$. Finally, all
remaining elements of $\wt{[n]}$ are involved in the through
strings  $\{k,k'\}$.

One then readily verifies (see Fig.\,\ref{shortdecomposition})
that $\al=\ep_{t+1}\ep_t\cdots \ep_0\ep_{t+1}$ has the same inner
and outer strings as $\ep_{t+1}$ hence
$\al\mathrel{\Hc}\ep_{t+1}$. Moreover, the through strings of
$\al$ are
$$\{1,3'\},\{2,4'\},\dots,\{t-1,1'\},\{t,2'\}.$$
Via \eqref{embeddingofSk}, $\al$ realizes the cyclic permutation
$k\mapsto k+2\!\pmod{t}$ which has order $t$ since $t$ is odd.
Thus, for each odd $t<n$, the monoid $\H(\A_n)$ contains a cyclic
group of order $t$ as a unary subsemigroup. The maximal subgroups
of $\A_n$ are precisely the cyclic groups of odd order at most
$n$. So, we have already shown that $\var\H(\A_n)$ contains each
maximal subgroup of $\A_n$, with the possible exception of the
group of units of $\A_n$ which is cyclic of order $n$. Since $n$
is not a prime power, $n=k\ell$ for some co-prime numbers
$k,\ell$. As already pointed out, the cyclic groups
$\mathcal{C}_k$ and $\mathcal{C}_\ell$ of orders $k$ and $\ell$,
respectively, belong to $\var\H(\A_n)$ whence so does the cyclic
group of order $n$ which is isomorphic to $\mathcal{C}_k
\times\mathcal{C}_\ell$.
\end{proof}
\begin{Cor} There exists a group in $\var\A_n$ that is not in
$\var\H(\A_n)$ if and only if $n$ is even or a prime power.
\end{Cor}

Analogously to the partial Brauer monoid $P\B_n$, one could also
define the \emph{partial} annular monoid $P\A_n$  by considering
all elements of $P\B_n$ which admit a representation by an annular
diagram in which any two distinct strings have empty intersection.
Clearly, for each $n$, there is a unary semigroup embedding
$P\A_n\hookrightarrow P\A_{n+1}$. The elements of $P\A_n$ can have
rank $t$ for each $t\le n$ and the maximal subgroups of $P\A_n$
are precisely the cyclic groups of orders at most $n$.

In contrast to the ordinary annular case, it is no longer true
that there is a group in $\var P\A_n$ that is not in
$\var\H(P\A_n)$ for each even $n$. The fact that some elements of
$\wt{[n]}$ need not be involved in any string gives the
projections more freedom to gain cyclic permutations in
$\H(P\A_n)$, as the following result demonstrates.

\begin{Prop}
\label{partialannular} For each $n\ge 5$ and $t\le n-3$ there
exists a cyclic subgroup of order $t$ in $\H(P\A_n)$.
\end{Prop}

\begin{proof}
For odd $t$ this follows immediately from Proposition
\ref{noprimepowerodd}. For even $t$ it can be shown that the
element $\al$ consisting of the through strings
$$\{1,2'\}, \{2,3'\},\dots,\{t-1,t'\},\{t,1'\}$$
along with the outer string $\{t{+}2,t{+}3\}$ and the inner string
$\{(t{+}2)',(t{+}3)'\}$, and else having no other strings can be
written as a product of $\frac{5t}2 + 4$ projections, see
Fig.\,\ref{longdecomposition}. Clearly, $\al$ realizes a cyclic
permutation of order~$t$.
\end{proof}

\begin{figure}[p]
\unitlength=.9mm
\begin{picture}(100,235)(-100,-52)
\begin{rotate}{90}
\gasset{AHnb=0,ExtNL=y,Nfill=y,Nw=1,Nh=1,Nmr=.5,NLdist=1}
\node[NLangle=180](A1)(-45,0){$t+3$}
\node[NLangle=180](B1)(-45,10){$t+2$}
\node[NLangle=180](C1)(-45,20){$t+1$}
\node[NLangle=180](D1)(-45,30){$t$}
\node[NLangle=180](E1)(-45,40){$t-1$}
\node[NLangle=180](F1)(-45,50){$t-2$}
\node[NLangle=180](G1)(-45,80){$3$}
\node[NLangle=180](H1)(-45,90){$2$}
\node[NLangle=180](I1)(-45,100){$1$} \node(A2)(-35,0){}
\node(B2)(-35,10){} \node(C2)(-35,20){} \node(D2)(-35,30){}
\node(E2)(-35,40){} \node(F2)(-35,50){} \node(G2)(-35,80){}
\node(H2)(-35,90){} \node(I2)(-35,100){}
\drawedge[curvedepth=-2](A1,B1){} \drawedge[curvedepth=2](A2,B2){}
\drawedge(D1,D2){} \drawedge(E1,E2){} \drawedge(F1,F2){}
\drawedge(G1,G2){} \drawedge(H1,H2){} \drawedge(I1,I2){}
\drawedge[linegray=1,ELside=r](A1,A2){$\varepsilon_1$}
\drawedge[curvedepth=-3](B2,D2){} \put(-46,65){$\vdots$}
\put(-36,65){$\vdots$} \put(-26,65){$\vdots$}
\put(-16,65){$\vdots$} \put(-6,65){$\vdots$} \put(4,65){$\vdots$}
\node(A3)(-25,0){} \node(B3)(-25,10){} \node(C3)(-25,20){}
\node(D3)(-25,30){} \node(E3)(-25,40){} \node(F3)(-25,50){}
\node(G3)(-25,80){} \node(H3)(-25,90){} \node(I3)(-25,100){}
\drawedge[curvedepth=3](B3,D3){} \drawedge[curvedepth=-2](D3,E3){}
\drawedge(A3,A2){$\varepsilon_2$} \drawedge(E3,E2){}
\drawedge(F3,F2){} \drawedge(G3,G2){} \drawedge(H3,H2){}
\drawedge(I3,I2){} \node(A4)(-15,0){} \node(B4)(-15,10){}
\node(C4)(-15,20){} \node(D4)(-15,30){} \node(E4)(-15,40){}
\node(F4)(-15,50){} \node(G4)(-15,80){} \node(H4)(-15,90){}
\node(I4)(-15,100){} \drawedge[curvedepth=2](D4,E4){}
\drawedge[curvedepth=2](D4,C4){}
\drawedge[ELside=r](A3,A4){$\varepsilon_3$} \drawedge(B3,B4){}
\drawedge(F3,F4){} \drawedge(G3,G4){} \drawedge(H3,H4){}
\drawedge(I3,I4){} \node(A5)(-5,0){} \node(B5)(-5,10){}
\node(C5)(-5,20){} \node(D5)(-5,30){} \node(E5)(-5,40){}
\node(F5)(-5,50){} \node(G5)(-5,80){} \node(H5)(-5,90){}
\node(I5)(-5,100){} \drawedge[curvedepth=3](F5,D5){}
\drawedge[curvedepth=-2](D5,C5){}
\drawedge(A5,A4){$\varepsilon_4$} \drawedge(B5,B4){}
\drawedge(F5,F4){} \drawedge(G5,G4){} \drawedge(H5,H4){}
\drawedge(I5,I4){} \node(A6)(5,0){} \node(B6)(5,10){}
\node(C6)(5,20){} \node(D6)(5,30){} \node(E6)(5,40){}
\node(F6)(5,50){} \node(G6)(5,80){} \node(H6)(5,90){}
\node(I6)(5,100){} \drawedge[curvedepth=-3](F6,D6){}
\drawedge[ELside=r](A5,A6){$\varepsilon_5$} \drawedge(B5,B6){}
\drawedge(C5,C6){} \drawedge(G5,G6){} \drawedge(H5,H6){}
\drawedge(I5,I6){} \put(9,0){$\dots$} \put(9,10){$\dots$}
\put(9,20){$\dots$} \put(9,30){$\dots$} \put(9,80){$\dots$}
\put(9,90){$\dots$} \put(9,100){$\dots$} \node(A7)(17,0){}
\node(B7)(17,10){} \node(C7)(17,20){} \node(D7)(17,30){}
\node(E7)(17,60){} \node(F7)(17,70){} \node(G7)(17,80){}
\node(H7)(17,90){} \node(I7)(17,100){}
\drawedge[curvedepth=-2](F7,G7){} \node(A8)(27,0){}
\node(B8)(27,10){} \node(C8)(27,20){} \node(D8)(27,30){}
\node(E8)(27,60){} \node(F8)(27,70){} \node(G8)(27,80){}
\node(H8)(27,90){} \node(I8)(27,100){} \put(16,45){$\vdots$}
\put(26,45){$\vdots$} \put(36,45){$\vdots$} \put(46,45){$\vdots$}
\put(56,45){$\vdots$} \put(66,45){$\vdots$} \put(86,45){$\vdots$}
\put(96,45){$\vdots$} \put(106,45){$\vdots$}
\put(116,45){$\vdots$} \put(126,45){$\vdots$} \drawedge(H7,H8){}
\drawedge(I7,I8){} \drawedge(D7,D8){} \drawedge(C7,C8){}
\drawedge(B7,B8){} \drawedge(A7,A8){}
\drawedge[curvedepth=2](F8,G8){} \drawedge[curvedepth=2](F8,E8){}
\node(A9)(37,0){} \node(B9)(37,10){} \node(C9)(37,20){}
\node(D9)(37,30){} \node(E9)(37,60){} \node(F9)(37,70){}
\node(G9)(37,80){} \node(H9)(37,90){} \node(I9)(37,100){}
\drawedge(I9,I8){} \drawedge(H9,H8){} \drawedge(D9,D8){}
\drawedge(C9,C8){} \drawedge(B9,B8){} \drawedge(A9,A8){}
\drawedge[curvedepth=-2](F9,E9){}
\drawedge[curvedepth=-3](F9,H9){} \node(A10)(47,0){}
\node(B10)(47,10){} \node(C10)(47,20){} \node(D10)(47,30){}
\node(E10)(47,60){} \node(F10)(47,70){} \node(G10)(47,80){}
\node(H10)(47,90){} \node(I10)(47,100){} \drawedge(I9,I10){}
\drawedge(E9,E10){} \drawedge(D9,D10){} \drawedge(C9,C10){}
\drawedge(B9,B10){} \drawedge(A9,A10){}
\drawedge[curvedepth=3](F10,H10){}
\drawedge[curvedepth=-2](H10,I10){} \node(A11)(57,0){}
\node(B11)(57,10){} \node(C11)(57,20){} \node(D11)(57,30){}
\node(E11)(57,60){} \node(F11)(57,70){} \node(G11)(57,80){}
\node(H11)(57,90){} \node(I11)(57,100){} \drawedge(F11,F10){}
\drawedge(E11,E10){} \drawedge(D11,D10){} \drawedge(C11,C10){}
\drawedge(B11,B10){}
\drawedge(A11,A10){$\varepsilon_{\frac{3t}2}$}
\drawedge[curvedepth=2](H11,I11){}
\drawedge[curvedepth=3](I11,G11){} \node(A12)(67,0){}
\node(B12)(67,10){} \node(C12)(67,20){} \node(D12)(67,30){}
\node(E12)(67,60){} \node(F12)(67,70){} \node(G12)(67,80){}
\node(H12)(67,90){} \node(I12)(67,100){} \drawedge(F11,F12){}
\drawedge(E11,E12){} \drawedge(D11,D12){} \drawedge(C11,C12){}
\drawedge(B11,B12){}
\drawedge(A12,A11){$\varepsilon_{\frac{3t}2+1}$}
\drawedge[curvedepth=-3](I12,G12){}
\drawbcedge(I12,77,130,A12,77,-30){} \node(A13)(87,0){}
\node(B13)(87,10){} \node(C13)(87,20){} \node(D13)(87,30){}
\node(E13)(87,60){} \node(F13)(87,70){} \node(G13)(87,80){}
\node(H13)(87,90){} \node(I13)(87,100){}
\drawbcedge(I13,77,130,A13,77,-30){}
\drawedge[curvedepth=-2](H13,I13){} \drawedge(G13,G12){}
\drawedge(F13,F12){} \drawedge(E13,E12){} \drawedge(D13,D12){}
\drawedge(C13,C12){} \drawedge(B13,B12){}
\drawedge[linegray=1](A13,A12){$\varepsilon_{\frac{3t}2+2}$}
\node(A14)(97,0){} \node(B14)(97,10){} \node(C14)(97,20){}
\node(D14)(97,30){} \node(E14)(97,60){} \node(F14)(97,70){}
\node(G14)(97,80){} \node(H14)(97,90){} \node(I14)(97,100){}
\drawedge[curvedepth=2](H14,I14){}
\drawedge[curvedepth=2](H14,G14){} \drawedge(G13,G14){}
\drawedge(F13,F14){} \drawedge(E13,E14){} \drawedge(D13,D14){}
\drawedge(C13,C14){} \drawedge(B13,B14){}
\drawedge[linegray=1](A14,A13){$\varepsilon_{\frac{3t}2+3}$}
\node(A15)(107,0){} \node(B15)(107,10){} \node(C15)(107,20){}
\node(D15)(107,30){} \node(E15)(107,60){} \node(F15)(107,70){}
\node(G15)(107,80){} \node(H15)(107,90){} \node(I15)(107,100){}
\drawedge(I15,I14){} \drawedge(F15,F14){} \drawedge(E15,E14){}
\drawedge(D15,D14){} \drawedge(C15,C14){} \drawedge(B15,B14){}
\drawedge[curvedepth=2](G15,F15){}
\drawedge[curvedepth=-2](H15,G15){}
\drawedge[linegray=1](A15,A14){$\varepsilon_{\frac{3t}2+4}$}
\node(A16)(117,0){} \node(B16)(117,10){} \node(C16)(117,20){}
\node(D16)(117,30){} \node(E16)(117,60){} \node(F16)(117,70){}
\node(G16)(117,80){} \node(H16)(117,90){} \node(I16)(117,100){}
\drawedge(I15,I16){} \drawedge(H15,H16){} \drawedge(E15,E16){}
\drawedge(D15,D16){} \drawedge(C15,C16){} \drawedge(B15,B16){}
\drawedge[curvedepth=-2](G16,F16){}
\drawedge[curvedepth=2](F16,E16){} \node(A17)(127,0){}
\node(B17)(127,10){} \node(C17)(127,20){} \node(D17)(127,30){}
\node(E17)(127,60){} \node(F17)(127,70){} \node(G17)(127,80){}
\node(H17)(127,90){} \node(I17)(127,100){} \drawedge(I17,I16){}
\drawedge(H17,H16){} \drawedge(G17,G16){} \drawedge(D17,D16){}
\drawedge(C17,C16){} \drawedge(B17,B16){}
\drawedge[curvedepth=-2](F17,E17){} \put(131,0){$\dots$}
\put(131,10){$\dots$} \put(131,20){$\dots$} \put(131,30){$\dots$}
\put(131,70){$\dots$} \put(131,80){$\dots$} \put(131,90){$\dots$}
\put(131,100){$\dots$} \node(A18)(139,0){} \node(B18)(139,10){}
\node(C18)(139,20){} \node(D18)(139,30){} \node(E18)(139,40){}
\node(F18)(139,70){} \node(G18)(139,80){} \node(H18)(139,90){}
\node(I18)(139,100){} \drawedge[curvedepth=-2](D18,E18){}
\put(138,55){$\vdots$} \put(148,55){$\vdots$}
\put(158,55){$\vdots$} \put(168,55){$\vdots$}
\put(178,55){$\vdots$} \node(A19)(149,0){} \node(B19)(149,10){}
\node(C19)(149,20){} \node(D19)(149,30){} \node(E19)(149,40){}
\node(F19)(149,70){} \node(G19)(149,80){} \node(H19)(149,90){}
\node(I19)(149,100){} \drawedge(I18,I19){} \drawedge(H18,H19){}
\drawedge(G18,G19){} \drawedge(F18,F19){} \drawedge(C18,C19){}
\drawedge(B18,B19){} \drawedge[curvedepth=2](D19,C19){}
\drawedge[curvedepth=2](D19,E19){}
\drawedge[linegray=1](A19,A18){$\varepsilon_{\frac{5t}2+1}$}
\node(A20)(159,0){} \node(B20)(159,10){} \node(C20)(159,20){}
\node(D20)(159,30){} \node(E20)(159,40){} \node(F20)(159,70){}
\node(G20)(159,80){} \node(H20)(159,90){} \node(I20)(159,100){}
\drawedge(I20,I19){} \drawedge(H20,H19){} \drawedge(G20,G19){}
\drawedge(F20,F19){} \drawedge(E20,E19){} \drawedge(B20,B19){}
\drawedge[curvedepth=-2](D20,C20){}
\drawedge[curvedepth=-2](B20,C20){}
\drawedge[linegray=1](A20,A19){$\varepsilon_{\frac{5t}2+2}$}
\node(A21)(169,0){} \node(B21)(169,10){} \node(C21)(169,20){}
\node(D21)(169,30){} \node(E21)(169,40){} \node(F21)(169,70){}
\node(G21)(169,80){} \node(H21)(169,90){} \node(I21)(169,100){}
\drawedge(I20,I21){} \drawedge(H20,H21){} \drawedge(G20,G21){}
\drawedge(F20,F21){} \drawedge(E20,E21){} \drawedge(D20,D21){}
\drawedge[curvedepth=-2](A21,B21){}
\drawedge[curvedepth=2](B21,C21){}
\drawedge[linegray=1](A21,A20){$\varepsilon_{\frac{5t}2+3}$}
\node(A22)(179,0){} \node(B22)(179,10){} \node(C22)(179,20){}
\node(D22)(179,30){} \node(E22)(179,40){} \node(F22)(179,70){}
\node(G22)(179,80){} \node(H22)(179,90){} \node(I22)(179,100){}
\drawedge(I22,I21){} \drawedge(H22,H21){} \drawedge(G22,G21){}
\drawedge(F22,F21){} \drawedge(E22,E21){} \drawedge(D22,D21){}
\drawedge[curvedepth=2](A22,B22){}
\drawedge[linegray=1](A22,A21){$\varepsilon_{\frac{5t}2+4}$}
\end{rotate}
\end{picture}
\caption{A cycle of order $t$ as a product of projections in
$P\A_n$}\label{longdecomposition}
\end{figure}

From this we obtain:
\begin{Cor}
\label{partialannularbadcases} If $n\notin\{p^k,p^k+1,2^k+2\}$ for
each prime $p$ and each $k\ge 1$, then $\var\H(P\A_n)$ and $\var
P\A_n$ contain the same groups.
\end{Cor}
\begin{proof} We may assume that $n\ge 15$. As already mentioned,
the variety of all groups in $\var P\A_n$ is generated by all
cyclic groups of orders at most $n$. By
Proposition~\ref{partialannular}, all cyclic groups of orders at
most $n-3$ belong to $\var\H(P\A_n)$. Since $n$ is not a prime
power it can be factored as $n=k\ell$ with $k,\ell$ co-prime and
$k,\ell\le n-3$. Since the cyclic groups of order $k$ and $\ell$
belong to $\var\H(P\A_n)$, so does the cyclic group of order $n$.
The same reasoning applies to the cyclic group of order $n-1$.
Consider finally the case of $n-2$. By assumption, $n-2$ either is
an odd prime power or has at least two distinct prime factors. In
the former case the claim follows from the proof of
Proposition~\ref{noprimepowerodd} and in the latter case the
argument is the same as for $n$ and $n-1$.
\end{proof}

On the other hand, the converse of
Corollary~\ref{partialannularbadcases} also holds.

\begin{Prop}
\label{partialannulargoodcases} If $n\in\{p^k,p^k+1,2^k+2\}$ for
some prime $p$ and some positive integer $k$, then there exists a
group in $\var P\A_n$ which is not in $\var\H(P\A_n)$.
\end{Prop}

\begin{proof} The case $n=p^k$ is obvious. Since the product of any two
\emph{distinct} projections of rank $n-1$ has rank less than
$n-1$, the group $\Hc$-class in $\H(P\A_n)$ of any projection of
rank $n-1$ is trivial, implying the claim for the case $n=p^k+1$.

Finally, in case $n=2^k+2$ we show that the cyclic group of order
$2^k=n-2$ is not in $\var\H(P\A_n)$. Let $\ep$ be a projection of
rank $n-2$ containing the outer string $\{i,i+1\}$ and the inner
string $\{i',(i+1)'\}$ and let $\ep^\circ$ be the projection
obtained from $\ep$ by removing these two strings. If $\eta$ is a
projection of rank $n-1$ such that $\eta\ep$ has rank $n-2$, then
$\eta\ep=\ep^\circ\ep$ (and likewise $\ep\eta=\ep\ep^\circ$).
Hence, if $\al$ is of rank $n-2$ and a product of projections then
we may assume that all these projections have rank  $n-2$.
Moreover, any product of two distinct projections of rank $n-2$
that have only through strings has rank less than $n-2$. Finally,
let $\ep_1, \ep_2, \ep_3$ be projections of rank $n-2$ such that
$\ep_1$ and $\ep_3$ have outer and inner strings but $\ep_2$ does
not. If $\ep_1\ep_2\ep_3$ has rank $n-2$ then $\ep_1=\ep_3$ and
$\ep_1\ep_2\ep_3=\ep_1\ep_3=\ep_1$.

Let $\alpha$ be of rank $n-2$ and assume that it is a product of
projections: $\alpha=\ep_0\ep_1\cdots \ep_r$. The observations in
the preceding paragraph imply that in addition we may assume that
$\ep_i$ has rank $n-2$ for each $i=0,\dots,r$ and that
$\ep_1,\dots, \ep_{r-1}$ have inner and outer strings, that is,
$\ep_1,\dots, \ep_{r-1}$ belong to $\A_n$. Assume further that
$\alpha$ is contained in a subgroup of $\H (P\A_n)$. We intend to
prove that the order of $\alpha$ is at most $\frac{n-2}2=2^{k-1}$.
For this purpose we may assume that $\alpha$ is $\Hc$-related to a
projection $\ep$. This implies immediately that $\ep_0=\ep=\ep_r$.

Now consider two cases: (i) $\ep$ has an inner and an outer
string, that is, $\ep$ belongs to $\A_n$, and (ii) $\ep$ has only
through strings, that is, $\ep$ does not belong to $\A_n$. In the
first case, $\alpha$ belongs to $\H(\A_n)$ and so the order of
$\alpha$ is at most $\frac{n-2}2$ by Corollary
\ref{maximalsubgroups}.

In the second case, we get $\ep_1=\ep_{r-1}$ and $\ep=\ep_1^\circ$
since $\ep\ep_1$ as well as $\ep_{r-1}\ep$ have  rank $n-2$. From
this it follows that the set $\{\ep,\ep_1,\ep\ep_1,\ep_1\ep\}$
forms a $2\times 2$-rectangular band under multiplication. In
particular, $\ep$ and $\ep_1$ are $\Dc$-related in $\H (P\A_n)$.
Green's Lemma implies that the order of $\alpha$ is the same as
the order of $\ep_1\alpha\ep_1=\ep_1\cdots\ep_{r-1}$. The latter
element belongs to $\H(\A_n)$, so its order is at most
$\frac{n-2}2$, again by Corollary \ref{maximalsubgroups}. Since no
group element of rank less than $n-2$ can have order $n-2$ we
actually have shown that $\H (P\A_n)$ does not contain a group
element of order $n-2=2^k$.

Altogether, the cyclic group of order $2^k$ belongs to $\var
P\A_n$ but not to $\var\H (P\A_n)$, as required.
\end{proof}

\subsection{Membership of $\mathcal{K}_3$}
In order to complete the results we need to check membership of
$\mathcal{K}_3$.

\begin{Prop}\label{membershipofC3}
The unary semigroup $\mathcal{K}_3$ is contained in
\begin{enumerate}
\item $\var\C_n$ for each $n\ge 2$,
\item $\var P\A_n\subseteq \var P\B_n$ for each $n\ge 3$,
\item $\var \A_n\subseteq \var \B_n$ for each $n\ge 4$.
\end{enumerate}
\end{Prop}
\begin{proof} In the first case, consider the unary subsemigroup $\mathcal{U}_1$ of $\C_2$
generated by the projections of rank~1 --- these are
\[
\{\{1,1',2,2'\}\},\{\{1,1'\},\{2\},\{2'\}\},
\{\{1\},\{1'\},\{2,2'\}\}.
\]
It is easy to calculate that $\mathcal{U}_1$ contains 13
partitions: 9 of rank~1 and 4 of rank~0. The $\Dc$-class of
$\mathcal{U}_1$ consisting of partitions of rank~1 is shown in
Fig.\,\ref{C3inC2} where the idempotents are marked with $\star$.
\begin{figure}[ht]
\begin{picture}(50,50)
\multiput(0,0)(0,20){3}{%
\multiput(0,0)(20,0){3}{\put(0,0){$\bullet$}%
\put(0,8){$\bullet$}\put(8,0){$\bullet$}\put(8,8){$\bullet$}}}
\drawrect(-1,39,11,51) \multiput(4,4)(0,20){3}{$\star$}
\multiput(24,24)(0,20){2}{$\star$}
\multiput(44,4)(0,40){2}{$\star$}
\multiput(0,0)(20,0){3}{%
\drawrect(-1,19,3,23)}
\drawline[AHnb=0](-1,27)(-1,31)(11,31)(11,19)(7,19)(7,27)(-1,27)
\multiput(0,0)(20,0){3}{%
\drawrect(-1,7,3,11)}
\drawline[AHnb=0](-1,-1)(-1,3)(7,3)(7,11)(11,11)(11,-1)(-1,-1)
\multiput(28,-8)(0,20){3}{%
\drawrect(-1,7,3,11)}
\multiput(48,0)(0,20){3}{%
\drawrect(-1,7,3,11)} \drawrect(19,27,31,31) \drawrect(39,-1,51,3)
\drawline[AHnb=0](19,39)(19,51)(31,51)(31,47)(23,47)(23,39)(19,39)
\drawline[AHnb=0](39,39)(39,51)(43,51)(43,43)(51,43)(51,39)(39,39)
\drawline[AHnb=0](21,-2)(18,1)(29,12)(32,9)(21,-2)
\drawline[AHnb=0](38,29)(41,32)(52,21)(49,18)(38,29)
\end{picture}
\caption{The upper $\Dc$-class of the subsemigroup $\mathcal{U}_1$
of $\C_2$}\label{C3inC2}
\end{figure}

Now it is clear that if one factors $\mathcal{U}_1$ by the ideal
of all elements of rank~0, then the resulting unary semigroup is
isomorphic to $\mathcal{K}_3$. Thus, $\mathcal{K}_3$ belongs to
the variety $\var\C_2$, and hence, to the variety $\var\C_n$ for
each $n\ge 2$.

For the second case consider the unary subsemigroup
$\mathcal{U}_2$ of $P\A_3$ generated by the projections
\begin{gather*}
\{\{1,1'\},\{2,3\},\{2',3'\}\},\\
\{\{1,2\},\{1',2'\},\{3,3'\}\},\\
\{\{1,1'\},\{2\},\{2'\},\{3\},\{3'\}\}.
\end{gather*}
Again it is easy to calculate that $\mathcal{U}_2$ contains 9
partitions of rank~1 and 9~partitions of rank~0. The partitions of
rank~1 form a regular $\Dc$-class depicted in Fig.\,\ref{C3inPA3}.
As above, the idempotents are marked with $\star$.
\begin{figure}[hb]
\unitlength 1.25mm
\begin{picture}(40,40)
\gasset{AHnb=0,ExtNL=y,Nfill=y,Nw=1,Nh=1,Nmr=.5,NLdist=1}
\multiput(3,3)(0,16){3}{$\star$}
\multiput(19,19)(0,16){2}{$\star$}
\multiput(35,3)(0,32){2}{$\star$} \node(A11)(0,0){}
\node(A12)(0,4){} \node(A13)(0,8){} \node(B11)(8,0){}
\node(B12)(8,4){} \node(B13)(8,8){} \drawedge(A13,B13){}
\drawedge[curvedepth=2](B11,B12){} \node(A14)(0,16){}
\node(A15)(0,20){} \node(A16)(0,24){} \node(B14)(8,16){}
\node(B15)(8,20){} \node(B16)(8,24){}
\drawedge[curvedepth=2](B14,B15){}
\drawedge[curvedepth=2](A16,A15){} \drawedge(A14,B16){}
\node(A17)(0,32){} \node(A18)(0,36){} \node(A19)(0,40){}
\node(B17)(8,32){} \node(B18)(8,36){} \node(B19)(8,40){}
\drawedge[curvedepth=2](B17,B18){}
\drawedge[curvedepth=2](A18,A17){} \drawedge(A19,B19){}
\node(A21)(16,0){} \node(A22)(16,4){} \node(A23)(16,8){}
\node(B21)(24,0){} \node(B22)(24,4){} \node(B23)(24,8){}
\drawedge(A23,B21){} \drawedge[curvedepth=2](B22,B23){}
\node(A24)(16,16){} \node(A25)(16,20){} \node(A26)(16,24){}
\node(B24)(24,16){} \node(B25)(24,20){} \node(B26)(24,24){}
\drawedge(A24,B24){} \drawedge[curvedepth=2](B25,B26){}
\drawedge[curvedepth=2](A26,A25){} \node(A27)(16,32){}
\node(A28)(16,36){} \node(A29)(16,40){} \node(B27)(24,32){}
\node(B28)(24,36){} \node(B29)(24,40){} \drawedge(A29,B27){}
\drawedge[curvedepth=2](B28,B29){}
\drawedge[curvedepth=2](A28,A27){} \node(A31)(32,0){}
\node(A32)(32,4){} \node(A33)(32,8){} \node(B31)(40,0){}
\node(B32)(40,4){} \node(B33)(40,8){} \drawedge(A33,B33){}
\node(A34)(32,16){} \node(A35)(32,20){} \node(A36)(32,24){}
\node(B34)(40,16){} \node(B35)(40,20){} \node(B36)(40,24){}
\drawedge(A34,B36){} \drawedge[curvedepth=2](A36,A35){}
\node(A37)(32,32){} \node(A38)(32,36){} \node(A39)(32,40){}
\node(B37)(40,32){} \node(B38)(40,36){} \node(B39)(40,40){}
\drawedge(A39,B39){} \drawedge[curvedepth=2](A38,A37){}
\end{picture}
\caption{The upper $\Dc$-class of the subsemigroup $\mathcal{U}_2$
of $P\A_3$}\label{C3inPA3}
\end{figure}

Again, it follows that the quotient of $\mathcal{U}_2$ by the
ideal of all elements of rank~0 is isomorphic to $\mathcal{K}_3$.
We see that $\mathcal{K}_3$ belongs to the variety $\var P\A_3$,
and hence, to the varieties $\var P\A_n$ and $\var P\B_n$ for each
$n\ge 3$.

Finally, for the third case consider the unary subsemigroup
$\mathcal{U}_3$ of $\A_4$ generated by the projections
\begin{gather*}
\{\{1,1'\},\{2,3\},\{2',3'\},\{4,4'\}\},\\
\{\{1,1'\},\{2,2'\},\{3,4\},\{3',4'\}\},\\
\{\{1,2\},\{1',2'\}, \{3,3'\}, \{4,4'\}\}.
\end{gather*}
It can be easily shown that $\mathcal{U}_3$ contains 13
partitions: 9 of rank~2 and 4 of rank~0. (Actually, one can
observe that $\mathcal{U}_3$ is isomorphic to $\mathcal{U}_1$
where the isomorphism is induced by the following mapping of the
base sets: $1,2\mapsto 1$, $3,4\mapsto 2$, $1',2'\mapsto 1'$ and
$3',4'\mapsto 2'$.) Fig.\,\ref{C3inA4} presents the top
$\Dc$-class of $\mathcal{U}_3$ consisting of the partitions of
rank~2.
\begin{figure}[ht]
\begin{picture}(52,52)
\gasset{AHnb=0,ExtNL=y,Nfill=y,Nw=1,Nh=1,Nmr=.5,NLdist=1}
\multiput(5,5)(0,20){3}{$\star$}
\multiput(25,25)(0,20){2}{$\star$}
\multiput(45,5)(0,40){2}{$\star$} \node(A11)(0,0){}
\node(A12)(0,4){} \node(A13)(0,8){} \node(A14)(0,12){}
\node(B11)(12,0){} \node(B12)(12,4){} \node(B13)(12,8){}
\node(B14)(12,12){} \drawedge(A11,B11){} \drawedge(A12,B14){}
\drawedge[curvedepth=2](A14,A13){}
\drawedge[curvedepth=2](B12,B13){} \node(A15)(0,20){}
\node(A16)(0,24){} \node(A17)(0,28){} \node(A18)(0,32){}
\node(B15)(12,20){} \node(B16)(12,24){} \node(B17)(12,28){}
\node(B18)(12,32){} \drawedge[curvedepth=2](B16,B17){}
\drawedge[curvedepth=2](A16,A15){} \drawedge(A17,B15){}
\drawedge(A18,B18){} \node(A19)(0,40){} \node(A110)(0,44){}
\node(A111)(0,48){} \node(A112)(0,52){} \node(B19)(12,40){}
\node(B110)(12,44){} \node(B111)(12,48){} \node(B112)(12,52){}
\drawedge[curvedepth=2](B110,B111){}
\drawedge[curvedepth=2](A111,A110){} \drawedge(A19,B19){}
\drawedge(A112,B112){} \node(A21)(20,0){} \node(A22)(20,4){}
\node(A23)(20,8){} \node(A24)(20,12){} \node(B21)(32,0){}
\node(B22)(32,4){} \node(B23)(32,8){} \node(B24)(32,12){}
\drawedge(A21,B23){} \drawedge(A22,B24){}
\drawedge[curvedepth=2](B21,B22){}
\drawedge[curvedepth=2](A24,A23){} \node(A25)(20,20){}
\node(A26)(20,24){} \node(A27)(20,28){} \node(A28)(20,32){}
\node(B25)(32,20){} \node(B26)(32,24){} \node(B27)(32,28){}
\node(B28)(32,32){} \drawedge(A27,B27){} \drawedge(A28,B28){}
\drawedge[curvedepth=2](B25,B26){}
\drawedge[curvedepth=2](A26,A25){} \node(A29)(20,40){}
\node(A210)(20,44){} \node(A211)(20,48){} \node(A212)(20,52){}
\node(B29)(32,40){} \node(B210)(32,44){} \node(B211)(32,48){}
\node(B212)(32,52){} \drawedge(A29,B211){} \drawedge(A212,B212){}
\drawedge[curvedepth=2](B29,B210){}
\drawedge[curvedepth=2](A211,A210){} \node(A31)(40,0){}
\node(A32)(40,4){} \node(A33)(40,8){} \node(A34)(40,12){}
\node(B31)(52,0){} \node(B32)(52,4){} \node(B33)(52,8){}
\node(B34)(52,12){} \drawedge(A31,B31){} \drawedge(A32,B32){}
\drawedge[curvedepth=2](B33,B34){}
\drawedge[curvedepth=2](A34,A33){} \node(A35)(40,20){}
\node(A36)(40,24){} \node(A37)(40,28){} \node(A38)(40,32){}
\node(B35)(52,20){} \node(B36)(52,24){} \node(B37)(52,28){}
\node(B38)(52,32){} \drawedge(A37,B35){} \drawedge(A38,B36){}
\drawedge[curvedepth=2](A36,A35){}
\drawedge[curvedepth=2](B37,B38){} \node(A39)(40,40){}
\node(A310)(40,44){} \node(A311)(40,48){} \node(A312)(40,52){}
\node(B39)(52,40){} \node(B310)(52,44){} \node(B311)(52,48){}
\node(B312)(52,52){} \drawedge(A39,B39){} \drawedge(A312,B310){}
\drawedge[curvedepth=2](A311,A310){}
\drawedge[curvedepth=2](B311,B312){}
\end{picture}
\caption{The upper $\Dc$-class of the subsemigroup $\mathcal{U}_3$
of $\A_4$}\label{C3inA4}
\end{figure}

Thus, factoring $\mathcal{U}_3$ by the ideal of all elements of
rank $0$, one gets a unary semigroup isomorphic to
$\mathcal{K}_3$. Therefore, $\mathcal{K}_3$ belongs to the variety
$\var\A_4$, and hence, to the varieties $\var\A_n$ for all even
$n\ge 4$ (recall that there is a unary semigroup embedding
$\A_n\hookrightarrow\A_{n+2}$) and $\var\B_n$ for each $n\ge 4$.

It remains to verify that $\mathcal{K}_3$ belongs to the variety
$\var\A_5$ (as then it also belongs to all the varieties
$\var\A_n$ with odd $n\ge 5$). Here an obvious modification of the
above construction works, namely, we add to each of the 13
partitions forming $\mathcal{U}_3$ the new through string
$\{5,5'\}$. It is easy to see that the resulting 13 partitions lie
in $\A_5$ and form a unary subsemigroup isomorphic to
$\mathcal{U}_3$.
\end{proof}

We can summarize the results of this section as follows.

\begin{Thm}
\label{mainresultpartitionsemigroups} The following regular
$*$-semigroups are not finitely based:
\begin{enumerate}
\item $\C_n$  for $n\ge 2$,
\item $P\B_n$ for $n\ge 3$,
\item $\B_n$ for $n\ge 4$,
\item $\A_n$ for $n\ge 4$, $n$ even or a prime power,
\item $P\A_n$ for $n\ge 3$, $n$ of the form $2^k+2$, $p^k$ or $p^k+1$ for
a  prime $p$ and $k\ge 1$.
\end{enumerate}
\end{Thm}

\begin{proof}
From Corollaries~\ref{groups in the gap},
\ref{differenceofanandhan}, and~\ref{differenceofanandhanodd} and
Propositions \ref{partialannulargoodcases}
and~\ref{membershipofC3}, it follows that Theorem~\ref{Theorem
2.1} applies in each case.
\end{proof}

Given this result, the question arises what happens in the cases
not covered by Theorem \ref{mainresultpartitionsemigroups}. First
of all, we may formulate
\begin{Problem}
\begin{enumerate}
\item  Is $\A_n$ finitely based for $n$ odd, not a prime power?
\item Is $P\A_n$ finitely based for $n\notin\{2^k+2,p^k,p^k+1\}$
($p$  prime, $k\ge 1$)?
\end{enumerate}
\end{Problem}

Remaining are now only some cases for small $n$. In case $n=1$ we
have: $\B_1\cong \A_1$ is the trivial monoid which is of course
finitely based and $P\B_1\cong P\A_1$ is the two element
semilattice (with trivial involution) which is also finitely
based. In case $n=2$, $\B_2\cong \A_2$ is a Clifford semigroup (a
cyclic group of order $2$ with zero adjoined) which is finitely
based, and $P\B_2\cong P\A_2$ which turns out to be an ideal
extension of a $2\times 2$ rectangular band (with involution) by
the symmetric inverse semigroup of rank 2 --- we do not know if
this is finitely based. Finally, in case $n=3$ we observe that
$\B_3$ is an ideal extension of a $3\times 3$ rectangular band
(with involution) by the symmetric group $\Sim_3$ and $\A_3$ is an
ideal extension of a $3\times 3$ rectangular band (with
involution) by the cyclic group of order $3$ --- in neither case
we know the answer.  So we may formulate
\begin{Problem}
Are the regular $*$-semigroups $P\B_2\cong P\A_2$, $\B_3$, $\A_3$
finitely based?
\end{Problem}

\section{Further applications}
\subsection{Unary Rees matrix semigroups}
We had used in \cite{adv} unary Rees matrix semigroups as a tool in the proof of
Theorem 1.2; in turn, here we shall show
that this theorem allows one to solve the finite
basis problem for a large family of unary Rees matrix semigroups.

An $I\times I$-matrix $P=(p_{ij})$ over $\mathcal{G}\cup\{0\}$,
where $\mathcal{G}$ is a group, is called
\emph{block-diagonalizable} if there exists a partition $\pi$ of
the set $I$ such that $p_{ij}\ne 0$ if and only if $i\m{\pi}j$. If
one defines a graph $\Gamma(P)$ on the set $I$ in which two
distinct vertices $i$ and $j$ are adjacent if and only if
$p_{ij}\ne0$, then it is clear that block-diagonalizable matrices
correspond to graphs whose connected components are cliques (i.e.\
complete graphs). We say that $P$ is $*$-\emph{regular} if for all
$i,j\in I$, one has $p_{ji}=p_{ij}^{-1}$ whenever $p_{ij}\in
\mathcal{G}$ and $p_{ii}=e$, where $e$ is the identity element of
$\mathcal{G}$. (Recall that this property ensures that the unary
\Rm\ \sm\ $\Mc^0(I,\mathcal{G},I;P)$ is a regular $*$-semigroup.)

\begin{Thm}
\label{Rees matrix} Let $P$ be an $I\times I$-matrix over
$\mathcal{G}\cup\{0\}$, where $\mathcal{G}$ is a group. Suppose
that $P$ is $*$-regular and not block-diagonalizable. If
$\mathcal{G}$ does not belong to the group variety
$\var\mathcal{H}$, where $\mathcal{H}$ is the subgroup generated
by the non-zero entries of $P$, then the unary \Rm\ \sm\
$\Mc^0(I,\mathcal{G},I;P)$ is not finitely based.
\end{Thm}

\begin{proof}
Let $P=(p_{ij})$. Since $P$ is not block-diagonalizable, there is
a connected component $C$ in $\Gamma(P)$ which is not a clique.
Let $Q$ be a maximal clique in $C$. As $C$ is connected, there
exist $i_0\in Q$ and $j_0\in C\setminus Q$ such that $i_0$ and
$j_0$ are adjacent. At the same time, there should be a vertex
$k_0\in Q$ such that $j_0$ and $k_0$ are not adjacent ---
otherwise $Q\cup\{j_0\}$ would make a larger clique in $C$. Thus,
the submatrix $P_0$ of $P$ corresponding to the set
$I_0=\{i_0,j_0,k_0\}$ is of the form
$$P_0=\begin{pmatrix}
e      & g & h\\
g^{-1} & e & 0\\
h^{-1} & 0 & e
\end{pmatrix}$$
where $g=p_{i_0j_0},h=p_{i_0k_0}$ belong to $\mathcal{G}$. The
unary \Rm\ \sm\ $\Mc^0(I_0,\mathcal{G},I_0;P_0)$ is then a unary
subsemigroup in $\Mc^0(I,\mathcal{G},I;P)$, and the obvious
homomorphism $\mathcal{G}\to \mathcal{E}$, where
$\mathcal{E}=\{e\}$ is the trivial group, extends to a unary
semigroup homomorphism from $\Mc^0(I_0,\mathcal{G},I_0;P_0)$ onto
$\mathcal{K}_3$. Thus, $\mathcal{K}_3$ belongs to the variety
$\var\Mc^0(I,\mathcal{G},I;P)$.

The Hermitian subsemigroup $\H(\Mc^0(I,\mathcal{G},I;P))$ of
$\Mc^0(I,\mathcal{G},I;P)$ is generated by the elements $(i,e,i)$
where $i$ runs over $I$. This implies that the group coordinates
of triples $(i,g,j)$ in $\H(\Mc^0(I,\mathcal{G},I;P))$ belong to
the subgroup $\mathcal{H}$ generated by the non-zero entries of
$P$. Hence $\H(\Mc^0(I,\mathcal{G},I;P))$ is a unary subsemigroup
of the unary \Rm\ \sm\ $\Mc^0(I,\mathcal{H},I;P)$. It is not hard
to see that each group in the variety
$\var\Mc^0(I,\mathcal{H},I;P)$ belongs to the group variety $\var
\mathcal{H}$. Since $\mathcal{G}$ does not belong to
$\var\mathcal{H}$ but obviously belongs to
$\var\Mc^0(I,\mathcal{G},I;P)$, we are in a position to apply
Theorem~\ref{Theorem 2.1}.
\end{proof}

A comprehensive treatment of the finite basis problem for unary
\Rm\ semigroups forms the subject of a paper by
M.~Jackson and the third author \cite{jacksonvolkov}.

\subsection{Varietal joins}
Recall that the \emph{join} $\mathbf{V}\vee\mathbf{W}$ of two
varieties $\mathbf{V}$ and $\mathbf{W}$ is the least variety
containing both $\mathbf{V}$ and $\mathbf{W}$. We show how
Theorem~\ref{Theorem 2.1} can be used to produce interesting
examples of non-finitely based joins of varieties of unary
semigroups.

Denote by $\mathbf{CSR^*}$ the variety generated by the unary
semigroup $\mathcal{K}_3$ (that is, the variety of all
\emph{combinatorial strict regular $*$-semigroups}, see \cite{A1})
and let $\mathbf{I}$ be the variety of all inverse semigroups.
\begin{Thm}
\label{Theorem 3.2} Let $\pv K$ and $\pv A$ be varieties of unary
semigroups such that:
\begin{enumerate}
\item $\pv K$ contains $\pv{CSR^*}$,
\item $\pv A$ consists of inverse semigroups and contains
a group not contained in $\H(\pv K)$.
\end{enumerate}
Then no variety in the interval $[\pv{CSR^*}\vee \pv A, \pv
K\vee\pv{I}]$ is finitely based.
\end{Thm}

\begin{proof}
Let $\mathcal{S}\in \pv K\vee\pv I$; then there exist
$\mathcal{K}\in \pv K$, $\mathcal{I}\in \pv I$ such that
$\mathcal{S}$ \emph{divides} (that is, $\mathcal{S}$ is a
homomorphic image of a substructure of)
$\mathcal{K}\times\mathcal{I}$ whence $\H(\mathcal{S})$ divides
$\H(\mathcal{K})\times \H(\mathcal{I})$. Observe that
$\H(\mathcal{I})$ is a semilattice (with trivial involution).
Further, since $\H(\mathcal{K}_3)=\mathcal{K}_3$ and $\pv
{CSR^*}\subseteq \pv K$ we have
$\pv{CSR^*}=\H(\pv{CSR^*})\subseteq \H (\pv K)$ so that $\H (\pv
K)$ contains all semilattices with trivial involution since
$\pv{CSR^*}$ does so. Altogether, we have $\H(\mathcal{S})\in
\H(\pv K)$, that is $\H(\pv K\vee\pv I)=\H(\pv K)$, and thus, for
any variety $\pv V$ in the interval $[\pv{CSR^*}\vee \pv A, \pv
K\vee\pv{I}]$, we have $\H(\pv V)\subseteq \H(\pv K)$. By
assumption (2), there exists a group in $\pv A\subseteq \pv V$
that is not in $\H(\pv K)\supseteq \H(\pv V)$. Thus,
Theorem~\ref{Theorem 2.1} applies to the variety~$\pv V$.
\end{proof}

The conditions of Theorem~\ref{Theorem 3.2} are obviously
fulfilled if $\pv{CSR^*}\subseteq \pv K$ and $\pv A$ contains a
group that is not in $\pv K$; so, for example $\pv
K=[x^m=x^{m+n}]$ for fixed $n\ge 1$ and $m\ge 2$ and $\pv A=\pv G$
(the variety of all groups) meet the requirements.

Recall that a variety $\Vc$ of algebraic structures is a
\emph{Cross variety} if
\renewcommand{\labelenumi}{\theenumi)}
\begin{enumerate}
\item $\Vc$ is generated by a finite structure,
\item $\Vc$ contains only finitely many subvarieties,
\item $\Vc$ is finitely based.
\end{enumerate}
For an interesting treatment of Cross varieties of plain
semigroups consult Sapir~\cite{S}. The variety $\mathbf{CSR^*}$ is
a Cross variety, see \cite[Theorems 5.1 and 5.2, Corollary
5.4]{A1}. Now let $\mathbf{A}_p$ denote the variety of all abelian
groups of exponent $p$ ($p$ is a prime number); clearly,
$\mathbf{A}_p$ is a a Cross variety. By \cite[Corollaries 5.4 and
6.5]{A1}, the join $\mathbf{CSR^*}\vee\mathbf{A}_p$ contains only
fourteen subvarieties; however, by the above remark, the join is
not finitely based and therefore is not a Cross variety. We thus
have a simple example of two Cross varieties whose join is not a
Cross variety. A plain semigroup example of this kind found
in~\cite[Corollary 2.1]{S} is much more involved (with 39
subvarieties).

\section{Existence varieties of locally inverse semigroups}
In this section we give an application to existence varieties of
the method of proof of Theorem \ref{Theorem 2.1}. Recall that an
\emph{existence variety} (shortly \emph{e-variety}) of regular
semigroups is a class of regular semigroups closed under taking
direct products, \emph{regular} subsemigroups and homomorphic
images. This section assumes the reader's acquaintance with some
basics of the theory of regular semigroups.

While research into the structure of regular semigroups was
particularly active in the 1970s and early 1980s, a universal
algebra approach for regular semigroups has been introduced at the
end of the 1980s by Ka\softd{}ourek and Szendrei \cite{KS} for
orthodox semigroups, and, independently, by Hall \cite{H1,H2} for
regular semigroups in general. We shall recall the basic
definitions and results necessary to understand the following
treatment. For further information consult the papers
\cite{KS,H1,H2,Y1,A2,A3}.

A regular semigroup $\mathcal{S}=\langle S,\cdot\rangle$ is
\emph{locally inverse} if for each idempotent $e$ of $\mathcal S$,
the \emph{local submonoid} $eSe$ is an inverse semigroup. The
class $L\pv I$ of all (regular) locally inverse semigroups is a
typical example of an existence variety. Observe that \Rm\
semigroups over groups are locally inverse (moreover, in such a
semigroup each local submonoid is a group with 0 adjoined or the
trivial group).

It is known \cite[Theorem 7.6]{N1} that a regular semigroup
$\mathcal S$ is locally inverse if and only if for any two $x,y\in
\mathcal S$ the set $xV(yx)y$ is a singleton (as usual, $V(z)$
denotes the set of all inverses of the element $z$). This gives
rise to the \emph{sandwich operation} $\wedge$ that can be defined
on any locally inverse semigroup by setting $x\wedge y$ to be the
unique element of $xV(yx)y$, so that in this context, locally
inverse semigroups are treated as algebras of type $(2,2)$.

As explained in \cite{A2, A3}, the adequate concept of equational
theory for e-varieties of locally inverse semigroups is based on
the signature $\{\cdot,\wedge\}$ and is with respect to a doubled
alphabet $X\cup X'$. Here $X$ is, as usual, a countably infinite
set of variables and $X'=\{x'\mid x\in X\}$ is a disjoint copy of
$X$; the elements of $X'$ are devoted to represent inverses of the
elements which are represented by the elements of $X$. The terms
are over this extended alphabet and are in the signature
$\{\cdot,\wedge\}$ where $\cdot$ stands for the associative
operation of multiplication and $\wedge$ for the sandwich
operation. Given a term $w(x_1,\dots,x_n,x'_1,\dots,x'_n)$ of this
kind, a value of that term in the locally inverse semigroup
$\mathcal S$ is obtained by substituting the variables $x_i,x'_i$
by elements $s_i,s'_i$ in $\mathcal S$ in such a way that each
$s'_i$ is an inverse of $s_i$. (In this evaluation it definitely
may happen that distinct variables $x,y$, say, will be substituted
with the same element $s$, while the formal inverses $x'$ and
$y'$, respectively, are substituted with distinct inverses
$s^\sharp$ and $s^\flat$, say, of $s$.) Given this notion of
evaluation of terms in a locally inverse semigroup, it is clear
what it means that a locally inverse semigroup $\mathcal S$
\emph{satisfies} a \emph{bi-identity} $u=v$ of terms of that kind.
The following Birkhoff type theorem then holds \cite{A2}.

\begin{Thm}\label{Theorem 4.1} A class $\pv V$ of locally inverse semigroups is an
e-variety if and only if it is definable by bi-identities, that
is, $\pv V$ consists of all locally inverse semigroups that
satisfy a certain set of bi-identities.\end{Thm}

Now call a set $B$ of bi-identities a \emph{basis} of $\pv V$ if
 a locally inverse semigroup $\mathcal S$ is a member
of $\pv V$ if and only if $\mathcal S$ satisfies all bi-identities
of $B$. This semantic notion of basis is equivalent to a syntactic
one: $B$ is a basis of  $\pv V$ if and only if $B$ axiomatizes the
\emph{bi-equational theory} which is the set of all bi-identities
over a (fixed) countable infinite set $X$ of variables satisfied
by all members of $\pv V$. The latter means that  each bi-identity
satisfied by all members of $\pv V$ can be derived, using natural
deduction rules, from the bi-identities of $B\cup B(L\pv I)$ where
$B(L\pv I)$ is a basis for the bi-equational theory of the class
of all locally inverse semigroups. A set consisting of four
independent bi-identities which may serve as $B(L\pv I)$ has been
found in \cite{A3}. For more analogues between the theory of
e-varieties of regular semigroups and varieties of universal
algebras see \cite{A2,A3,KS,Y1}.

The objective of this section is to obtain an analogue of
Theorem~\ref{Theorem 2.1} giving a sufficient condition for an
e-variety $\pv V$ of locally inverse semigroups to have no finite
basis of bi-identities. Let $\mathcal S$ be a locally inverse
semigroup and $a_1,\dots,a_k\in \mathcal S$. For each $i$ take an
element $a'_i\in V(a_i)$. Then the closure of the set
$\{a_1,\dots,a_k,a'_1,\dots,a'_k\}$ under multiplication and
sandwich operation  is a locally inverse subsemigroup of $\mathcal
S$, and is the least locally inverse subsemigroup of $\mathcal S$
containing the set $\{a_1,\dots,a_k,a'_1,\dots,a'_k\}$ (by
\cite{Y1}). We call such a subsemigroup a {\it $k$-generated}
locally inverse subsemigroup of $\mathcal S$. Define the {\it
content} $c(t)$ of a term $t$ inductively by $c(x)=c(x')=\{x\}$
and $c(uv)=c(u\we v)=c(u)\cup c(v)$. In order to prove that an
e-variety $\pv V$ has no a finite basis of bi-identities it is
sufficient to prove for each natural number $k$ the existence of a
locally inverse semigroup $\mathcal{T}_k$ such that
$\mathcal{T}_k\not\in \pv V$ but $\mathcal{T}_k$ satisfies each
bi-identity $u=v$ that holds in $\pv V$ and for which $\vert
c(u)\cup c(v)\vert \le k$. The latter is equivalent to the
property that each $k$-generated locally inverse subsemigroup (as
defined above) is contained in~$\pv V$.

For each e-variety $\pv V$ denote by $\Co(\pv V)$ the
sub-e-variety of $\pv V$ generated by all idempotent generated
members of $\Vc$. We also need the 5-element \Rm\ semigroup
$\mathcal{A}_2$ with the sandwich matrix
\begin{equation}
\label{matrix for TA}
\begin{pmatrix}
0 & e\\
e & e
\end{pmatrix}.
\end{equation} We
are ready to formulate an e-variety analogue of Theorem
\ref{Theorem 2.1}.

\begin{Thm}\label{Theorem 4.4} Let $\pv V$ be a locally inverse
e-variety containing the semigroup $\mathcal{A}_2$. If $\pv V$
contains a group which is not in $\Co(\pv V)$ then $\pv V$ has no
finite basis for its bi-identities.
\end{Thm}

\begin{proof} This can be proved in a manner similar to the proof of
Theorem 2.2 in \cite{adv}.  As mentioned above, we have to prove,
for each $k$, the existence of a locally inverse semigroup
$\mathcal{T}_k$ such that $\mathcal{T}_k\notin \pv V$ but each
$k$-generated locally inverse subsemigroup of $\mathcal{T}_k$ is
contained in $\pv V$.

Let $\mathcal G$ be a group in $\pv V$ that is not contained in
$\Co(\pv V)$. Since there must be a bi-identity which holds in
$\Co(\pv V)$ but fails in $\mathcal G$, we may assume that
$\mathcal G$ is generated by finitely many elements, say
$g_1,\dots, g_m$, and for convenience we may assume that this set
of generators is closed under taking inverse elements and so
generates $\mathcal{G}$ as a semigroup. Next, let $\mathcal{T}_k$
be the \Rm\ semigroup in the proof of \cite[Theorem 2.2]{adv},
but with $n=2k+1$ being replaced with $n=4k+1$. In that proof it
has been shown that $(1,g_j,mn)\in \left<E(\mathcal{T}_k)\right>$
for $j=1,\dots, m$ (here $\langle E(\mathcal{T}_k)\rangle$ denotes
the idempotent generated subsemigroup of $\mathcal{T}_k$). It
follows that $\{1\}\times \mathcal{G}\times\{mn\}\subseteq
\left<E(\mathcal{T}_k)\right>$ whence
$\left<E(\mathcal{T}_k)\right>$ contains a subgroup isomorphic to
$\mathcal{G}$. Consequently, $\left<E(\mathcal{T}_k)\right>\notin
\Co(\pv V)$ which implies that $\mathcal{T}_k\notin \pv V$.

Finally, consider any $k$-generated locally inverse subsemigroup
of $\mathcal{T}_k$, that is, choose
$a_1,\dots,a_k,a'_1,\dots,a'_k\in \mathcal{T}_k$ such that
$a'_i\in V(a_i)$ for each $i$. Let $\mathcal{T}$ be the locally
inverse subsemigroup generated by
$\{a_1,\dots,a_k,a'_1,\dots,a'_k\}$ (that is, the closure of that
set under multiplication and sandwich operation). It is clear that
at most $4k$ indices of $\{1,\dots,nm\}$ can occur in the triple
representation of the elements $a_i,a'_i$. Therefore, analogously
to the unary case proved in \cite[Theorem 2.2 ??]{adv}, there exist numbers
$\la_1,\dots,\la_m$ such that
$$1\le\la_1\le n<\la_2\le 2n < \dots (m-1)n<\la_m\le mn$$
and $\mathcal{T}$ is contained in the semigroup
$\mathcal{T}_k(\la_1,\dots,\la_m)$. Again as in
Section 1, we can show that $\mathcal{T}_k(\la_1,\dots,\la_m)$ is
isomorphic to a homomorphic image of the direct product
$\mathcal{G}\times \mathcal{U}_k$ of the group $\mathcal G$ and a
completely 0-simple semigroup $\mathcal{U}_k$ with trivial
subgroups. Now $\mathcal{G}\in \pv V$ by our assumption and
$\mathcal{U}_k\in \pv V$ by a result of Hall \cite{H2} because
$\pv V$ contains $\mathcal{A}_2$. This completes the proof.
\end{proof}

Theorem \ref{Theorem 4.4} can be in particular applied to certain
joins of e-varieties. Here is an example. Denote by $\pv{CSR}$ the
e-variety generated by the semigroup $\mathcal{A}_2$  and by $\pv
{GI}$ the e-variety of all orthodox locally inverse semigroups
(these semigroups are often called \emph{generalized inverse}).
The proof of the next corollary is analogous to that of
Theorem~\ref{Theorem 3.2} and is left to the reader.

\begin{Cor} \label{Corollary 4.6} Let $\pv K$ and $\pv A$ be locally inverse
e-varieties such that:
\renewcommand{\labelenumi}{(\theenumi)}
\begin{enumerate}
\item $\pv K$ contains $\pv{CSR}$,
\item $\pv A$ consists of orthodox semigroups and contains a group
not contained in $\Co(\pv K)$.
\end{enumerate}
Then no e-variety in the interval $[\pv{CSR}\vee \pv A, \pv
K\vee\pv {GI}]$ is finitely based.
\end{Cor}

\noindent\textbf{Acknowledgement.} The second author was supported
by Grant No.\ 144011 of the Ministry of Science and Technological
Development of the Republic of Serbia.

\begin{thebibliography}{99}
\bibitem{Alm95}
J.\,Almeida, \emph{Finite Semigroups and Universal Algebra}, World
Scientific, Singapore, 1995.

%\bibitem{AM}
%J.\,Ara\'ujo and J.\,D.\,Mitchell, \emph{An elementary proof that
%every singular matrix is a product of idempotent matrices}, Amer.\
%Math.\ Monthly \textbf{112} (2005), 641--645.

\bibitem{A1}
K.\,Auinger, \emph{Strict regular $*$-semigroups}, pp.190--204 in:
Proceedings of the Conference on Semigroups with Applications, J.
M. Howie, W. D. Munn and H.-J. Weinert (eds.), World Scientific,
Singapore, 1992.

\bibitem{A2}
K.\,Auinger, \emph{The bifree locally inverse semigroup on a set},
J. Algebra \textbf{ 166} (1994), 630--650.

\bibitem{A3}
K.\,Auinger, \emph{A system of bi-identities for locally inverse
semigroups}, Proc.\ Amer.\ Math.\ Soc. \textbf{ 123} (1995),
979--988.

\bibitem{adv} K.\,Auinger, I.\,Dolinka and M.\,V.\,Volkov \emph{Matrix indentities involving multiplication and transposition}, preprint.

\bibitem{brauer}
R.\,Brauer, \emph{On algebras which are connected with the
semisimple continuous groups}, Ann.\ Math. \textbf{38} (1937),
857--872.

%\bibitem{BIG}
%A. Ben-Israel and Th.\ Greville, \emph{Generalized Inverses:
%Theory and Applications}, Springer-Verlag, Berlin--Heidelberg--New
%York, 2003.

\bibitem{BuSa81}
S. Burris and H. P. Sankappanavar, \emph{A Course in Universal
Algebra}, Springer-Verlag, Berlin--Heidelberg--New York, 1981.

\bibitem{CP}
A.\,H.\,Clifford and G.\,B.\,Preston, \emph{The Algebraic Theory
of Semigroups I}, Amer.\ Math.\ Soc., Providence, 1961.

%\bibitem{Dolinka}
%I. Dolinka, \emph{On identities of finite involution semigroups},
%Semigroup Forum, to appear.

%\bibitem{D}
%M.\,P.\,Drazin, \emph{Regular semigroups with involution},
%pp.29--46 in: Proceedings of the Symposium on Regular semigroups,
%Northern Illinois University, De Kalb, 1979.

%\bibitem{Erdos}
%J.\,A.\,Erdos, \emph{On products of idempotent matrices}, Glasgow
%Math.\ J. \textbf{8} (1967), 118--122.

\bibitem{grahamlehrer}
J.\,J.\,Graham and G.\,I.\,Lehrer, \emph{Cellular algebras},
Invent.\ Math. \textbf{123} (1996), 1--34.

\bibitem{H1}
T.\,E.\,Hall, \emph{Identities for existence varieties of regular
semigroups}, Bull. Austral. Math. Soc. \textbf{ 40} (1989),
59--77.

\bibitem{H2}
T.\,E.\,Hall, \emph{Regular semigroups: amalgamation and the
lattice of existence varieties}, Algebra Universalis \textbf{ 29}
(1991), 79--108.

\bibitem{jacksonvolkov} M.\,Jackson and M.\,V.\,Volkov, \emph{The algebra of adjacency patterns: Rees matrix semigroups with reversion}, in A. Blass, N. Dershowitz, W. Reisig (eds.), Fields of Logic and Computation, Essays Dedicated to Yuri Gurevich on the Occasion of His 70th Birthday [Lect. Notes Comp. Sci., 6300], Springer-Verlag, Berlin-Heidelberg-N.Y. 2010, 414-443.

\bibitem{jones}
V.\,F.\,R.\,Jones, \emph{A quotient of the affine Hecke algebra in
the Brauer algebra}, Enseign.\ Math. II. S\'er. \textbf{40}
(1994), 313--344.

\bibitem{KS}
J.\,Ka\softd{}ourek and M.\,B.\,Szendrei, \emph{A new approach in
the theory of orthodox semigroups}, Semigroup Forum \textbf{ 40}
(1990), 257--296.

\bibitem{KR}
K.\,H.\,Kim and F.\,Roush, \emph{On groups in varieties of
semigroups}, Semigroup Forum \textbf{ 16} (1978), 201--202.

%\bibitem{kleiman}
%E.\,I.\,Kleiman, \emph{Bases of identities of varieties of inverse
%semigroups}, Sibirsk.\ Mat.\ Zh. \textbf{20} (1979), 760--777
%[Russian; English transl.\ Sib.\ Math. J. \textbf{20}, 530--543].

\bibitem{KMM}
G.\,Kudryavtseva, V.\,Maltcev and V.\,Mazorchuk, \emph{${\mathcal
L}$- and $\mathcal R$-cross-sections in the Brauer semigroup},
Semigroup Forum \textbf{72} (2006), 223--248.

\bibitem{KM2}
G.\,Kudryavtseva and V.\,Mazorchuk, \emph{On presentation of
Brauer-type monoids}, Central Europ.\ J. Math. \textbf{4} (2006),
413--434.

%\bibitem{LidlNiederreiter}
%R.\,Lidl and H.\,Niederreiter, \emph{Finite Fields},
%Addison-Wesley, Cambridge, 1997.

%\bibitem{mks}
%W.\,Magnus, A.\,Karras and D.\,Solitar, \emph{Combinatorial Group
%Theory}, Wiley, New York--London--Singapore, 1966.

\bibitem{malcevmazorchuk}
V.\,Maltcev and V.\,Mazorchuk, \emph{Presentation of the singular
part of the Brauer monoid}, Math.\ Bohem. \textbf{132} (2007),
297--323.

%\bibitem{margolissapir}
%S.\,W.\,Margolis and M.\,V.\,Sapir, \emph{Quasi-identities of
%finite semigroups and symbolic dynamics}, Israel J. Math.
%\textbf{92} (1995), 317--331.

\bibitem{Maz1}
V.\,Mazorchuk, \emph{On the structure of Brauer semigroups and its
partial analogue}, Problems in Algebra \textbf{13} (1998), 29--45.

\bibitem{Maz2}
V.\,Mazorchuk, \emph{Endomorphisms of $\B_n$, ${\mathcal P}\B_n$
and $\C_n$}, Comm.\ Algebra \textbf{30} (2002), 3489--3513.

%\bibitem{Meyer}
%C.\,D.\,Meyer, \emph{Matrix Analysis and Applied Linear Algebra},
%SIAM, Philadelphia, 2000.

%\bibitem{moore}
%E.\,H.\,Moore, \emph{On the reciprocal of the general  algebraic
%matrix}, Bull.\ Amer.\ Math.\ Soc. \textbf{26} (1920),  394--395.

\bibitem{N1}
K.\,S.\,S.\,Nambooripad, \emph{Structure of regular semigroups I},
Mem.\ Amer.\ Math.\ Soc. \textbf{224} (1979).

\bibitem{N2}
K.\,S.\,S.\,Nambooripad, \emph{The natural partial order on a
regular semigroup}, Proc.\ Edinburgh Math. Soc. \textbf{32}
(1980), 249--260.

\bibitem{Ne}
H.\,Neumann, \emph{Varieties of Groups}, Springer-Verlag,
Berlin--Heidelberg--New York, 1967.

%\bibitem{oatespowell}
%S.\,Oates and M.\,B.\, Powell, \emph{Identical relations in finite
%groups}, J. Algebra  \textbf{1} (1964), 11--39.

%\bibitem{P}
%R.\,Penrose, \emph{A generalized inverse for matrices}, Proc.\
%Cambridge Phil.\ Soc. \textbf{51} (1955), 406--413.

\bibitem{Per89}
P.\,Perkins, \emph{Finite axiomatizability for equational theories
of computable groupoids}, J. Symbolic Logic \textbf{54} (1989),
1018--1022.

%\bibitem{Procesi}
%C.\,Procesi, \emph{Lie Groups: an Approach through Invariants and
%Representations}, Springer-Verlag, Berlin--Heidelberg--New York,
%2006.

\bibitem{sapirinherently}
M.\,V.\,Sapir, \emph{Inherently non-finitely based finite
semigroups}, Mat.\ Sb. \textbf{133}, no.2 (1987), 154--166
[Russian; English transl.\ Math. USSR-Sb. \textbf{61} (1988),
155--166].

\bibitem{sapirburnside}
M.\,V.\,Sapir, \emph{Problems of Burnside type and the finite
basis property in varieties of semigroups}, Izv.\ Akad.\ Nauk
SSSR, Ser.\ Mat. \textbf{51} (1987), 319--340 [Russian; English
transl.\ Math.\ USSR-Izv. \textbf{30} (1987), 295--314].

\bibitem{S}
M.\,V.\,Sapir, \emph{On Cross semigroup varieties and related
questions}, Semigroup Forum \textbf{ 42} (1991), 345--364.

%\bibitem{sapirinverse}
%M.\,V.\,Sapir, \emph{Identities of finite inverse semigroups},
%Internat.\ J. Algebra Comput. \textbf{3} (1993), 115--124.

%\bibitem{Sapir}
%M. V. Sapir, \emph{Combinatorics on words with applications},
%IBP-Litp 1995/32: Rapport de Recherche Litp, Universit\'e Paris 7,
%1995 (available online under\\
%\url{http://www.math.vanderbilt.edu/~msapir/ftp/course/course.pdf}).

%\bibitem{Serre}
%J.-P. Serre, \emph{Cours d'Arithmetique}, Presses Universitaires
%de France, Paris, 1980.

\bibitem{V}
M.\,V.\,Volkov, \emph{On finite basedness of semigroup varieties},
Mat.\  Zametki \textbf{ 45}, no.3 (1989),  12--23 [Russian;
English transl.\ Math.\ Notes \textbf{ 45} (1989), 187--194].

\bibitem{volkovjaponicae}
M.\,V.\,Volkov, \emph{The finite basis problem for finite
semigroups}, Sci.\ Math.\ Jpn. \textbf{53} (2001), 171--199.

\bibitem{xi}
C.\,Xi, \emph{Partition algebras are cellular}, Comp.\ Math.
\textbf{119} (1999), 99--109.

\bibitem{Y1}
Y.\,T.\,Yeh, \emph{The existence of $e$-free objects in
e-varieties of regular semigroups}, Internat.\ J. Algebra Comput.
\textbf{ 2} (1992), 471--484.
\end{thebibliography}
\end{document}
