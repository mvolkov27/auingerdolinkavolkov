\documentclass[11pt]{article}

\usepackage{amsmath,amsthm,amssymb}
\usepackage{url}
\usepackage{longtable}
\usepackage{amsfonts}
\usepackage{eepic}
\usepackage{amsgen}
\usepackage{eucal}
\usepackage{times}
\usepackage{hhline}
\usepackage{pstricks,pst-node,pst-text,pst-3d}
\usepackage{color}

\DeclareMathOperator{\var}{\mathsf{var}}

\def\cal{\mathcal}
\def\Ac{{\cal A}}
\def\Bc{{\cal B}}
\def\Cc{{\cal C}}
\def\Dc{\mathrsfs{D}}
\def\Ec{{\cal E}}
\def\Fc{{\cal F}}
\def\Gc{{\cal G}}
\def\Hc{\mathrsfs{H}}
\def\Ic{{\cal I}}
\def\Jc{{\cal J}}
\def\Kc{{\cal K}}
\def\Lc{\mathrsfs{L}}
\def\Mc{{\cal M}}
\def\Nc{{\cal N}}
\def\Oc{{\cal O}}
\def\Pc{{\cal P}}
\def\Qc{{\cal Q}}
\def\Rc{\mathrsfs{R}}
\def\Sc{{\cal S}}
\def\Tc{{\cal T}}
\def\Uc{{\cal U}}
\def\Vc{\mathbf{V}}
\def\Wc{\mathbf{W}}
\def\Xc{{\cal X}}
\def\Yc{{\cal Y}}
\def\Zc{{\cal Z}}

\def\H{\mathrm H}
\def\P{\mathrm P}

\textwidth=18.71cm
\oddsidemargin=-1.5cm
\parindent=0pt

\begin{document}

\setlongtables
\begin{longtable}{|p{2.2cm}|p{1.8cm}|p{4.2cm}|p{4.2cm}|p{4.2cm}|}
\caption*{\textbf{Proof reading for ``Matrix identities involving multiplication and transposition'' by Auinger et al}}\\
\hline
\textbf{Location} & \textbf{Type} & \textbf{In the proofs} & \textbf{In the original} & \textbf{Should be} \\
\hhline{|=|=|=|=|=|}
\endfirsthead
\hline
\multicolumn{5}{|l|}{\slshape continued from previous page}\\
\hline
\textbf{Location} & \textbf{Type} & \textbf{In the proofs} & \textbf{In the original} & \textbf{Should be} \\
\hhline{|=|=|=|=|=|}
\endhead
\hline
\multicolumn{5}{|r|}{\slshape continued on next page}\\
\hline
\endfoot
\hline
\endlastfoot
Throughout the text & Editor's intervention & non{\red-}finitely based &
nonfinitely based & As in the proofs

(we accept the change)\\
\hline
P.1, Background and Motivation, line~+9 & Editor's intervention & \dots much attention as well{\red :} see, for instance \dots &
\dots much attention as well, see, for instance \dots & As in the proofs

(we accept the change)\\
\hline
P.1, footnote, line~+4 & Update & 21000 & 21000 & 21101\\
\hline
P.1, footnote, line~+5 & Update & Faculty of Mathematics and Mechanics, Ural State University &
Faculty of Mathematics and Mechanics, Ural State University & Institute of Mathematics and Computer Science, Ural Federal University\\
\hline
P.1, footnote, line~+6 & Update & 620083 & 620083 & 620000\\
\hline
P.2, line~+21 & Typo (our fault) & \dots may be {\red a} summarized \dots &
\dots may be {\red a} summarized \dots & \dots may be summarized \dots\\
\hline
P.2, Theorem, line~+1 & Editor's intervention & \rule{0pt}{1pt}{\red None} of {\red the} following sets of matrix identities admits {\red a} finite identity basis: &
Each of following sets of matrix identities admits no finite identity basis: & As in the proofs

(we accept the change)\\
\hline
P.2, Theorem, lines +2, +4, +6, +9 (4 times) & Editor's intervention & the identities {\red for} \dots &
the identities of \dots & As in the proofs

(we accept the change)\\
\hline
P.3, line +5 & Editor's intervention & $\langle$displayed formula$\rangle$ & $\langle$inline formula$\rangle$
& As in the proofs

(we accept the change)\\
\hline
P.3, line~+18 & Editor's intervention & \dots then so is $u^*$. & \dots then so is $(u)^*$. & As in the original
(we  \textbf{\red do not} accept the change)\\
\hline
P.3, line~+20 & Editor's intervention &  $u\mapsto u^*$. & $u\mapsto (u)^*$. & As in the original
(we \textbf{\red do not} accept the change)\\
\hline
P.3, line~$-$3 & Typo (our fault) & A variety is {\red is} said to be \dots &
A variety is {\red is} said to be \dots & A variety is said to be \dots\\
\hline

P.4, lines 1--2 & Editor's intervention & \dots forming direct products {\red and} taking unary
subsemigroups \dots & \dots forming direct products, taking unary subsemigroups \dots
& As in the proofs

(we accept the change)\\
\hline
P.4, line $-$2 & Editor's intervention & if $p_{jk}=0$\red{,} & if $p_{jk}=0$;
& As in the proofs

(we accept the change)\\
\hline
P.5, line~+4 & Editor's intervention & If the group $\mathcal G$ involved &
If the involved group $\mathcal G$ & As in the proofs

(we accept the change)\\
\hline
P.5, display (1.1) & Editor's intervention & otherwise\red{,} & otherwise; & As in the proofs

(we accept the change)\\
\hline
P.5, line~+1 after display (1.1) & Editor's intervention & \dots semigroup that will be quite
useful is \dots & \dots semigroup that will be quite useful in the sequel is \dots & As in the proofs

(we accept the change)\\
\hline
P.5, line~$-$5 & Editor's intervention & \dots has dimension $n-1${\red,} whence \dots & \dots has dimension
$n-1$ whence \dots & As in the proofs

(we accept the change)\\
\hline
P.6, line~$-$16 & Editor's intervention & The following easy observation will be useful as it helps \dots &
The following easy observation will be useful in the sequel as it helps \dots & As in the proofs

(we accept the change)\\
\hline
P.6, line~$-$8 & Editor's intervention & $\H(\Tc)\in\var\H(\Sc)$, and so $\H(\var\Sc)\subseteq\var\H(\Sc)$. &
$\H(\Tc)\in\var\H(\Sc)$. Since this holds for an arbitrary $\Tc\in\var\Sc$, we conclude that $\H(\var\Sc)\subseteq\var\H(\Sc)$.
& As in the proofs

(we accept the change)\\
\hline
P.7, line~+1 & Editor's intervention & \dots there exists a group $\mathcal{G}\in\Vc\setminus\H(\Vc)$ &
\dots there exists a group $\mathcal{G}\in\Vc$ for which $\mathcal{G}\notin\H(\Vc)$.
& As in the proofs

(we accept the change)\\
\hline
P.7, line~$-$10 & Editor's intervention & denotes the $n\times n$-matrix & denotes the $n\times n$-matrix of the form
& As in the proofs

(we accept the change)\\
\hline
P.7, matrix $M_n(g)$, entry (4,4) & Editor's intervention & \rule{0pt}{1pt}{\red $\vdots$} (produced by \verb+\vdots+) &
$\ddots$ (produced by \verb+\ddots+) &
As in the original (we \textbf{\red do not} accept the change)\\
\hline
P.7, line~$-$8 & Editor's intervention & (This construction is in a sense a combination of those of [3] and [53].) &
(This construction is in a sense a combination of those of the first and the third authors' papers [3] and [53].)
& As in the proofs

(we accept the change)\\
\hline
P.8, line +6 & Overfull &\multicolumn{3}{c|}{The row of dots is too long}\\
\hline
P.8, line~$-$9 & Editor's intervention & \rule{0pt}{1pt}{\red As} $2k<n$ according to \dots & Using that $2k<n$ according to \dots
& As in the proofs

(we accept the change)\\
\hline
P.9, line~+3 & Editor's intervention & For each $i$ {\red with} \dots & For each $i$ such that \dots
& As in the proofs

(we accept the change)\\
\hline
P.10, line~+11 & Editor's intervention & such that $\mathcal{G}\in\Vc\setminus\P_d(\Vc)$ \dots &
such that $\mathcal{G}\in\Vc$ but $\mathcal{G}\notin\P_d(\Vc)$ \dots
& As in the proofs

(we accept the change)\\
\hline
P.10, line~+18 & Editor's intervention & These words have already been used \dots &
These words already have been used \dots
& As in the proofs

(we accept the change)\\
\hline
P.10, line~$-$14 & Editor's intervention & Let $x_1,x_2,\dots$ be a sequence of letters. &
Let $x_1,x_2,\dots,x_n,\dots$ be a sequence of letters.
& As in the proofs

(we accept the change)\\
\hline
P.10, line~$-$8 & Editor's intervention & \rule{0pt}{1pt}{\red Aiming at a} contradiction, suppose \dots &
Arguing by contradiction, suppose \dots
& As in the proofs

(we accept the change)\\
\hline
P.11, line~+2 & Editor's intervention & \dots in {\red Fig. 1 (left)}.  &
\dots shown in the left hand part of Fig.~1
& As in the proofs

(we accept the change)\\
\hline
P.11, lines~2--3 & Editor's intervention & All odd{\red-numbered} columns \dots  &
All odd columns \dots
& As in the proofs

(we accept the change)\\
\hline
P.11, line~+4 & Editor's intervention & All even{\red-numbered} columns &
All even columns & As in the proofs

(we accept the change)\\
\hline
P.11, line~+5 & Editor's intervention & \dots to
${\red (1,2,\ldots,r,\ldots,1,2,\ldots,r)^t}$ {\red where} the block $1,2,\dots,r$ occurs
$r$ times.

& \dots to the transpose of the row
$(1,2,\ldots,r,\ldots,1,2,\ldots,r)$ in which the block $1,2,\dots,r$ occurs
$r$ times.
& We \textbf{\red do not} accept the change in the proposed form. The notation $(\dots)^t$
for the transpose is inconsistent with the notation elsewhere in the paper. We suggest:

\dots to the transpose of $(1,2,\ldots,r,\ldots,1,2,\ldots,r)$ where the block $1,2,\dots,r$ occurs
$r$ times.\\
\hline
P.11, line~+8 & Editor's intervention & (shown in {\red Fig. 1, right})  &
(shown in the right hand part of Fig.~1)
& As in the proofs

(we accept the change)\\
\hline
P.11, line~+11 & Editor's intervention & Let $v_{t}$ be the word in the $t^{\mathrm{th}}$ row of $M_A$. &
Let $v_{t}$ be the word in the $t^{\mathrm{th}}$ row of the matrix $M_A$.
& As in the proofs

(we accept the change)\\
\end{longtable}
\end{document}
