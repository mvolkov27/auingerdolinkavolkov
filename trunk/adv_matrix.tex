%%%%%%%%%%%%%%%%%%%%%%%%%%%%%%%%%%%%%%%%%%%%%%%%%%%%%%%%%
% Part 1 (of 2) of a new version obtained by splitting  %
% the version of 25/04/2009 into two independent papers %
%%%%%%%%%%%%%%%%%%%%%%%%%%%%%%%%%%%%%%%%%%%%%%%%%%%%%%%%%

%%%%%%%%%%%%%%%%%%%%%%%%%%%%%
% Version of March 31, 2010 %
%%%%%%%%%%%%%%%%%%%%%%%%%%%%%

\documentclass[11pt,reqno]{amsart}

\usepackage{cite}
\usepackage{url}
\usepackage{amssymb}
\usepackage{amscd}
\usepackage{amsfonts}
\usepackage{amsmath}
\usepackage{eepic}
%\usepackage{gastex}
\usepackage{rotating}
\usepackage{amsthm}
\usepackage{amsgen}
\usepackage{amsmath}
\usepackage[dvips]{color}
\usepackage{eucal}

\DeclareMathOperator{\dom}{dom} \DeclareMathOperator{\ran}{ran}
\DeclareMathOperator{\rk}{rk} \DeclareMathOperator{\tr}{tr}
\DeclareMathOperator{\var}{\mathsf{var}}
\DeclareMathOperator{\Id}{Eq}

\DeclareSymbolFont{rsfscript}{OMS}{rsfs}{m}{n}
\DeclareSymbolFontAlphabet{\mathrsfs}{rsfscript}

\numberwithin{equation}{section}

\def\bb{\mathbb}

\newtheorem{Thm}{Theorem}[section]
\newtheorem{Prop}[Thm]{Proposition}
\newtheorem{Lemma}[Thm]{Lemma}
\newtheorem{Def}{Definition}[section]
\newtheorem{Res}[Thm]{Result}
\newtheorem{Cor}[Thm]{Corollary}
\newtheorem{Example}[Thm]{Example}
\newtheorem{Examples}[Thm]{Examples}
\newtheorem{Conjecture}[Thm]{Conjecture}
\newtheorem{Not}[Thm]{Notation}

\theoremstyle{remark}
\newtheorem{Rmk}{Remark}[section]
\newtheorem{Problem}{Problem}[section]

\def\eq{\simeq}
\def\La{\Lambda}
\def\om{\omega}
\def\cd{\cdot}
\def\wh{\widehat}
\def\pv#1{{\bf #1}}
\def\cal{\mathcal}
\def\Ac{{\cal A}}
\def\Bc{{\cal B}}
\def\Cc{{\cal C}}
\def\Dc{\mathrsfs{D}}
\def\Ec{{\cal E}}
\def\Fc{{\cal F}}
\def\Gc{{\cal G}}
\def\Hc{\mathrsfs{H}}
\def\Ic{{\cal I}}
\def\Jc{{\cal J}}
\def\Kc{{\cal K}}
\def\Lc{\mathrsfs{L}}
\def\Mc{{\cal M}}
\def\Nc{{\cal N}}
\def\Oc{{\cal O}}
\def\Pc{{\cal P}}
\def\Qc{{\cal Q}}
\def\Rc{\mathrsfs{R}}
\def\Sc{{\cal S}}
\def\Tc{{\cal T}}
\def\Uc{{\cal U}}
\def\Vc{\mathbf{V}}
\def\Wc{\mathbf{W}}
\def\Xc{{\cal X}}
\def\Yc{{\cal Y}}
\def\Zc{{\cal Z}}
\def\es{{\Ec\Sc}}
\def\cs{{\Cc\Sc}}
\def\si{\sigma}
\def\Si{\Sigma}
\def\al{\alpha}
\def\ga{\gamma}
\def\de{\delta}
\def\ep{\varepsilon}
\def\be{\beta}
\def\la{\lambda}
\def\te{\theta}
\def\ka{\kappa}
\def\rh{\rho}
\def\ta{\tau_\alpha}
\def\m{\mathrel}
\def\ol{\overline}
\def\mo{\models}
\def\s{\sigma}
\def\De{\Delta}
\def\ze{\zeta}
\def\io{\iota}
\def\fx{F(X)}
\def\ka{\kappa}
\def\we{\wedge}
\def\bp{\bar\phi}
\def\bps{\bar\psi}
\def\Co{{\mathrm C}}
\def\Ko{\Cal K}
\def\H{\mathrm H}
\def\P{\mathrm P\!}
\def\he#1{#1#1^*}
\def\re{regular $*$-}
\def\wi{weakly invertible}
\def\ig{idempotent-generated}
\def\Rm{Rees matrix}
\def\sm{semi\-group}
\def\va{variet}
\def\evar{\mathsf{evar}}
\def\A{\mathfrak{A}}
\def\C{\mathfrak{C}}
\def\B{\mathfrak{B}}
\def\Sim{\mathfrak{S}}
\def\ov{\overline}
\def\inv{^{-1}}
\def\wt{\widetilde}
\def\id{identit}
\def\fb{finitely based}
\def\TB{\ensuremath{\mathcal{T\kern-1pt B}_2^1}}
\def\TA{\ensuremath{\mathcal{T\kern-1pt A}_2^1}}

%%%%%%%%%%%%%%%%%%%
% TITLE & AUTHORS %
%%%%%%%%%%%%%%%%%%%

\title[Matrix Semigroups with Unary Operations]{Equational Theories of Matrix Semigroups\\ with Unary Operations}

\author{K.~Auinger}
\address{Fakult\"at f\"ur Mathematik, Universit\"at Wien, Nordbergstrasse 15,  A-1090 Wien, Austria}
\email{karl.auinger@univie.ac.at}
%
\author{I.~Dolinka}
\address{Department of Mathematics and Informatics, University of Novi Sad, Trg Dositeja Obradovi\'ca 4, 21000 Novi Sad, Serbia}
\email{dockie@dmi.uns.ac.rs}
%
\author {M.~V.~Volkov}
\address{Faculty of Mathematics and Mechanics, Ural State University, Lenina 51, 620083 Ekaterinburg, Russia}
\email{mikhail.volkov@usu.ru}

%%%%%%%%%%%%%%%%%%%%%%%%%
% THE PAPER STARTS HERE %
%%%%%%%%%%%%%%%%%%%%%%%%%

\begin{document}

\begin{abstract}
We study equational theories of matrices in the language involving multiplication and unary operations such as
transposition or Moore-Penrose inversion. We prove that in many cases such theories are not finitely axiomatizable.
\end{abstract}

\maketitle

\section{Background and motivation}
A fundamental and widely studied question connected with an algebraic structure is whether its equational theory is
finitely axiomatizable. This question is often referred to as the \emph{finite basis problem} because structures for
which it is answered in the affirmative are usually said to be \emph{finitely based}. Being very natural by itself, the
finite basis problem has also revealed a number of interesting and unexpected relations to many issues of theoretical
and practical importance ranging from feasible algorithms for membership in certain classes of formal languages (see
\cite{Alm95}) to classical number-theoretic conjectures (such as the Twin Prime, Goldbach, existence of odd perfect
numbers and the infinitude of even perfect numbers: see \cite{Per89} where it is shown that each of these conjectures
is equivalent to the finite basability of a particular groupoid).

Since matrices play a distinguished role in mathematics, algebraic structures whose carriers are sets of matrices form
important and popular objects of study. In particular, the finite basis problem has been exhaustively investigated for
the semigroup $\langle\mathrm{M}_n(\mathcal{K}),\cdot\rangle$ of all $n\times n$-matrices ($n\ge2$) over a field
$\mathcal{K}$. Corresponding results can be summarized in the following statement.
\begin{Thm}\label{Theorem 1.1}
Let $\mathcal{K}$ be a field, $n\ge2$. The semigroup $\langle\mathrm{M}_n(\mathcal{K}),\cdot\rangle$ is finitely based
if and only if $\mathcal{K}$ is infinite.
\end{Thm}
The ``if'' part of Theorem~\ref{Theorem 1.1} is a straightforward corollary of the general fact that the semigroup of
all $n\times n$-matrices over an infinite field satisfies only trivial identities, that is, consequences of the
associative law, see \cite[Lemma~2]{GoMi78}. The ``only if'' part, that is, the claim that the semigroup
$\langle\mathrm{M}_n(\mathcal{K}),\cdot\rangle$ over a finite field is non-finitely based, was proved in the mid-1980s
by the third author~\cite[Proposition~3]{V} and Sapir~\cite[Corollary 6.2]{sapirburnside}. For a forthcoming discussion
it is worth noting that methods used in~\cite{V} and~\cite{sapirburnside} were rather different but each of them
sufficed to cover all finite semigroups of the form $\langle\mathrm{M}_n(\mathcal{K}),\cdot\rangle$. Moreover, Sapir's
approach implied that all these semigroups are even inherently non-finitely based, where a finite semigroup is called
\emph{inherently non-finitely based} if it is not contained in any finitely based locally finite variety.

Matrices admit several fairly natural unary operations such as transposition, for instance. Here is the starting point
of the present paper: we shall treat the finite basis problem for \textbf{unary semigroups of matrices} where a
\emph{unary semigroup} is merely a semigroup equipped with some additional unary operation. For this we first have to
adapt to the unary environment the methods of~\cite{V} and~\cite{sapirburnside}. We present the corresponding results
in Section~2 while Section~3 collects several applications to the finite basis problem for unary semigroups of
matrices. Both methods of Section~2 are used here, and it turns out that they in some sense complement one another
since, in contrast to the plain semigroup case, none of them alone suffices to cover, say, all finite unary semigroups
of the form $\langle\mathrm{M}_n(\mathcal{K}),\cdot,{}^T\rangle$ where the unary operation $A\mapsto A^T$ is the usual
transposition of matrices. For this particular case our main result is the following.
\begin{Thm}\label{main result involution}
Let $\mathcal{K}=\langle K,+,\cdot\rangle$
be a finite field and $n\ge 2$. Then
\begin{enumerate}
\item the involutory semigroup $\langle \mathrm{M}_n(\mathcal{K}),\cdot,{}^T\rangle$
is not finitely based;
\item the involutory semigroup $\langle \mathrm{M}_n(\mathcal{K}),\cdot,{}^T\rangle$
is inherently non-finitely ba\-sed if and only if either $n\ge 3$ or $n=2$ and $\vert K\vert\mathrel{\not\equiv 3}
(\bmod\ 4)$.
\end{enumerate}
\end{Thm}
Other types of unary operations on matrices studied in this paper include, for instance, Moore-Penrose inversion and
symplectic transposition. We also consider unary semigroups of Boolean matrices under transposition.

We mention in passing that tools developed in Section~2 admit many further applications that will be published in a
separate paper.

We assume the reader's acquaintance with basic concepts of equational logic and the theory of varieties such as the
HSP-theorem, see, e.g., \cite[Chapter~II]{BuSa81}. As far as \sm\ notions are concerned, we adopt the standard
terminology and notation from~\cite{CP}. It should be noted, however, that the presentation is to a reasonable extent
self-contained so that most of the material  should be accessible to readers without any specific \sm-theoretic
background.

\section{Tools}\label{mainresult}

\subsection{Preliminaries}
As mentioned, by a \emph{unary semigroup} we mean an algebraic
structure $\mathcal{S}=\langle S,\cdot,{}^*\rangle$ of type
$(2,1)$ such that the binary operation $\cdot$ is associative,
i.\,e.\ $\langle S,\cdot\rangle$ is a semigroup. In general, we do
not assume any additional identities involving the unary operation
${}^*$. If the identities $(xy)^* = y^*x^*$ and $(x^*)^* = x$
happen to hold in $\mathcal{S}$, in other words, if the unary
operation $x\mapsto x^*$ is an involutory anti-automorphism of the
semigroup $\langle S,\cdot\rangle$, we call $\mathcal{S}$ an
\emph{involutory semigroup}. If, in addition, the identity
$x=xx^*x$ holds, $\mathcal{S}$ is said to be a \emph{regular
$*$-semigroup}. Each group, subject to its inverse operation
$x\mapsto x^{-1}$ is an involutory semigroup, even a regular
$*$-semigroup; throughout the paper, any group is considered as a
unary semigroup with respect to this inverse unary operation.

A wealth of examples of involutory semigroups and regular
$*$-semigroups can be obtained via the following `unary' version
of the well known \Rm\ construction (see \cite[Section~3.1]{CP}
for a description of the construction in the plain semigroup
case). Let $\mathcal{G}=\langle G,\cdot,{}^{-1}\rangle$ be a
group, $0$ a symbol beyond $G$, and $I$ a non-empty set. We
formally set $0^{-1}=0$. Given an $I\times I$-matrix $P=(p_{ij})$
over $G\cup\{0\}$ such that $p_{ij}=p_{ji}^{-1}$ for all $i,j\in
I$, we define a multiplication~$\cdot$ and a unary operation
${}^*$ on the set $(I\times G\times I)\cup\{0\}$ by the following
rules:
\begin{gather*}
a\cdot 0=0\cdot a=0\ \text{ for all $a\in (I\times G\times I)\cup \{0\}$},\\
(i,g,j)\cdot(k,h,\ell)=\left\{\begin{array}{cl}
(i,gp_{jk}h,\ell)&\ \text{if}\ p_{jk}\ne0,\\
0 &\ \text{if}\ p_{jk}=0;
\end{array}\right.\\
(i,g,j)^* = (j,g^{-1},i),\ 0^* = 0.
\end{gather*}
It can be easily checked that $\langle(I\times G\times I)\cup
\{0\},\cdot,{}^*\rangle$ becomes  an involutory semigroup; it will
be a regular $*$-semigroup precisely when $p_{ii}=e$ (the identity
element of the group $\mathcal{G}$) for all $i\in I$. We denote
this unary semigroup  by $\Mc^0(I,\mathcal{G},I;P)$ and call it
the \emph{unary \Rm\ \sm\ over $\mathcal{G}$ with the sandwich
matrix $P$}. If the involved group $\mathcal G$ happens to be the
trivial group $\mathcal{E}=\{e\}$ then we usually shall ignore the
group entry and represent the non-zero elements of such a Rees
matrix semigroup by the pairs $(i,j)$ with $i,j\in I$.

In this paper, the 10-element unary \Rm\ semigroup over the
trivial group $\mathcal{E}=\{e\}$ with the sandwich matrix
$$\begin{pmatrix}
e & e & e\\
e & e & 0\\
e & 0 & e
\end{pmatrix}$$
plays a key role; we denote this \sm\ by $\mathcal{K}_3$. Thus,
subject to the convention mentioned above, $\mathcal{K}_3$
consists of the nine pairs $(i,j)$, $i,j\in\{1,2,3\}$, and the
element $0$, and the operations restricted to its non-zero
elements can be described as follows:
\begin{gather}
\label{operations in C3}
(i,j)\cdot(k,\ell)=\left\{\begin{array}{cl}
(i,\ell)&\ \text{if}\ (j,k)\ne(2,3),(3,2),\\
0 &\ \text{otherwise};
\end{array}\right.\\
(i,j)^* = (j,i)\notag.
\end{gather}

Another unary semigroup that will be quite useful in the sequel is
the \emph{free involutory semigroup} $\mathrm{FI}(X)$ on a given
alphabet $X$. It can be constructed as follows. Let
$\overline{X}=\{x^*\mid x\in X\}$ be a disjoint copy of $X$ and
define $(x^*)^*=x$ for all $x^*\in \overline{X}$. Then
$\mathrm{FI}(X)$ is the free semigroup $(X\cup\overline{X})^+$
endowed with an involution ${}^*$ defined by
$$(x_1\cdots x_m)^* = x_m^*\cdots x_1^*$$
for all $x_1,\dots,x_m\in X\cup \overline{X}$. We will refer to
elements of $\mathrm{FI}(X)$ as to \emph{involutory words over
$X$} while elements of the free semigroup $X^+$ will be referred
to as (plain semigroup) words over $X$.

\subsection{A unary version of the critical semigroup method}
Here we present a `unary' modification of the approach used in~\cite{V}. According to the classification proposed
in~\cite{volkovjaponicae}, this approach is referred to as the \emph{critical semigroup method}.

The formulation of the corresponding result involves two simple
operators on unary semigroup varieties. For any unary semigroup
$\mathcal{S}=\langle S,\cdot,{}^*\rangle$ we denote by
$\H(\mathcal{S})$ the unary subsemigroup of $\mathcal{S}$ which is
generated by all elements of the form $xx^*$, where $x\in S$. We
call $\H(\mathcal{S})$ the \emph{Hermitian subsemigroup} of
$\mathcal{S}$. For any variety $\Vc$ of unary semigroups, let
$\H(\Vc)$ be the subvariety of $\Vc$ generated by all Hermitian
subsemigroups of members of $\Vc$. Likewise, given a positive
integer $n$, let $\P_n(\mathcal{S})$ be the unary subsemigroup of
$\mathcal{S}$ which is generated by all elements of the form
$x^n$, where $x\in S$, and let $\P_n(\Vc)$ be the subvariety of
$\Vc$ generated by all subsemigroups $\P_n(\mathcal{S})$, where
$\mathcal{S}\in \Vc$.

Denote by $\var\Sc$ the variety generated by a given unary
semigroup $\Sc$. The following easy observation will be useful in
the sequel as it helps calculating the effect of the operators
$\H$ and $\P_n$.

\begin{Lemma}\label{Lemma 3.1}
$\H(\var\Sc)=\var\H(\Sc)$ and $\P_n(\var\Sc)= \var\P_n(\Sc)$ for
every unary semigroup $\Sc$ and for each $n\in \bb N$.
\end{Lemma}

\begin{proof}
The non-trivial part of the first claim is the inclusion
$\H(\var\Sc)\subseteq\var\H(\Sc)$. Let $\Tc\in\var\Sc$, then $\Tc$
is a homomorphic image of a unary subsemigroup $\Uc$ of a direct
product of several copies of $\Sc$. But then $\H(\Tc)$ is a
homomorphic image of $\H(\Uc)$. As is easy to see, $\H(\Uc)$ is a
unary subsemigroup of a direct product of several copies of
$\H(\Sc)$. Thus $\H(\Tc)\in\var\H(\Sc)$. Since this holds for an
arbitrary $\Tc\in\var\Sc$, we conclude that
$\H(\var\Sc)\subseteq\var\H(\Sc)$. The second assertion can be
treated in a completely similar way.
\end{proof}

We are now ready to state the main result of this subsection.

\begin{Thm}\label{Theorem 2.1}
Let $\Vc$ be any unary semigroup variety such that
$\mathcal{K}_3\in\Vc$. If either
\begin{itemize}
\item there exists a group $\mathcal{G}$  such that $\mathcal{G}\in\Vc$ but
$\mathcal{G}\notin\H(\Vc)$
\end{itemize}
or
\begin{itemize}
\item there exist a positive integer $d$ and a group $\mathcal{G}$ of exponent
dividing $d$ such that $\mathcal{G}\in\Vc$ but
$\mathcal{G}\notin\P_d(\Vc)$,
\end{itemize}
then $\Vc$ has no finite basis of identities.
\end{Thm}

\begin{proof}
Assume first that there exists a group $\mathcal{G}\in\Vc$ for
which $\mathcal{G}\notin\H(\Vc)$.

\medskip

1. First we recall the basic idea of `the critical semigroup
method' in the unary setting. Suppose that ${\Vc}$ is \fb. If
$\Si$ is a finite identity basis of the variety ${\Vc}$ then there
exists a positive integer $\ell$ such that all identities from
$\Si$ depend on at most $\ell$ letters. Therefore identities from
$\Si$ hold in a unary semigroup $\mathcal{S}$ whenever all
$\ell$-generated unary subsemigroups of $\mathcal{S}$ satisfy
$\Si$. In other words, $\mathcal{S}$ belongs to $\Vc$ whenever all
of its $\ell$-generated unary subsemigroups are in ${\Vc}$. We see
that in order to prove our theorem it is sufficient to construct,
for any given positive integer $k$, a unary \sm\
$\mathcal{T}_k\notin{\Vc}$ for which all $k$-generated unary
sub\sm s of $\mathcal{T}_k$ belong to ${\Vc}$.

\medskip

 2. Fix an \id y $u(x_1,\ldots,x_m) = v(x_1,\ldots,x_m)$ that holds in ${\H(\Vc)}$
but fails in the group $\mathcal{G}$. The latter means that, for
some $g_1,\dots,g_m\in G$, substitution of $g_i$ for $x_i$ yields
\begin{equation}\label{2.1}
u(g_1,\ldots,g_m) \ne v(g_1,\ldots,g_m).
\end{equation}
Now, for each positive integer $k$, let $n =\max\{4,2k+1\}$,
$I=\{1,\dots,nm\}$ and consider the unary \Rm\ \sm \
$\mathcal{T}_k=\Mc^0(I,\mathcal{G},I;P_k)$ over the group
$\mathcal{G}$ with the sandwich matrix
$$P_k=\left(\begin{array}{ccccccc}
M_n(g_1) & E_n & O_n & O_n & \cdots & O_n & E_n^T \\
E_n^T & M_n(g_2) & E_n & O_n & \cdots & O_n & O_n \\
O_n & E_n^T & M_n(g_3) & E_n & \cdots & O_n & O_n \\
\vdots & \vdots & \vdots & \vdots & \ddots & \vdots & \vdots \\
O_n & O_n & O_n & O_n & \cdots & E_n & O_n\\
O_n & O_n & O_n & O_n & \cdots & M_n(g_{m-1}) & E_n \\
E_n & O_n & O_n & O_n & \cdots & E_n^T & M_n(g_m)
\end{array}\right),$$
where $O_n$ is the zero $n\times n$-matrix, $E_n$ is the $n\times
n$-matrix having $e$ (the identity of $\mathcal{G}$) in the
position $(n,1)$ and 0 in all other positions, $E_n^T$ is the
transpose of $E_n$, and $M_n(g)$ denotes the $n\times n$-matrix of
the form
$$M_n(g)=\left(\begin{array}{ccccccc}
e & g &  0 & \cdots & 0 & 0 & e \\
g^{-1} & e & e & \cdots & 0 & 0 & 0 \\
0 & e & e & \cdots & 0 & 0 & 0 \\
\vdots & \vdots & \vdots & \ddots & \vdots & \vdots & \vdots \\
0 & 0 & 0 & \cdots & e & e & 0 \\
0 & 0 & 0 & \cdots & e & e & e \\
e & 0 & 0 & \cdots & 0 & e & e
\end{array} \right).$$
(This construction is in a sense a combination of those of the
first and the third authors' papers \cite{A1} and \cite{V}.) We
are going to prove that $\mathcal{T}_k$ enjoys the two properties
needed, namely, it does not belong to ${\Vc}$, but each
$k$-generated unary sub\sm \ of $\mathcal{T}_k$ lies in ${\Vc}$.

\medskip

3. In order to prove that $\mathcal{T}_k\notin\Vc$, we construct
an identity that holds in ${\Vc}$, but fails in $\mathcal{T}_k$.
Consider the following $m$ terms in $mn$ letters
$x_1,\dots,x_{mn}$:

\smallskip

\leftline{$w_1=[\he{x_1}\cdots\he{x_n}] [\he{(x_{n+1}\cdots
x_{2n})}\he{x_{2n}}]\cdots$} \rightline{[$\he{(x_{(m-1)n+1}\cdots
x_{mn})}\he{x_{mn}}]$,}

\leftline{$w_2=[\he{(x_1\cdots x_n)}\he{x_n}]
[\he{x_{n+1}}\cdots\he{x_{2n}}]\times$}
\centerline{$[\he{(x_{2n+1}\cdots x_{3n})}\he{x_{3n}}]\cdots$}
\rightline{$[\he{(x_{(m-1)n+1}\cdots x_{mn})}\he{x_{mn}}]$,}

\leftline{$w_3=[\he{(x_1\cdots x_n)}\he{x_n}] [\he{(x_{n+1}\cdots
x_{2n})}\he{x_{2n}}]\times$}
\rightline{$[\he{x_{2n+1}}\cdots\he{x_{3n}}]\cdots[\he{(x_{(m-1)n+1}\ldots
x_{mn})}\he{x_{mn}}]$,}

\hbox to 5.0 in{\dotfill }

\leftline{$w_m=[\he{(x_1\cdots x_n)}\he{x_n}] [\he{(x_{n+1}\cdots
x_{2n})}\he{x_{2n}}]\ldots$} \centerline{$[\he{(x_{(m-2)n+1}\cdots
x_{(m-1)n})} \he{x_{(m-1)n}}]\times$}
\rightline{$[\he{x_{(m-1)n+1}}\cdots\he{x_{mn}}]$.}

\smallskip

\noindent Substituting $w_i$ for $x_i$ in $u$ respectively $v$, we
get the identity
\begin{equation}\label{2.2}
u(w_1,\ldots,w_m) = v(w_1,\ldots,w_m) \end{equation} which holds
in the \va y ${\Vc}$. Indeed, if we take any
$\mathcal{S}\in{\Vc}$, then, since $\he{s}\in\H(\mathcal{S})$ for
any $s\in\mathcal{S}$, all the values of $w_i$ belong to the
Hermitian sub\sm\ $\H(\mathcal{S})$ of $\mathcal{S}$. This sub\sm,
however, lies in ${\H(\Vc)}$, and therefore, satisfies the \id y
$u = v$.

Now we shall show that \eqref{2.2} fails in $\mathcal{T}_k$.
Indeed, substituting $(i,e,i)\in\mathcal{T}_k$ for $x_i$, we
calculate that the value of every term of the form
$$\he{(x_{(j-1)n+1}\cdots x_{jn})}\he{x_{jn}}$$
is equal to $((j-1)n+1,e,jn)$ while the value of each term of the
form
$$\he{x_{(j-1)n+1}}\cdots \he{x_{jn}}$$
is equal to  $((j-1)n+1,g_j,jn)$. Hence the value of $w_j$ is just
$(1,g_j,mn)$. Therefore, under this substitution, the left hand
part of (\ref{2.2}) takes the value $(s,u(g_1,\ldots,g_m),t)$ for
suitable $s,t\in \{1,mn\}$ while the value of the right hand part
of (\ref{2.2}) is $(s',v(g_1,\ldots,g_m),t')$ (again for suitable
$s',t'\in\{1,mn\}$). In view of the inequality (\ref{2.1}), these
elements do not coincide in $\mathcal{T}_k$.

\smallskip

4. It remains to prove that each $k$-generated unary sub\sm\ of
$\mathcal{T}_k$ lies in ${\Vc}$. For every $m$-tuple
$(\la_1,\ldots,\la_m)$ of positive integers satisfying
\begin{equation} \label{2.3}
1\leq\la_1\leq n<\la_2\leq 2n<\la_3\leq\ldots (m-1)n <\la_m\leq
mn,
\end{equation}
consider the unary sub\sm \ $\mathcal{T}_k(\la_1,\ldots,\la_m)$ of
$\mathcal{T}_k$ consisting of 0 and all triples $(i,g,j)$ such
that $g\in \mathcal{G}$ and $i,j\not\in\{\la_1,\ldots,\la_m\}$.
Using that $2k<n$ according to our choice of $n$, one concludes
that any given $k$ elements of $\mathcal{T}_k$ must be contained
in $\mathcal{T}_k(\la_1,\ldots,\la_m)$ for suitable
$\la_1,\ldots,\la_m$. Thus it is sufficient to prove that each
semigroup  of the form $\mathcal{T}_k(\la_1,\ldots,\la_m)$ belongs
to the variety ${\Vc}$.

Let us fix positive integers $\la_1,\ldots,\la_m$ satisfying
(\ref{2.3}). When multiplying triples from
$\mathcal{T}_k(\la_1,\ldots,\la_m)$, the
$\la_1^{\mathrm{th}},\dots,\la_m^{\mathrm{th}}$ rows and columns
of the sandwich matrix $P_k$ are never involved. Therefore we can
identify $\mathcal{T}_k(\la_1,\ldots,\la_m)$ with the unary \Rm\
\sm\ $\Mc^0(I',\mathcal{G},I';P'_k)$ over the group $\mathcal{G}$
where $I'=I\setminus\{\la_1,\dots, \la_m\}$ and the sandwich
matrix $P'_k= P_k(\la_1,\ldots,\la_m)$ is obtained from $P_k$ by
deleting its $\la_1^{\mathrm{th}},\dots,\la_m^{\mathrm{th}}$ rows
and columns. Note that by \eqref{2.3} exactly one row and one
column of each block $M_n(g_i)$ is deleted.

Now we transform the matrix $P_k(\la_1,\ldots,\la_m)$ as follows.
For each $i$ such that $(i-1)n+2<\la_i$, we multiply successively
\begin{align}
\label{transform}
&\text{the row $((i-1)n+2)$ by $g_i$ from the left and}\notag\\[-1ex]
&\text{the column $((i-1)n+2)$ by $g_i^{-1}$ from the right;}\notag\\[-.5ex]
&\text{the row $((i-1)n+3)$ by $g_i$ from the left and}\notag\\[-1ex]
&\text{the column $((i-1)n+3)$ by $g_i^{-1}$ from the right;}\\[-.5ex]
&\hbox to 3.0 in{\dotfill }\notag\\[-.5ex]
&\text{the row $(\la_i-1)$ by $g_i$ from the left and}\notag\\[-1ex]
&\text{the column $(\la_i-1)$ by $g_i^{-1}$ from the right.}\notag
\end{align}

In order to help the reader to understand the effect of the
transformations~\eqref{transform}, we illustrate their action on
the block obtained from $M_n(g_i)$ by removing the
$\la_i^{\mathrm{th}}$ row and column in the following scheme in
which $\lambda_i$ has been chosen to be equal to $(i-1)n+5$. (The
transformations have no effect beyond $M_n(g_i)$ because all the
rows and columns of $P_k(\la_1,\ldots,\la_m)$ involved
in~\eqref{transform} have non-zero entries only within
$M_n(g_i)$.)

{\small
\begin{center}
\begin{tabular}{cc}
The block obtained from $M_n(g_i)$ by erasing  & After the first\\
the $((i-1)n+5)^{\mathrm{th}}$ row and column  & transformation \\[1ex]
$\begin{pmatrix}
e        & g_i & 0 & 0 &  0 &\cdots & 0 & e \\
g_i^{-1} & e   & e & 0 &  0 &\cdots & 0 & 0 \\
0        & e   & e & e &  0 &\cdots & 0 & 0 \\
0        & 0   & e & e &  0 &\cdots & 0 & 0 \\
0        & 0   & 0 & 0 &  e &\cdots & 0 & 0 \\
\vdots   & \vdots & \vdots & \vdots & \vdots &\ddots & \vdots & \vdots \\
0        & 0   & 0 & 0 &  0 &\cdots & e & e \\
e        & 0   & 0 & 0 &  0 &\cdots & e & e
\end{pmatrix}$
& $\begin{pmatrix}
e        & e   & 0 & 0 &  0 &\cdots & 0 & e \\
e   & e  & g_i & 0 &  0 &\cdots & 0 & 0 \\
0   & g_i^{-1} & e & e &  0 &\cdots & 0 & 0 \\
0        & 0   & e & e &  0 &\cdots & 0 & 0 \\
0        & 0   & 0 & 0 &  e &\cdots & 0 & 0 \\
\vdots   & \vdots & \vdots & \vdots & \vdots &\ddots & \vdots & \vdots \\
0        & 0   & 0 & 0 &  0 &\cdots & e & e \\
e        & 0   & 0 & 0 &  0 &\cdots & e & e
\end{pmatrix}$\\
&\\
After the second & After the third\\
transformation   & transformation\\[1ex]
$\begin{pmatrix}
e        & e   & 0 & 0 &  0 &\cdots & 0 & e \\
e        & e   & e & 0 &  0 &\cdots & 0 & 0 \\
0        & e & e & g_i &  0 &\cdots & 0 & 0 \\
0 & 0   & g_i^{-1} & e &  0 &\cdots & 0 & 0 \\
0        & 0   & 0 & 0 &  e &\cdots & 0 & 0 \\
\vdots   & \vdots & \vdots & \vdots & \vdots &\ddots & \vdots & \vdots \\
0        & 0   & 0 & 0 &  0 &\cdots & e & e \\
e        & 0   & 0 & 0 &  0 &\cdots & e & e
\end{pmatrix}$
& $\begin{pmatrix}
e        & e   & 0 & 0 &  0 &\cdots & 0 & e \\
e        & e   & e & 0 &  0 &\cdots & 0 & 0 \\
0        & e   & e & e &  0 &\cdots & 0 & 0 \\
0        & 0   & e & e &  0 &\cdots & 0 & 0 \\
0        & 0   & 0 & 0 &  e &\cdots & 0 & 0 \\
\vdots   & \vdots & \vdots & \vdots & \vdots &\ddots & \vdots & \vdots \\
0        & 0   & 0 & 0 &  0 &\cdots & e & e \\
e        & 0   & 0 & 0 &  0 &\cdots & e & e
\end{pmatrix}$\\
\end{tabular}
\end{center}

} \noindent Now it should be clear that also in general the
transformations~\eqref{transform} result in a matrix $Q_k$ all of
whose non-zero entries are equal to $e$. On the other hand, it is
known (see, e.\,g., \cite[Proposition~6.2]{A1}) that the
transformations~\eqref{transform} of the sandwich matrix do not
change the unary \sm\ $\mathcal{T}_k(\la_1,\ldots,\la_m)$; in
other words $\mathcal{T}_k(\la_1,\ldots,\la_m)$ is isomorphic to
the $I'\times I'$ unary \Rm \ \sm \ $\mathcal{R}_k$ over
$\mathcal{G}$ with the sandwich matrix $Q_k$. Let $\mathcal{U}_k$
be the $I'\times I'$ unary \Rm \ \sm\ over the trivial group
$\mathcal{E}$ with the sandwich matrix $Q_k$. It is easy to check
that the mapping $\mathcal{G}\times \mathcal{U}_k\rightarrow
\mathcal{R}_k$ defined by
$$(g,(i,e,j))\mapsto (i,g,j),\quad  (g,0)\mapsto 0$$
for all $g\in \mathcal{G}$, $i,j\in I'$, is a unary semigroup
homomorphism onto $\mathcal{R}_k$. Now we note that $\mathcal{G}
\in {\Vc}$ and $\mathcal{U}_k$ belongs to the \va y generated by
$\mathcal{K}_3$ (see \cite[Theorem 5.2]{A1}). This yields
$\mathcal{T}_k(\la_1,\ldots,\la_m)\cong\mathcal{R}_k\in{\Vc}$.

\smallskip

The case when there exist a positive integer $d$ and a group
$\mathcal{G}$ of exponent dividing $d$ such that
$\mathcal{G}\in\Vc$ but $\mathcal{G}\notin\P_d(\Vc)$ can be
treated in a very similar way. The construction of the critical
semigroups remains the same, and the only modification to be made
in the rest of the proof is to replace the terms $w_i$ above by
the following plain semigroup words:

\smallskip

\leftline{$w_1=[x_1^d\cdots x_n^d][x_{n+1}\cdots x_{2n}]^d\cdots
[x_{(m-1)n+1}\cdots x_{mn}]^d$,} \leftline{$w_2=[x_1\cdots
x_n]^d[x_{n+1}^d\cdots x_{2n}^d]\cdots [x_{(m-1)n+1}\cdots
x_{mn}]^d$,}

\hbox to 3.3 in {\dotfill }

\leftline{$w_m=[x_1\cdots x_n]^d[x_{n+1}\cdots
x_{2n}]^d\cdots[x_{(m-1)n+1}^d\cdots x_{mn}^d]$.}

\smallskip

\noindent (These words already have been used in the plain
semigroup case by the third author~\cite{V}.)
\end{proof}

\subsection{A unary version of the method of inherently
non-finitely based semigroups} Recall that a finite algebraic structure of type $\tau$ is called \emph{inherently
non-finitely based} if it is not contained in any finitely based locally finite variety of type $\tau$. Here we prove a
sufficient condition for an involutory semigroup to be inherently non-finitely based and exhibit two concrete examples
of involutory semigroups satisfying this condition. These examples will be essential in our applications in Section~3.

Let $x_1,x_2,\dots,x_n,\dots$ be a sequence of letters. The
sequence $\{Z_n\}_{n=1,2,\dots}$ of \emph{Zimin words} is defined
inductively by $Z_1=x_1$, $Z_{n+1}=Z_nx_{n+1}Z_n$. We say that an
involutory word $v$ is an \emph{involutory isoterm for a unary
semigroup $\mathcal{S}$} if the only involutory word $v'$ such
that $\mathcal{S}$ satisfies the involutory semigroup identity
$v=v'$ is the word $v$ itself.

\begin{Thm}
\label{Theorem 2.2} Let $\mathcal{S}$ be a finite involutory
semigroup. If all Zimin words are involutory isoterms for
$\mathcal{S}$, then $\mathcal{S}$ is inherently non-finitely
based.
\end{Thm}

\begin{proof}
Arguing by contradiction, suppose that $\mathcal{S}$ belongs to a
\fb\ locally finite variety $\Vc$. If $\Si$ is a finite identity
basis of $\Vc$, then there exists a positive integer $\ell$ such
that all identities from $\Si$ depend on at most $\ell$ letters.
Clearly, all identities in $\Si$  hold in $\mathcal{S}$. In the
following, our aim will be to construct, for any given positive
integer $k$, an infinite, finitely generated involutory \sm\
$\mathcal{T}_k$ which satisfies all identities in at most $k$
variables that hold in $\mathcal{S}$. In particular,
$\mathcal{T}_k$ will satisfy all identities from $\Si$. This
yields a contradiction, as then we must conclude that
$\mathcal{T}_\ell\in\Vc$, which is impossible by the local
finiteness of $\Vc$.

We shall employ a construction invented by
Sapir~\cite{sapirburnside}, see also his lecture
notes~\cite{Sapir}. We fix $k$ and let $r=6k+2$. Consider the
$r^2\times r$-matrix $M$ shown in Fig.~\ref{matrices} on the left.
\begin{figure}[th]
$$M=\begin{pmatrix}
  1 & 1 & \cdots & 1 & 1\\

  \vdots & \vdots & \ddots & \vdots & \vdots \\
  1 & r & \cdots & 1 & r\\
  2 & 1 & \cdots & 2 & 1\\

  \vdots & \vdots & \ddots & \vdots & \vdots\\
  2 & r & \cdots & 2 & r\\
  \vdots & \vdots & \ddots & \vdots & \vdots\\
  r & 1 & \cdots & r & 1\\

  \vdots & \vdots & \ddots & \vdots & \vdots\\
  r & r & \cdots & r & r\\
\end{pmatrix} \qquad
 M_{A}=\begin{pmatrix}
  a_{11} & a_{12} & \cdots & a_{1r-1} & a_{1 r} \\

  \vdots & \vdots & \ddots & \vdots & \vdots \\
  a_{11} & a_{r2} & \cdots & a_{1r-1} & a_{rr}\\
  a_{21} & a_{12} & \cdots & a_{2r-1} & a_{1r}\\

  \vdots & \vdots & \ddots & \vdots & \vdots \\
  a_{21} & a_{r2} & \cdots & a_{2r-1} & a_{rr}\\
  \vdots & \vdots & \ddots & \vdots & \vdots \\
  a_{r1} & a_{12} & \cdots & a_{rr-1} & a_{1r}\\

  \vdots & \vdots & \ddots & \vdots & \vdots \\
  a_{r1} & a_{r2} & \cdots & a_{rr-1} & a_{rr}\\
\end{pmatrix}$$
\caption{The matrices $M$ and $M_A$}\label{matrices}
\end{figure}
All odd columns of $M$ are identical and equal to the transpose of
the row $(1,1,\ldots,1,2,2,\ldots,2,\ldots,r,r,\ldots,r)$ where
each number occurs $r$ times. All even columns of $M$ are
identical and equal to the transpose of the row
$(1,2,\ldots,r,1,2,\ldots,r,\ldots,1,2,\ldots,r)$ in which the
block $1,2,\dots,r$ occurs $r$ times.

Now consider the alphabet $A=\{a_{ij}\mid 1\leq i,j\leq r\}$ of
cardinality $r^2$. We convert the matrix $M$ to the matrix $M_{A}$
(shown in Fig.~\ref{matrices} on the right) by replacing numbers
by letters according to the following rule: whenever the number
$i$ occurs in the column $j$ of $M$, we substitute it with the
letter $a_{ij}$ to get the corresponding entry in $M_A$.

Let $v_{t}$ be the word in the $t^{\mathrm{th}}$ row of the matrix
$M_A$. Consider the endomorphism $\gamma:A^+\rightarrow A^+$
defined by
$$\gamma(a_{ij})=v_{(i-1)r+j}.$$
Let $V_k$ be the set of all factors of the words in the sequence
$\{\gamma^m(a_{11})\}_{m=1,2,\dots}$ and let $0$ be a symbol
beyond $V_k$. We define a multiplication $\cdot$ on the set
$V_k\cup\{0\}$ as follows:
$$
u\cdot v=\left\{\begin{array}{cl}
uv&\ \text{if}\ u,v,uv\in V_k,\\
0 &\ \text{otherwise}.
\end{array}\right.
$$
Clearly, $\langle V_k\cup\{0\},\cdot\rangle$ becomes a semigroup
which we denote by $\mathcal{V}_k^0$. Using this semigroup, we can
conveniently reformulate two major combinatorial results by Sapir:

\begin{Prop} {\rm\cite[Proposition 2.1]{Sapir}}
\label{avoidable words} Let $X_k=\{x_1,\dots,x_k\}$ and $w\in
X_k^+$. Assume that there exists a homomorphism
$\varphi:X_k^+\to\mathcal{V}_k^0$ for which $\varphi(w)\ne0$. Then
there is an endomorphism $\psi:X_k^+\to X_k^+$ such that the word
$\psi(w)$ appears as a factor in the Zimin word $Z_k$.
\end{Prop}

\begin{Prop} {\rm\cite[Lemma 4.14]{Sapir}}
\label{avoidable identities} Let $X_k=\{x_1,\dots,x_k\}$ and
$w,w'\in X_k^+$. Assume that there exists a homomorphism
$\varphi:X_k^+\to\mathcal{V}_k^0$ for which
$\varphi(w)\ne\varphi(w')$. Then the identity $w=w'$ implies a
non-trivial semigroup identity of the form $Z_{k+1}=z$.
\end{Prop}

Now let $\overline{\mathcal{V}}_k^0$ denote the semigroup
anti-isomorphic to $\mathcal{V}_k^0$; we shall use the notation
$x\mapsto x^*$ for the mutual anti-isomorphisms between
$\mathcal{V}_k^0$ and $\overline{\mathcal{V}}_k^0$ in both
directions and denote $\{v^*\mid v\in V_k\}$ by $\overline{V}_k$.
Let
$$\mathcal{T}_k=\langle V_k\cup\overline{V}_k\cup\{0\},\cdot,{}^*\rangle$$
be the 0-direct union of $\mathcal{V}_k^0$ and
$\overline{\mathcal{V}}_k^0$; this means that we identify $0$ with
$0^*$, preserve the multiplication in both $\mathcal{V}_k^0$ and
$\overline{\mathcal{V}}_k^0$, and set $u\cdot v^*=u^*\cdot v=0$
for all $u,v\in V_k$. This is the unary semigroup we need.

It is clear that $\mathcal{T}_k$ is infinite and is generated (as
a unary semigroup) by the set $A$ which is finite. It remains to
verify that $\mathcal{T}_k$ satisfies every identity in at most
$k$ variables that holds in our initial unary semigroup
$\mathcal{S}$. So, let $p,q\in\mathrm{FI}(X_k)$ and suppose that
the identity $p=q$ holds in $\mathcal{S}$ but fails in
$\mathcal{T}_k$. Then there exists a unary semigroup homomorphism
$\varphi:\mathrm{FI}(X_k)\to\mathcal{T}_k$ for which
$\varphi(p)\ne\varphi(q)$. Hence, at least one of the elements
$\varphi(p)$ and $\varphi(q)$ is not equal to $0$; (without loss
of generality) assume that $\varphi(p)\ne 0$. Then we may also
assume $\varphi(p)\in V_k$; otherwise we may consider the identity
$p^*=q^*$ instead of $p=q$. Since $\varphi(p)\ne0$, there is no
letter $x\in X_k$ such that $p$ contains both $x$ and $x^*$. Now
we define a substitution
$\sigma:\mathrm{FI}(X_k)\to\mathrm{FI}(X_k)$ as follows:
$$\sigma(x)=\left\{\begin{array}{ll}
x^* & \text{if } p \text{ contains } x^*,\\
x & \text{otherwise}.\end{array}\right.$$ Then $\sigma(p)$ does
not contain any starred letter, thus being a plain word in
$X_k^+$. Since $\sigma^2$ is the identity mapping, we have
$\varphi(p)= (\varphi\sigma)(\sigma(p))$, and $\varphi\sigma$ maps
$X_k^+$ into $\mathcal{V}_k^0$. Now we consider two cases.

\emph{\textbf{Case 1:} $\sigma(q)$ contains a starred letter.} We
apply Proposition~\ref{avoidable words} to the plain word
$\sigma(p)$ and the semigroup homomorphism
$X_k^+\to\mathcal{V}_k^0$ obtained by restricting $\varphi\sigma$
to $X_k^+$. We conclude that there is an endomorphism $\psi$ of
$X_k^+$ such that the word $\psi(\sigma(p))$ appears as a factor
in the Zimin word $Z_k$. Thus, $Z_k=z'\psi(\sigma(p))z''$ for some
$z',z''$ (that may be empty). The endomorphism $\psi$ extends in a
natural way to an endomorphism of the free involutory semigroup
$\mathrm{FI}(X_k)$ and there is no harm in denoting the extension
by $\psi$ as well. The identity $p=q$ implies the identity
\begin{equation}
\label{consequence of p=q}
z'\psi(\sigma(p))z''=z'\psi(\sigma(q))z''.
\end{equation}
The left hand side of~\eqref{consequence of p=q} is $Z_k$ and the
identity is not trivial because its right hand side involves a
starred letter. Since $p=q$ holds in our initial semigroup
$\mathcal{S}$, so does~\eqref{consequence of p=q}. But this
contradicts the assumption that all Zimin words are involutory
isoterms for $\mathcal{S}$.

\emph{\textbf{Case 2:} $\sigma(q)$ contains no starred letter.} In
this case $\sigma(q)$ is a plain word in $X_k^+$, and we are in a
position to apply Proposition~\ref{avoidable identities} to the
semigroup identity $\sigma(p)=\sigma(q)$ and the semigroup
homomorphism $X_k^+\to\mathcal{V}_k^0$ obtained by restricting
$\varphi\sigma$ to $X_k^+$. We conclude that $\sigma(p)=\sigma(q)$
implies a non-trivial semigroup identity $Z_{k+1}=z$. Therefore
the identity $p=q$ implies $Z_{k+1}=z$, and we again get a
contradiction.
\end{proof}

As a concrete example of an inherently non-finitely based
involutory semigroup, consider the \emph{twisted Brandt monoid}
$\TB=\langle B_2^1,\cdot,{}^*\rangle$, where $B_2^1$ is the set of
the following six $2\times 2$-matrices:
$$\begin{pmatrix} 0 & 0\\ 0 & 0\end{pmatrix},\
\begin{pmatrix} 1 & 0\\ 0 & 0\end{pmatrix},\
\begin{pmatrix} 0 & 1\\ 0 & 0\end{pmatrix},\
\begin{pmatrix} 0 & 0\\ 1 & 0\end{pmatrix},\
\begin{pmatrix} 0 & 0\\ 0 & 1\end{pmatrix},\
\begin{pmatrix} 1 & 0\\ 0 & 1\end{pmatrix},$$
the binary operation $\cdot$ is the usual matrix multiplication
and the unary operation ${}^*$ fixes the matrices
$$\begin{pmatrix} 0 & 0\\ 0 & 0\end{pmatrix},\
\begin{pmatrix} 0 & 1\\ 0 & 0\end{pmatrix},\
\begin{pmatrix} 0 & 0\\ 1 & 0\end{pmatrix},\
\begin{pmatrix} 1 & 0\\ 0 & 1\end{pmatrix}$$
and swaps each of the matrices
$$\begin{pmatrix} 1 & 0\\ 0 & 0\end{pmatrix},\
\begin{pmatrix} 0 & 0\\ 0 & 1\end{pmatrix}$$
with the other one.

\begin{Cor}
\label{twisted Brandt} The twisted Brandt monoid $\TB$ is
inherently non-\fb.
\end{Cor}

\begin{proof}
By Theorem~\ref{Theorem 2.2} we only have to show that $\TB$
satisfies no non-trivial involutory semigroup identity of the form
$Z_n=z$. If $z$ is a plain semigroup word, we can refer to
\cite[Lemma~3.7]{sapirburnside} which shows that the semigroup
$\langle B_2^1,\cdot\rangle$ does not satisfy any non-trivial
\textbf{semigroup} identity of the form $Z_n=z$. If we suppose
that the involutory word $z$ contains a starred letter, we can
substitute the matrix $\left(\begin{smallmatrix}1 & 0\\ 0 &
0\end{smallmatrix}\right)$ for all letters occurring in $Z_n$ and
$z$. Since this matrix is idempotent, the value of the word $Z_n$
under this substitution equals $\left(\begin{smallmatrix}1 & 0\\ 0
& 0\end{smallmatrix}\right)$. On the other hand, $z$ evaluates to
a product involving the matrix $\left(\begin{smallmatrix}1 & 0\\ 0
& 0\end{smallmatrix}\right)^*= \left(\begin{smallmatrix}0 & 0\\ 0
& 1\end{smallmatrix}\right)$, and it is easy to see that such a
product is equal to either $\left(\begin{smallmatrix}0 & 0\\ 0 &
1\end{smallmatrix}\right)$ or $\left(\begin{smallmatrix}0 & 0\\ 0
& 0\end{smallmatrix}\right)$. Thus, the identity $Z_n=z$ cannot
hold in $\TB$ in this case as well.
\end{proof}

An equivalent way to define $\TB$ is to consider the 5-element
unary \Rm\ semigroup over the trivial group $\mathcal{E}=\{e\}$
with the sandwich matrix
$$\begin{pmatrix}
0 & e\\
e & 0
\end{pmatrix}$$
and then to adjoin to this unary \Rm\ semigroup an identity
element. For convenience and later use we note that $\TB$ can thus
be realized as the set
$$\{(1,1),(1,2),(2,1),(2,2),0,1\}$$ endowed with the operations
\begin{gather}
\label{operations in TB}
(i,j)\cdot(k,\ell)=\left\{\begin{array}{cl}
(i,\ell)&\ \text{if}\ (j,k)\in \{(1,2),(2,1)\},\\
0 &\ \text{otherwise};
\end{array}\right.\\
1\cdot x=x=x\cdot 1,\ 0\cdot x=0=x\cdot 0 \text{ for all }x; \notag\\
 (i,j)^* = (j,i),\ 1^*=1,\ 0^*=0\notag.
\end{gather}

Suppose that $\mathcal{A}$ is a finite algebraic structure for
which the variety $\var\mathcal{A}$ contains an inherently
non-finitely based algebraic structure. Immediately from the
definition it follows that $\mathcal{A}$ is also inherently
non-\fb. This observation is useful, in particular, for the
justification of our second example of an involutory inherently
non-\fb\ semigroup. This is a `twisted version' $\TA$ of another
6-element semigroup that often shows up under the name $A_2^1$ in
the theory of semigroup varieties. The unary semigroup $\TA$ is
formed by the 6 matrices
$$\begin{pmatrix} 0 & 0\\ 0 & 0\end{pmatrix},\
\begin{pmatrix} 1 & 0\\ 0 & 0\end{pmatrix},\
\begin{pmatrix} 0 & 1\\ 0 & 0\end{pmatrix},\
\begin{pmatrix} 1 & 0\\ 1 & 0\end{pmatrix},\
\begin{pmatrix} 0 & 1\\ 0 & 1\end{pmatrix},\
\begin{pmatrix} 1 & 0\\ 0 & 1\end{pmatrix}$$
under the usual matrix multiplication and the unary operation that
swaps each of the matrices
$$\begin{pmatrix} 1 & 0\\ 0 & 0\end{pmatrix},\
\begin{pmatrix} 0 & 1\\ 0 & 1\end{pmatrix}$$
with the other one and fixes all other matrices. Alternatively,
$\TA$ is obtained from the 5-element unary \Rm\ semigroup
$\mathcal{A}_2$ over $\mathcal{E}=\{e\}$ with the sandwich matrix
\begin{equation}
\label{matrix for TA}
\begin{pmatrix}
0 & e\\
e & e
\end{pmatrix}
\end{equation}
by adjoining an identity element. Again, for later use, we note
that $\TA$ can be realized as the set
$$\{(1,1),(1,2),(2,1),(2,2),0,1\}$$ endowed with the operations
\begin{gather}
\label{operations in TA}
(i,j)\cdot(k,\ell)=\left\{\begin{array}{cl}
(i,\ell)&\ \text{if}\ (j,k)\ne (1,1)\\
0 &\ \text{if}\ (j,k)=(1,1);
\end{array}\right.\\
1\cdot x=x=x\cdot 1,\ 0\cdot x=0=x\cdot 0 \text{ for all }x; \notag\\
 (i,j)^* = (j,i),\ 1^*=1,\ 0^*=0\notag.
\end{gather}

\begin{Cor}
\label{twisted A} The involutory monoid $\TA$ is inherently
non-\fb.
\end{Cor}

\begin{proof}
We represent $\TA$ as in \eqref{operations in TA} and $\TB$ as in
\eqref{operations in TB} and consider the direct square
$\TA\times\TA$. It is then easy to check that the twisted Brandt
monoid $\TB$ is a homomorphic image of the unary subsemigroup of
$\TA\times\TA$ generated by the pairs $(1,1)$,
$\bigl((1,1),(2,2)\bigr)$ and $\bigl((2,2),(1,1)\bigr)$. Thus,
$\TB$ belongs to $\var\TA$. Since by Corollary~\ref{twisted
Brandt} $\TB$ is inherently non-\fb, so is $\TA$.
\end{proof}

Sapir~\cite[Proposition~7]{sapirburnside} has shown that a (plain)
finite semigroup $\mathcal{S}$ is inherently non-\fb\ \textbf{if
and only if} all Zimin words are isoterms for $\mathcal{S}$, that
is, $\mathcal{S}$ satisfies no non-trivial semigroup identity of
the form $Z_n=z$. Our Theorem~\ref{Theorem 2.2} models the `if'
part of this statement but we do not know whether or not the `only
if' part transfers to the involutory environment. Some partial
results in this direction have been recently obtained by the
second author \cite{Dolinka}. Here we present yet another special
result which however suffices for our purposes.

\begin{Prop}
\label{NINFB} Let $\mathcal{S}=\langle S,\cdot,{}^*\rangle$ be a
finite involutory semigroup and suppose that there exists an
involutory word $\om(x)$ in one variable $x$ such that
$\mathcal{S}$ satisfies the identity $x=x\om(x)x$. Then
$\mathcal{S}$ is not inherently non-\fb.
\end{Prop}

\begin{proof}
This is a consequence of some deep facts obtained by Margolis and
Sapir in~\cite{margolissapir} and we shall follow their main
arguments. The reason that these arguments apply here is that in
$\mathcal{S}$, the Green relation $\Rc$ can be expressed in terms
of equational logic. This ensures that we can construct a finite
set of identities that defines a locally finite variety of
involutory semigroups containing $\mathcal{S}$.

First of all, we recall the definition of the relation $\Rc$. As
usual, for a semigroup $a\in S$, the set $\{as\mid s\in
S\}\cup\{a\}$ (that is, the right ideal generated by $a$) is
denoted by $aS^1$. Now we say that elements $a,b\in S$ are
$\Rc$-\emph{related} if $aS^1=bS^1$.

Since $\mathcal{S}$ satisfies the identity $x=x\om(x)x$, we have
$a\mathrel{\Rc}a\om(a)$ for each element $a\in S$. Thus, for
$a,b\in S$, we have $a\mathrel{\Rc}b$ if and only if
$a\om(a)\mathrel{\Rc}b\om(b)$. Since $a\om(a)$ and $b\om(b)$ are
idempotents, that latter condition is equivalent to the two
equalities $a\om(a)\cdot b\om(b)=b\om(b)$ and $b\om(b)\cdot
a\om(a)=a\om(a)$. In particular, for $u,v\in S$ we have
$uv\mathrel{\Rc}u$ if and only if $uv\om(uv)\cdot u\om(u)=u\om(u)$
(since the second equality $u\om(u)\cdot uv\om(uv)=uv\om(uv)$ is
always true).

Let $Z_n'$ be the word obtained from the Zimin word $Z_n$ by
deleting the last letter (which is $x_1$), that is, $Z_n'x_1=Z_n$.
Further let $h$ denote the $\Rc$-\emph{height} of $\mathcal{S}$,
that is, $h$ is the length of the longest possible $\Rc$-chain
$$s_1\mathrel{{<}_{\Rc}}s_2\mathrel{{<}_{\Rc}}\cdots\mathrel{{<}_{\Rc}}s_k$$
where $a\mathrel{{<}_{\Rc}}b$ means that $aS^1\subseteq bS^1$ but
$bS^1\nsubseteq aS^1$. Set $n=h+1$; Lemma~7 in
\cite{margolissapir} shows that $\mathcal{S}$ satisfies
$Z_n'\mathrel{\Rc}Z_n$ whence it satisfies the identity
\begin{equation}
\label{ziminR} Z_n\om(Z_n)\cdot Z_n'\om(Z_n')=Z_n'\om(Z_n').
\end{equation}
On the other hand, each involutory semigroup $\mathcal{T}$ which
satisfies $x=x\om(x)x$ and \eqref{ziminR} necessarily satisfies
$Z_n'\mathrel{\Rc}Z_n$, hence (its semigroup reduct) belongs to
the quasivariety $\mathbf{Q}_n$ of semigroups defined by the
quasi-identity
\begin{equation*}
xZ_n=yZ_n\rightarrow xZ_n'=yZ_n'.
\end{equation*}
Lemma 8 in \cite{margolissapir} then shows that a finitely
generated semigroup $\mathcal{T}\in\mathbf{Q}_n$ is finite if and
only if it is periodic and all subgroups of $\mathcal{T}$ are
locally finite. We note that an involutory semigroup is finitely
generated if and only if so is its semigroup reduct. Hence it
suffices to find a finite number of identities which hold in
$\mathcal{S}$ and which force each (involutory) semigroup to be
periodic and to have only locally finite subgroups.

We can proceed as at the end of \cite{margolissapir}: first, since
$\mathcal{S}$ is finite, it satisfies the identity
$x^k=x^{k+\ell}$ for some $k,\ell\ge 1$. This identity definitely
forces any (involutory) semigroup to be periodic. Next, let
$\mathcal{G}$ be the direct product of all maximal subgroups of
$\mathcal{S}$. By the Oates-Powell theorem~\cite{oatespowell}, see
also \cite[\S5.2]{Ne}, the locally finite variety
$\var\mathcal{G}$ generated by the finite group $\mathcal{G}$ can
be defined by a single identity $v(x_1,\dots,x_m)=1$. The left
hand side $v$ of this identity can be assumed to contain no
negative letters, that is, $v$ is a plain semigroup word in the
letters $x_1,\dots,x_m$. Now let
$\mathcal{F}=\mathcal{F}(x_1,\dots,x_m)$ be the $m$-generated
relatively free semigroup in the (locally finite) semigroup
variety generated by the semigroup reduct of $\mathcal{S}$. Let
$e$ be an idempotent in the minimal ideal of $\mathcal{F}$ and let
$u(x_1,\dots,x_m)$ be a word  whose value in $\mathcal{F}$ is $e$.
It follows that $\mathcal{F}$ (and therefore $\mathcal{S}$)
satisfies the identity
\begin{equation}
\label{idempotent law} u=u^2.
\end{equation}
For every element $g\in\mathcal{F}$, the product $ege$ belongs to
the maximal subgroup of $\mathcal{F}$ with idempotent $e$.
Consequently, $\mathcal{F}$ (and therefore $\mathcal{S}$)
satisfies the identity
\begin{equation}
\label{crucial law} v(ux_1u,\dots,ux_mu)=u.
\end{equation}
Note that both sides of that identity are plain semigroup words in
the letters $x_1,\dots,x_m$.

Now consider the variety of involutory semigroups defined by the
identities $x=x\om(x)x$, $x^k=x^{k+\ell}$ together with
\eqref{ziminR}, \eqref{idempotent law} and \eqref{crucial law}. By
construction, $\mathcal{S}$ is a member of that variety. Let
$\mathcal{T}$ be any finitely generated member; then, as already
mentioned, the semigroup reduct of $\mathcal{T}$ is also finitely
generated. The first and the third identity ensure that the
semigroup reduct of $\mathcal{T}$ belongs to the quasivariety
$\mathbf{Q}_n$, and therefore it is finite provided that it is
periodic and all its subgroups are locally finite. Periodicity is,
of course, guaranteed by the second identity. Finally, each group
$\mathcal{H}$ that satisfies the identities \eqref{idempotent law}
and \eqref{crucial law} satisfies the identity
$v(x_1,\dots,x_m)=1$, whence $\mathcal{H}$ belongs to
$\var\mathcal{G}$ and so is locally finite. Altogether,
$\mathcal{T}$ is finite and the proposition is proved.
\end{proof}

Proposition \ref{NINFB} implies in particular that no finite regular $*$-semigroup can be inherently non-\fb\ as one
can use $x^*$ in the role of the term $\om(x)$. In particular, the unary semigroup $\langle B_2^1,\cdot,{}^T\rangle$,
where the unary operation is the usual matrix transposition, is not inherently non-\fb\ (this fact was first discovered
by Sapir, see~\cite{sapirinverse}), even though it is not finitely based~\cite{kleiman}.

\section{Applications}

\subsection{A property of matrices of rank 1}
Recall that, for a field $\mathcal{K}$, we denote the set of all
$n\times n$-matrices over $\mathcal{K}$ by
$\mathrm{M}_n(\mathcal{K})$. We start with registering a simple
property of rank~1 matrices. This property is, of course, known,
but we do provide a proof for the sake of completeness.
\begin{Lemma}
\label{rank 1} If $A\in\mathrm{M}_n(\mathcal{K})$, $n\ge2$, has
rank $1$, then $A^nBA^{n-1}=A^{n-1}BA^n$ for any matrix
$B\in\mathrm{M}_n(\mathcal{K})$.
\end{Lemma}

\begin{proof}
By the Cayley-Hamilton theorem, $A$ satisfies the equation
$$A^n-\tr(A)A^{n-1}=0,$$
where $\tr(A)$ stands for the trace of the matrix $A$. Hence
$$A^nBA^{n-1}=\tr(A)A^{n-1}BA^{n-1}=A^{n-1}B\tr(A)A^{n-1}=A^{n-1}BA^n,$$
as required.
\end{proof}

Let $\mathrm{L}_n(\mathcal{K})$ denote the set of all $n\times
n$-matrices of rank at most~1 over $\mathcal{K}$. Adding the
identity matrix to $\mathrm{L}_n(\mathcal{K})$ we obtain the set
which we denote by By $\mathrm{L}^1_n(\mathcal{K})$. Clearly, it
is closed under matrix multiplication. From Lemma~\ref{rank 1} we
immediately obtain
\begin{Cor}
\label{identity for rank 1} For any field $\mathcal{K}$ and
$n\ge2$, the semigroup
$\langle\mathrm{L}^1_n(\mathcal{K}),\cdot\rangle$ satisfies the
identity
\begin{equation}
\label{abelian groups} x^nyx^{n-1}=x^{n-1}yx^n.
\end{equation}
\end{Cor}
Observe that every group satisfying \eqref{abelian groups} is
abelian.

\subsection{Matrix semigroups with Moore-Penrose inverse}
Certainly, the most common unary operation for matrices is
transposition. However, it is convenient for us to start with
analyzing matrix semigroups with Moore-Penrose inverse because
this analysis will help us in considering semigroups with
transposition.

We first recall the notion of Moore-Penrose inverse. This has been
discovered by Moore \cite{moore} and independently by Penrose
\cite{P} for complex matrices, but has turned out to be a fruitful
concept in a more general setting---see \cite{BIG} for a
comprehensive treatment.

The following results were obtained by Drazin \cite{D}.

\begin{Prop}\label{Theorem 3.3} {\rm\cite[Proposition 1]{D}}
Let $\mathcal{S}$ be an involutory semigroup. Then, for any given
$a\in\mathcal{S}$, the four equations
\begin{equation}
\label{penrose} axa=a,\ xax=x,\ (ax)^*=ax,\ (xa)^*=xa
\end{equation}
have at most one common solution $x\in\mathcal{S}$.
\end{Prop}

For an element $a$ of  an involutory semigroup $\mathcal{S}$, we
denote by $a^\dag$ the unique common solution $x$ of the equations
\eqref{penrose}, provided it exists, and call $a^\dag$ the
\emph{Moore-Penrose inverse} of $a$.

Recall that an element $a\in\mathcal{S}$ is said to be
\emph{regular}, if there is an $x\in\mathcal{S}$ such that
$axa=a$. Concerning existence of the Moore-Penrose inverse, we
have the following
\begin{Prop}\label{Theorem 3.4} {\rm\cite[Proposition 2]{D}}
Let $\mathcal{S}$ be an involutory semigroup satisfying the
quasi-identity
\begin{equation}
\label{3.1} x^*x=x^*y=y^*x=y^*y\rightarrow x=y.
\end{equation}
Then for an arbitrary $a\in\mathcal{S}$, the Moore-Penrose inverse
$a^\dag$ exists if and only if $a^*a$ and $aa^*$ are regular
elements.
\end{Prop}

Let $\langle R,+,\cdot\rangle$ be a ring. An \emph{involution of
the ring} is an involution $x\mapsto x^*$ of the semigroup
$\langle R,\cdot\rangle$ satisfying in addition the identity
$(x+y)^*=x^* +y^*$. For ring involutions, the
quasi-identity~\eqref{3.1} is easily seen to be equivalent to
\begin{equation}
\label{3.2} x^*x=0\rightarrow x=0.
\end{equation}
Now suppose that $\mathcal{K}=\langle K,+,\cdot\rangle$ is a field
that admits an involution $x\mapsto\ol x$. Then the matrix ring
$\mathrm{M}_n(\mathcal{K})$ has an involution that naturally
arises from the involution of $\mathcal{K}$, namely
$(a_{ij})\mapsto(a_{ij})^*:=(\ol{a_{ij}})^T$. This involution of
$\mathrm{M}_n(\mathcal{K})$ in general does not satisfy the
quasi-identity \eqref{3.2}. However, it does satisfy \eqref{3.2}
if and only if the equation
\begin{equation} \label{3.3}
x_1\ol{x_1}+x_2\ol{x_2}+\dots+x_n\ol{x_n}=0
\end{equation}
admits only the trivial solution $(x_1,\dots,x_n)=(0,\dots,0)$ in
$K^n$. Since all elements of $\mathrm{M}_n(\mathcal{K})$ are
regular, this means that the Moore-Penrose inverse exists ---
subject to the  involution $(a_{ij})\mapsto
(a_{ij})^*=(\ol{a_{ij}})^T$ --- whenever \eqref{3.3} admits only
the trivial solution. (The classical Moore-Penrose inverse is
thereby obtained by putting $\mathcal{K}=\bb C$, the field of
complex numbers, endowed with the usual complex conjugation
$z\mapsto \ol z$.) On the other hand, it is easy to see that the
condition that \eqref{3.3} has only the trivial solution is
necessary: if $(a_1,\dots,a_n)$ were a non-trivial solution to
\eqref{3.3}, then the matrix formed by $n$ identical rows
$(a_1,\dots,a_n)$ would have no Moore-Penrose inverse.

The proof of the main result of this subsection requires an
explicit calculation of the Moore-Penrose inverses of certain
rank~1 matrices. Thus, we present a simple method for such a
calculation. For a row vector $a=(a_1,\dots,a_n)\in K^n$, where
$\mathcal{K}=\langle K,+,\cdot\rangle$ is a field with an
involution $x\mapsto\ol x$, let $a^*$ denote the column vector
$(\ol{a_1},\dots,\ol{a_n})^T$. It is easy to see that any $n\times
n$-matrix $A$ of rank~1 over $\mathcal{K}$ can be represented as
$A=b^*c$ for some non-zero row vectors $b,c\in K^n$. Provided that
\eqref{3.3} admits only the trivial solution in $K^n$, one gets
$A^\dag$ as follows:
\begin{equation}
\label{calculating MP inverse for rank 1}
A^\dag=c^*(cc^*)^{-1}(bb^*)^{-1}b.
\end{equation}
Here $bb^*$ and $cc^*$ are non-zero elements of $\mathcal{K}$
whence their inverses in $\mathcal{K}$ exist. In order to
justify~\eqref{calculating MP inverse for rank 1}, it suffices to
check that the right hand side of~\eqref{calculating MP inverse
for rank 1} satisfies the simultaneous equations~\eqref{penrose}
with the matrix $A$ in the role of $a$, and this is
straightforward. Note that formula \eqref{calculating MP inverse
for rank 1} immediately shows that $A^\dag$ is a scalar multiple
of $A^*=c^*b$, namely
\begin{equation}\label{MP is scalar multiple}
A^\dag=\frac{1}{cc^*\cdot bb^*}A^*.
\end{equation}

So, we can formulate one of the highlights of the section --- a
result that reveals an unexpected feature of a rather classical
and well studied object.
\begin{Thm} \label{Theorem 3.5}
Let $\mathcal{K}=\langle K,+,\cdot\rangle$ be a field having an
involution $x\mapsto \ol x$ for which the equation $x\ol x +y\ol y
=0$ has only the trivial solution $(x,y)=(0,0)$ in $K^2$. Then the
unary semigroup
$\langle\mathrm{M}_2(\mathcal{K}),\cdot,{}^\dag\rangle$ of all
$2\times 2$-matrices over $\mathcal{K}$ endowed with Moore-Penrose
inversion $^\dag$ --- subject to the involution $(a_{ij})\mapsto
(a_{ij})^*= (\ol{a_{ij}})^T$
--- has no finite basis of identities.
\end{Thm}

\begin{proof}
Set $\Sc=\langle\mathrm{M}_2(\mathcal{K}),\cdot,{}^\dag\rangle$.
By Theorem~\ref{Theorem 2.1} and Lemma~\ref{Lemma 3.1} it is
sufficient to show that
\renewcommand{\labelenumi}{\theenumi)}
\begin{enumerate}
\item $\mathcal{K}_3\in\var\Sc$,
\item there exists a group $\mathcal{G}\in\var\Sc$
such that $\mathcal{G}\notin\var\H(\Sc)$.
\end{enumerate}

In order to prove 1), consider the following sets of rank~1
matrices in $\mathrm{M}_2(\mathcal{K})$:
\begin{eqnarray}
\notag H_{11}=\left\{\begin{pmatrix} x & x\\ x &
x\end{pmatrix}\right\}, & H_{12}=\left\{\begin{pmatrix} x & 0\\ x
& 0\end{pmatrix}\right\}, & H_{13}=\left\{\begin{pmatrix}
0 & x\\ 0 & x\end{pmatrix}\right\}, \\
\label{C3 is in M2} H_{21}=\left\{\begin{pmatrix} x & x\\ 0 &
0\end{pmatrix}\right\}, & H_{22}=\left\{\begin{pmatrix} x & 0\\ 0
& 0\end{pmatrix}\right\}, & H_{23}=\left\{\begin{pmatrix}
0 & x\\ 0 & 0\end{pmatrix}\right\},\\
\notag H_{31}=\left\{\begin{pmatrix} 0 & 0\\ x &
x\end{pmatrix}\right\}, & H_{32}=\left\{\begin{pmatrix} 0 & 0\\ x
& 0\end{pmatrix}\right\}, & H_{33}=\left\{\begin{pmatrix} 0 & 0\\
0 & x\end{pmatrix}\right\},
\end{eqnarray}
where in each case $x$ runs over $K\setminus\{0\}$. Observe that
$\mathcal{K}$ cannot be of characteristic $2$, since the equation
$x\ol x+y\ol y=0$ has only the trivial solution in ${K}^2$. Taking
this into account, a straightforward calculation shows that
\begin{equation}
\label{multiplication in M2} H_{ij}\cdot
H_{k\ell}=\left\{\begin{array}{cl}
H_{i\ell}&\ \text{if}\ (j,k)\ne(2,3),(3,2),\\
0 &\ \text{otherwise}.
\end{array}\right.
\end{equation}
Hence the set
$$T=\bigcup_{1\le i,j\le3}H_{ij}\cup\{0\}$$
is closed under multiplication so that this set forms a
subsemigroup $\mathcal{T}$ of $\Sc$ and the partition $\Hc$ of $T$
into the classes $H_{ij}$ and $\{0\}$ is a congruence on
$\mathcal{T}$. Equation \eqref{MP is scalar multiple} shows that
\begin{equation}
\label{inversion in M2} H_{ij}^\dag=H_{ji}.
\end{equation}
We see that $T$ is closed under Moore-Penrose inversion and $\Hc$
respects $^\dag$, thus is a congruence on the unary semigroup
$\mathcal{T}'=\left<T,\cdot,{}^\dag\right>$. Now comparing
\eqref{multiplication in M2} and \eqref{inversion in M2} with the
multiplication and inversion rules in $\mathcal{K}_3$ (see
\eqref{operations in C3}), we conclude that $\mathcal{T}'/\Hc$ and
$\mathcal{K}_3$ are isomorphic as unary semigroups. Hence
$\mathcal{K}_3$ is in $\var\Sc$.

For 2) we merely let $\mathrm{GL}_2(\mathcal{K})$, the group of
all invertible $2\times 2$-matrices over $\mathcal{K}$, play the
role of $\mathcal{G}$. Since Moore-Penrose inversion on
$\mathrm{GL}_2(\mathcal{K})$ coincides with usual matrix
inversion, we observe that $\mathrm{GL}_2(\mathcal{K})$  is a
unary subsemigroup of $\Sc$. Moreover, since $AA^\dag$ is the
identity matrix for every invertible matrix $A$, we conclude that,
with the exception of the identity matrix, the Hermitian
subsemigroup $\H(\Sc)$ contains only matrices of rank~1, that is,
$\H(\Sc)\subseteq\mathrm{L}^1_2(\mathcal{K})$, the unary semigroup
of all matrices of rank at most~1 with the identity matrix
adjoined. By Corollary~\ref{identity for rank 1} the monoid
$\mathrm{L}^1_2(\mathcal{K})$ satisfies the identity
$x^2yx=xyx^2$. Consequently, each group in $\var\H(\Sc)$ is
abelian, while the group $\mathrm{GL}_2(\mathcal{K})$ is
non-abelian. Thus, $\mathrm{GL}_2(\mathcal{K})$ is contained in
$\var\Sc$ but is not contained in $\var\H(\Sc)$, as required.
\end{proof}

\begin{Rmk}\label{remark on MP inverse}
Apart from any subfield of $\bb C$ closed under complex
conjugation, Theorem~\ref{Theorem 3.5} applies, for instance, to
finite fields $\mathcal{K}=\langle K,+,\cdot\rangle$ for which
$|K|\equiv 3\!\pmod{4}$, endowed with the trivial involution
$x\mapsto \ol x=x$; the latter follows from the fact that the
equation $x^2+1=0$ admits no solution in $\mathcal{K}$ if and only
if $|K|\equiv 3\!\pmod{4}$ (cf.\ \cite[Theorem
3.75]{LidlNiederreiter}). Moreover, by slightly changing the
arguments one can show an analogous result for $\mathcal{K}$ being
any skew-field of quaternions closed under conjugation.
\end{Rmk}

The reader may ask whether or not the restriction on the size of
matrices is essential in Theorem~\ref{Theorem 3.5}. For some
fields, it definitely is. For instance, for finite fields with the
trivial involution $x\mapsto \ol x=x$, no extension of
Theorem~\ref{Theorem 3.5} to $n\times n$-matrices with $n>2$ is
possible simply because the Moore-Penrose inverse is only a
partial operation in this case. Indeed, it is a well known
corollary of the Chevalley-Warning theorem (see, e.g.,
\cite[Corollary 2 in \S1.2]{Serre}) that the equation
$x_1^2+\dots+x_n^2=0$ (that is \eqref{3.3} with the trivial
involution) admits a non-trivial solution in any finite field
whenever $n>2$.

The situation is somewhat more complicated for subfields of $\bb
C$. Theorem~\ref{Theorem 2.1} does not apply here because of the
following obstacle. It is well known (see, for example,
\cite[p.\,101]{mks}) that the two matrices
\begin{equation}
\label{free subgroup} \zeta=\begin{pmatrix} 1 & 0\\ 2 &
1\end{pmatrix} \ \text{ and } \ \eta=\begin{pmatrix} 1 & 2\\ 0 &
1\end{pmatrix}
\end{equation}
generate a free subgroup of $\left<\mathrm{SL}_2(\bb
Z),\cdot,{}^{-1}\right>$. On the other hand, it is easy to verify
that for any subfield $\mathcal{K}$ of $\bb C$ closed under
complex conjugation, the mapping $\langle\mathrm{SL}_2(\bb
Z),\cdot,{}^{-1}\rangle\to\langle\mathrm{M}_3(\mathcal{K}),\cdot,{}^\dag\rangle$
defined by $A\mapsto\left(\begin{matrix} \raisebox{-4.5pt}{$A$}\\
\begin{smallmatrix} 0 & 0 \end{smallmatrix}\end{matrix}
\begin{smallmatrix} 0 \\ 0\\ 0\end{smallmatrix}\right)$
is an embedding of unary semigroups. Thus, for $n>2$, the unary
semigroup
$\langle\mathrm{Sing}_n(\mathcal{K}),\cdot,{}^\dag\rangle$ of
singular $n\times n$-matrices contains a free non-abelian group,
whence every group belongs to the unary semigroup variety
generated by the unary monoid
$\langle\mathrm{Sing}^1_n(\mathcal{K}),\cdot,{}^\dag\rangle$. Now
we observe that $\mathrm{Sing}^1_n(\mathcal{K})$ is contained in
(actually, coincides with) the Hermitian subsemigroup of
$\langle\mathrm{M}_n(\mathcal{K}),\cdot,{}^\dag\rangle$. Indeed,
it was proved in \cite{Erdos} (see also \cite{AM} for a recent
elementary proof) that the semigroup
$\langle\mathrm{Sing}^1_n(\mathcal{K}),\cdot\rangle$ is generated
by idempotent matrices. For an arbitrary idempotent matrix
$A\in\mathrm{M}_n(\mathcal{K})$, let $$N(A)=\{x\in K^n\mid xA=0\}\
\text{ and } \  F(A)=\{x\in K^n\mid xA=x\}$$ be the null-space and
the fixed-point-space of $A$, respectively. Now consider two
matrices of orthogonal projectors: $P_1$, the matrix of the
orthogonal projector to the space $F(A)$, and $P_2$, the matrix of
the orthogonal projector to the space $N(A)^\perp$. As any
orthogonal projector matrix $P$ satisfies $P=P^2=P^\dag$, both
$P_1=P_1P_1^\dag$ and $P_2=P_2P_2^\dag$ belong to the Hermitian
subsemigroup $\H(\mathrm{M}_n(\mathcal{K}))$, but then $A$ also
belongs to $\H(\mathrm{M}_n(\mathcal{K}))$ since
$A=(P_1P_2)^\dag$, see \cite[Exercise~5.15.9a]{Meyer}. Thus,
$\mathrm{Sing}^1_n(\mathcal{K})\subseteq
\H(\mathrm{M}_n(\mathcal{K}))$, whence no group $\mathcal{G}$ can
satisfy the condition of Theorem~\ref{Theorem 2.1}.

However, the fact that Theorem~\ref{Theorem 2.1} cannot be applied
to, say, the unary semigroup $\langle\mathrm{M}_3(\bb
C),\cdot,{}^\dag\rangle$ does not yet mean that the identities of
this semigroup are finitely based. We thus have the following open
question.
\begin{Problem}
\label{problem on Moore-Penrose} Is the unary semigroup
$\langle\mathrm{M}_n(\mathcal{K}),\cdot,{}^\dag\rangle$ not
finitely based for each subfield $\mathcal{K}$ of $\bb C$ closed
under complex conjugation and for all $n>2$?
\end{Problem}

In connection with this problem, we observe that the proofs of
Theorem~\ref{Theorem 3.5} and Corollary~\ref{identity for rank 1}
readily yield the following:
\begin{Rmk}
\label{rank 1 + invertible} For each conjugation-closed subfield
$\mathcal{K}$ of $\bb C$  and for all $n>2$, the unary semigroup
$\langle\mathrm{L}_n(\mathcal{K})\cup\mathcal{G},\cdot,{}^\dag\rangle$
consisting of all matrices of rank at most~1 and all matrices from
some non-abelian subgroup $\mathcal{G}$ of
$\mathrm{GL}_n(\mathcal{K})$ has no finite identity basis.
\end{Rmk}

Another natural related structure is the semigroup
$\textrm{M}_n(\mathcal{K})$ endowed with \textbf{both} unary
operations ${}^\dag$ and ${}^*$. Here our techniques produce a
similar result.
\begin{Thm}\label{both operations}
Let $\mathcal{K}$ be a field as in Theorem~$\ref{Theorem 3.5}$;
then $\langle\mathrm{M}_2(\mathcal{K}),\cdot,{}^\dag,{}^*\rangle$
is not finitely based as an algebraic structure of type $(2,1,1)$.
\end{Thm}

\begin{proof}
The characteristic of $\mathcal{K}$ is not $2$ whence the group
$$\mathcal{G}=\{A\in \textrm{GL}_2(\mathcal{K})\mid A^\dag=A^*\}$$
is non-abelian. Indeed, on the prime subfield of $\mathcal{K}$,
the involution $x \mapsto \ol x$ is the identity automorphism; so,
for matrices over the prime subfield, conjugation ${}^*$ coincides
with transposition,
and thus, for example, $(\begin{smallmatrix} 0 & 1\\
1 & 0\end{smallmatrix})$ and $(\begin{smallmatrix} 0 & -1\\ 1 & 0
\end{smallmatrix})$ are two non-commuting members of $\mathcal{G}$. Set
$\Ac=\langle\mathrm{M}_2(\mathcal{K}),\cdot,{}^\dag,{}^*\rangle$;
the algebraic structure $\langle G,\cdot,{}^\dag,{}^*\rangle$,
that is, the group $\mathcal{G}$ with inversion taken twice as
unary operation, belongs to $\var\Ac$. Now, as in the proof of
Theorem~\ref{Theorem 3.5}, consider the set
$$T=\bigcup_{1\le i,j\le3}H_{ij}\cup\{0\}$$
where $H_{ij}$ are defined via \eqref{C3 is in M2}. Obviously,
$H_{ij}^*=H_{ji}$ whence $\Tc=\langle T,\cdot,{}^\dag,{}^*\rangle$
is a substructure of $\Ac$ and the partition $\Hc$ of $T$ into the
classes $H_{ij}$ and $\{0\}$ is a congruence on this substructure.
The quotient $\Tc/\Hc$ is then isomorphic to the semigroup
$\mathcal{K}_3$ endowed twice with its unary operation. We
conclude that  $\mathcal{K}_3$ treated this way also belongs to
$\var\Ac$. By Corollary~\ref{identity for rank 1} the identity
$x^2yx=xyx^2$ holds in $\H(\Ac)$ (by which we mean the
substructure of $\Ac$ generated by all elements of the form
$AA^\dag$). Now construct the semigroups $\Tc_k$ (by use of the
identity $x^2yx=xyx^2$) as in Step~2 in the proof of
Theorem~\ref{Theorem 2.1} and endow each of them twice with its
unary operation. The arguments in Steps~3 and~4 in the proof then
show that $\Tc_k$ does not belong to $\var \Ac$ while each
$k$-generated substructure of $\Tc_k$ does belong to $\var\Ac$.
Thus, $\Tc_k$ can play the role of critical structures for
$\var\Ac$ whence the desired conclusion follows by the reasoning
as in Step~1 in the proof of Theorem~\ref{Theorem 2.1}.
\end{proof}

Also in this setting, our result gives rise to a natural question.
\begin{Problem}
\label{problem on Moore-Penrose + conjugation} Is the algebraic
structure
$\langle\mathrm{M}_n(\mathcal{K}),\cdot,{}^\dag,{}^*\rangle$ of
type $(2,1,1)$ not finitely based for each subfield $\mathcal{K}$
of $\bb C$ closed under complex conjugation and for all $n>2$?
\end{Problem}

Here an observation similar to Remark~\ref{rank 1 + invertible}
can be stated: for each con\-jugation-closed subfield
$\mathcal{K}\subseteq\bb C$ and for all $n>2$, the algebraic
structure
$\langle\mathrm{L}_n(\mathcal{K})\cup\mathrm{GL}_n(\mathcal{K}),\cdot,{}^\dag,{}^*\rangle$
consisting of all matrices of rank at most~1 and all invertible
matrices has no finite identity basis.


\subsection{Matrix semigroups with transposition}
First of all, we observe that the involutory semigroup $\langle\mathrm{M}_n(\mathcal{K}),\cdot,{}^T\rangle$ is finitely
based for any field $\mathcal{K}$ of characteristic $0$. Indeed, we have already mentioned that the two matrices
$\zeta$ and $\eta$ in~\eqref{free subgroup} generate a free subgroup of $\langle\mathrm{SL}_2(\bb
Z),\cdot,{}^{-1}\rangle$ and hence a free subsemigroup of $\langle\mathrm{SL}_2(\bb Z), \cdot\rangle$. But
$\eta=\zeta^T$ whence the unary subsemigroup in $\langle\mathrm{SL}_2(\bb Z),\cdot,{}^T\rangle$ generated by $\zeta$ is
isomorphic to the free monogenic involutory semigroup $\mathrm{FI}(\{\zeta\})$. The latter semigroup is known to
contain as a unary subsemigroup a free involutory semigroup on countably many generators, namely, $\mathrm{FI}(Z)$
where
$$Z=\{\zeta \zeta ^T\zeta ,\ \zeta (\zeta ^T)^2\zeta ,\ \dots,\ \zeta (\zeta ^T)^n\zeta ,\ \dots\}.$$
Hence all identities holding in
$\langle\mathrm{M}_n(\mathcal{K}),\cdot,{}^T\rangle$ with $n\ge 2$
and $\mathcal{K}$ of characteristic $0$ follow from the
associativity and the involution laws $(xy)^T = y^Tx^T$, $(x^T)^T
= x$. Similarly, $\langle\mathrm{M}_n(R),\cdot,{}^*\rangle$ is
finitely based for each $n \ge 2$ and each subring $R\subseteq \bb
C$ closed under complex conjugation---here ${}^*$ stands for the
complex-conjugate transposition $(a_{ij})^*= (\ol{a_{ij}})^T$.

We note that Theorem~\ref{Theorem 2.1} and Corollary~\ref{identity
for rank 1} prove the non-existence of a finite identity bases for
the unary subsemigroup of $\langle\mathrm{M}_n(\bb
C),\cdot,^*\rangle$ [respectively $\langle\mathrm{M}_n(\bb
R),\cdot,{}^T\rangle$] that consists of all matrices of rank at
most~1 together with all unitary [respectively all orthogonal]
matrices.

For the case of finite fields, one can show that Theorem
~\ref{Theorem 2.1} solves the finite basis problem in the negative
for the involutory semigroup
$\langle\mathrm{M}_2(\mathcal{K}),\cdot,{}^T\rangle$ for each
finite field $\mathcal{K}$ except $\mathcal{K}=\mathbb{F}_2$, the
$2$-element field. Indeed, it can be verified that the involutory
semigroup $\langle\mathrm{M}_2(\bb F_2),\cdot,{}^T\rangle$
satisfies the identity
$$(xx^*)^3(yy^*)^3=(yy^*)^3(xx^*)^3$$
which does not hold in $\mathcal{K}_3$; consequently,
$\mathcal{K}_3$ is not in $\var\langle\mathrm{M}_2(\bb
F_2),\cdot,{}^T\rangle$ and Theorem~\ref{Theorem 2.1} does not
apply here. In the following theorem, we shall  demonstrate the
application of Theorem \ref{Theorem 2.1} only in the case when
$\mathcal{K}$ has odd characteristic. With some additional effort
we could include also the case when the characteristic of
$\mathcal{K}$ is $2$ and $\vert K\vert\ge 4$. We shall omit this
since that case will be covered by a different kind of proof
later.

\begin{Thm}\label{Theorem 3.6}
For each finite field $\mathcal{K}$ of odd characteristic, the
involutory semigroup
$\langle\mathrm{M}_2(\mathcal{K}),\cdot,{}^T\rangle$ has no finite
identity basis.
\end{Thm}

\begin{proof}
Let $\Sc=\langle\mathrm{M}_2(\mathcal{K}),\cdot,{}^T\rangle$. As
in the proof of Theorem \ref{Theorem 3.5} one shows that
$\mathcal{K}_3$ is in $\var\Sc$.  Furthermore, let $d$ be the
exponent of the group $\mathrm{GL}_2(\mathcal{K})$.  By
Corollary~\ref{identity for rank 1}, each group in
$\P_d(\var\Sc)=\var\P_d(\Sc)$ satisfies the law $x^2yx=xyx^2$ and
therefore is abelian. On the other hand, as in the proof of
Theorem \ref{both operations}, the group $\mathcal{G}=\{A\in
\mathrm{GL}_2(\mathcal{K})\mid A^T=A^{-1}\}$ is in $\var\Sc$ but
is non-abelian. Thus, Theorem~\ref{Theorem 2.1} applies.
\end{proof}

The next theorem contains the even characteristic case and proves,
in fact, a stronger assertion.

\begin{Thm}\label{inherently degree 2}
For each finite field $\mathcal{K}=\langle K,+,\cdot\rangle$ with
$\vert K\vert\mathrel{\not\equiv} 3\ (\bmod\ 4)$, the involutory
semigroup $\langle\mathrm{M}_2(\mathcal{K}),\cdot,{}^T\rangle$ is
inherently non-finitely based.
\end{Thm}

\begin{proof} As mentioned in Remark \ref{remark on MP inverse},
there exists $x\in K$ for which $1+x^2=0$. Now consider the
following matrices:
\begin{gather*}
H_{11}=\begin{pmatrix} 1 & x\\ x &x^2\end{pmatrix},\
H_{12}=\begin{pmatrix} 1 & 0\\ x &0\end{pmatrix},\
H_{21}=\begin{pmatrix} 1 & x\\ 0 & 0\end{pmatrix},\
H_{22}=\begin{pmatrix} 1 & 0\\ 0 & 0\end{pmatrix},\\
I=\begin{pmatrix}1 & 0\\ 0& 1\end{pmatrix},\ O=\begin{pmatrix} 0 &
0 \\ 0 & 0 \end{pmatrix}.
\end{gather*}
Then the set $M=\{H_{11},H_{12},H_{21}, H_{22},I, O\}$ is closed
under multiplication and transposition, hence $\mathcal{M}=\langle
M, \cdot,{}^T\rangle$ is an involutory subsemigroup of
$\langle\mathrm{M}_2(\mathcal{K}),\cdot,{}^T\rangle$. The mapping
$\TA\to \mathcal{M}$ given by
$$(i,j)\mapsto H_{ij},\ 0\mapsto O,\ 1\mapsto I$$
is an isomorphism of involutory semigroups. The result now follows
from Corollary~\ref{twisted A}.
\end{proof}

The case of matrix semigroups of degree greater than 2 is similar.

\begin{Thm} \label{inherently degree 3}
For each $n\ge 3$ and each finite field $\mathcal{K}$, the
involutory semigroup
$\langle\mathrm{M}_n(\mathcal{K}),\cdot,{}^T\rangle$ is inherently
non-finitely based.
\end{Thm}

\begin{proof} It follows from the Chevalley-Warning theorem
\cite[Corollary~2 in \S1.2]{Serre} that there exist $x,y\in K$
satisfying $1+x^2+y^2=0$. Now consider the following matrices:
\begin{gather*}
H_{11}=\begin{pmatrix} 1 & x & y\\ x &x^2& xy\\ y & xy &
y^2\end{pmatrix},\ H_{12}=\begin{pmatrix} 1 & 0 & 0\\ x &0 &0\\ y
& 0 & 0\end{pmatrix},\
H_{21}=\begin{pmatrix} 1 & x & y\\ 0 & 0 & 0\\ 0 & 0&0\end{pmatrix},\\
H_{22}=\begin{pmatrix} 1 & 0 & 0\\ 0 & 0&0\\0&0&0\end{pmatrix},\
I=\begin{pmatrix} 1&0&0\\0&1&0\\0&0&1\end{pmatrix},\
O=\begin{pmatrix} 0&0&0\\0&0&0\\0&0&0\end{pmatrix}.
\end{gather*}
Again, the set $M=\{H_{11},H_{12},H_{21},H_{22},I,O\}$ is closed
under multiplication and transposition, and as in the previous
proof, $\mathcal{M}=\langle M ,\cdot,{}^T\rangle$ forms an
involutory subsemigroup of
$\langle\mathrm{M}_3(\mathcal{K}),\cdot,{}^T\rangle$ that is
isomorphic with $\TA$. Hence $\langle
\mathrm{M}_3(\mathcal{K}),\cdot,{}^T\rangle$ is inherently
non-finitely based. The assertion for $\mathrm{M}_n(\mathcal{K})$
for $n\ge 3$ now follows in an obvious way.
\end{proof}

\begin{Rmk}
The statements of Theorems \ref{Theorem 3.6}, \ref{inherently
degree 2}, \ref{inherently degree 3} remain valid if the unary
operation $A\mapsto A^T$ is replaced with an operation of the form
$A\mapsto A^{\si T}$ for any automorphism $\si$ of $\mathcal{K}$,
where $(a_{ij})^{\si T}:=(a_{ij}^\si)^T$.
\end{Rmk}

We are ready to prove the result announced in Section 1 as
Theorem~\ref{main result involution} (for the reader's convenience
we reproduce this result here as Theorem~\ref{main result matrix
involution}). With the exception of the `only if' part of item
(2), this is a summary of Theorems \ref{Theorem 3.6},
\ref{inherently degree 2}, and \ref{inherently degree 3}.
\begin{Thm}\label{main result matrix involution} Let $n\ge 2$ and
$\mathcal{K}=\langle K,+,\cdot\rangle$ be a finite field. Then
\begin{enumerate}
\item the involutory semigroup $\langle \mathrm{M}_n(\mathcal{K}),\cdot,{}^T\rangle$
is not finitely based;
\item the involutory semigroup $\langle \mathrm{M}_n(\mathcal{K}),\cdot,{}^T\rangle$
is inherently non-finitely ba\-sed if and only if either $n\ge 3$ or $n=2$ and $\vert K\vert\mathrel{\not\equiv 3}
(\bmod\ 4)$.
\end{enumerate}
\end{Thm}

It remains to prove that $\langle\mathrm{M}_2(\mathcal{K}),\cdot,{}^T\rangle$ is \textbf{not} inherently non-finitely
based if $\vert K\vert\equiv 3\ (\bmod\ 4)$. The assertion follows from Proposition~\ref{NINFB}.

\begin{proof}
As already remarked, the only assertion of this theorem not covered by previous results is that
$\langle\mathrm{M}_2(\mathcal{K}),\cdot,{}^T\rangle$ is \textbf{not} inherently non-finitely based if $\vert
K\vert\equiv 3\ (\bmod\ 4)$. We employ Proposition~\ref{NINFB} to prove this.

Recall that the condition $\vert K\vert\equiv 3\ (\bmod\ 4)$ corresponds precisely to the case when each matrix $A$ in
$\langle\mathrm{M}_2(\mathcal{K}),\cdot,{}^T\rangle$ admits a Moore-Penrose inverse $A^\dag$ (Remark~\ref{remark on MP
inverse}). Let $A$ be a matrix of rank 1; by (\ref{MP is scalar multiple}) there exists a scalar $\alpha\in
\mathcal{K}\setminus\{0\}$ such that $\alpha A^\dag= A^T$. Let $r=\vert K\vert -1$; then $\alpha^r=1$. Since the
multiplicative subgroup of $\mathcal{K}$ is a cyclic subgroup of $\mathrm{GL}_2(\mathcal{K})$, the number $r$ divides
the exponent $d$ of $\mathrm{GL}_2(\mathcal{K})$ whence $\alpha^d=1$. Consequently,
$$A(A^TA)^d= A(\alpha A^\dag A)^d=\alpha^d A(A^\dag A)^d=A.$$
If $A\in\mathrm{GL}_2(\mathcal{K})$, we also have $A=A(A^TA)^d$ because $(A^TA)^d$ is the identity matrix; clearly, the
equality $A=A(A^TA)^d$ holds also for the case when $A$ is the zero matrix. Summarizing, we conclude that the identity
$x=x(x^Tx)^d$ holds in the involutory \sm\ $\langle\mathrm{M}_2(\mathcal{K}),\cdot,{}^T\rangle$. Setting
$\om(x):=x^T(xx^T)^{d-1}$, we see that $\langle\mathrm{M}_2(\mathcal{K}),\cdot,{}^T\rangle$ satisfies the identity
$x=x\om(x)x$, as required by Proposition~\ref{NINFB}.
\end{proof}

\begin{Rmk}
It is known~\cite[Corollary~6.2]{sapirburnside} that the matrix
semigroup $\langle\mathrm{M}_n(\mathcal{K}),\cdot\rangle$ is
inherently non-\fb\ (as a plain \sm) for every $n\ge 2$ and every
finite field $\mathcal{K}$. Thus, the involutory semigroups
$\langle\mathrm{M}_2(\mathcal{K}),\cdot,{}^T\rangle$ over finite
fields $\mathcal{K}$ such that $\vert K\vert\equiv 3\ (\bmod\ 4)$
provide a natural series of unary semigroups whose equational
properties essentially differ from the equational properties of
their semigroup reducts.
\end{Rmk}

\subsection{Matrix semigroups with symplectic transpose}
For a $2n\times 2n$-matrix
$$X=\begin{pmatrix} A & B\\ C & D\end{pmatrix}$$ with $A,B,C, D$ being
$n\times n$-matrices over any field $\mathcal{K}$, the
\emph{symplectic transpose} $X^S$ is defined by
$$X^S=\begin{pmatrix} D^T & -B^T\\ -C^T & A^T\end{pmatrix},$$
see e.g. \cite[(5.1.1)]{Procesi}. This definition is similar to
that of the involution in the twisted Brandt monoid
$\mathcal{TB}^1_2$ (defined in terms of $2\times 2$-matrices) and
leads to the following application.
\begin{Thm}\label{symplecticinvolution}
The involutory semigroup
$\left<\mathrm{M}_{2n}(\mathcal{K}),\cdot,{}^S\right>$ is
inherently non-finitely based for each $n\ge 1$ and each finite
field $\mathcal{K}=\left<K,+,\cdot\right>$.
\end{Thm}
\begin{proof} Consider the following sets of $2n\times 2n$-matrices:
\begin{gather*}
H_{11}=\left\{\pm \begin{pmatrix}O_n & I_n \\ O_n&O_n
\end{pmatrix}\right\},\
H_{12}=\left\{\pm \begin{pmatrix}I_n & O_n \\ O_n&O_n \end{pmatrix}\right\},\\
H_{21}=\left\{\pm \begin{pmatrix}O_n & O_n \\ O_n&I_n
\end{pmatrix}\right\},\
H_{22}=\left\{\pm \begin{pmatrix}O_n & O_n \\ I_n&O_n
\end{pmatrix}\right\}
\end{gather*}
where for any positive integer $k$, we denote be $I_k$,
respectively, $O_k$ the identity, respectively, zero $k\times
k$-matrix. Let
$$T=\bigcup_{1\le i,j\le 2}H_{ij}\cup \{O_{2n}, I_{2n}\}.$$
The set $T$ is closed under multiplication and symplectic
transposition whence $\mathcal{T}=\left<T,\cdot,{}^S\right>$ forms
an involutory subsemigroup of
$\left<\mathrm{M}_{2n}(\mathcal{K}),\cdot,{}^S\right>$. On the
other hand, the mapping
$$H_{ij}\mapsto (i,j),\ I_{2n}\mapsto 1,\ O_{2n}
\mapsto 0$$ is a homomorphism of $\mathcal{T}$ onto
$\mathcal{TB}^1_2$. Altogether, the twisted Brandt monoid
$\mathcal{TB}^1_2$ divides
$\left<\mathrm{M}_{2n}(\mathcal{K}),\cdot,{}^S\right>$.
\end{proof}
\subsection{Boolean matrices}
Recall that a \emph{Boolean} matrix is a matrix with entries $0$
and $1$ only. The multiplication of such matrices is as usual,
except that addition and multiplication of the entries is defined
as: $a+b=\max\{a,b\}$ and $a\cdot b=\min\{a,b\}$. Let $B_n$ denote
the set of all Boolean $n\times n$-matrices. It is well known that
the \sm\ $\langle B_n,\cdot\rangle$ is essentially the same as the
semigroup of all binary relations on an $n$-element set subject to
the usual composition of binary relations. The operation ${}^T$ of
forming the matrix transpose then corresponds to the operation of
forming the dual binary relation.

\begin{Thm}
\label{Theorem 3.7} For each integer $n\ge 2$, the involutory
semigroup $\Bc_n=\langle B_n,\cdot,{}^T\rangle$ of all Boolean
$n\times n$-matrices endowed with transposition is inherently
non-finitely based.
\end{Thm}

\begin{proof} Consider the Boolean matrices
\begin{gather*}
B_{11}=\begin{pmatrix} 0&1\\1&1\end{pmatrix},\
B_{12}=\begin{pmatrix} 1&0\\1&1 \end{pmatrix},\
B_{21}=\begin{pmatrix} 1&1\\0&1 \end{pmatrix},\
B_{22}=\begin{pmatrix} 1&1\\1&0\end{pmatrix},\\
O=\begin{pmatrix} 1&1\\1&1 \end{pmatrix},\
I=\begin{pmatrix}1&0\\0&1 \end{pmatrix}.
\end{gather*}
The set $M=\{B_{11}, B_{12}, B_{21}, B_{22},O,I\}$ is closed under
multiplication and transposition whence $\mathcal{M}=\langle
M,\cdot,{}^T\rangle$ is an involutory subsemigroup of $\Bc_2$. The
mapping $\TB\to \mathcal{M}$ given by
$$ (i,j)\mapsto B_{ij},\ 0\mapsto O,\ 1\mapsto I$$
is an isomorphism of involutory semigroups. By
Corollary~\ref{twisted Brandt} $\mathcal{M}$ is inherently
non-finitely based whence so is $\Bc_2$. Since $\Bc_2$ can be
embedded as an involutory semigroup into $\Bc_n$ for each $n$, the
result follows.
\end{proof}

\begin{Rmk}
We can unify Theorem~\ref{Theorem 3.7} and some results in
Subsection~2.3 by considering matrix semigroups with transposition
over \textbf{semirings}. A \emph{semiring} is an algebraic
structure $\mathcal{L}=\langle L,+,\cdot\rangle$ of type $(2,2)$
such that $\langle L,+\rangle$ is a commutative semigroup,
$\langle L,\cdot\rangle$ is a semigroup and multiplication
distributes over addition. From the proofs of
Theorems~\ref{inherently degree 2} and~\ref{Theorem 3.7} we see
that the involutory matrix semigroup
$\langle\mathrm{M}_n(\mathcal{L}),\cdot,{}^T\rangle$ over a finite
semiring is inherently non-finitely based whenever $n\ge 2$ and
the semiring $\mathcal{L}$ has a zero 0 (that is, a neutral
element for $\langle L,+\rangle$ which is at the same time an
absorbing element for $\langle L,\cdot\rangle$) and satisfies
either of the following two conditions:
\begin{enumerate}
\item there exist (not necessarily distinct) elements $e,x\ne 0$
such that $e^2=e$, $ex=xe=x$, $e+x^2=0$;
\item there exists an element $e\ne 0$ such that $e^2=e=e+e$.
\end{enumerate}
We have already met an infinite series of semirings satisfying
(1): it consists of the finite fields $\mathcal{K}=\langle
K,+,\cdot\rangle$ with $\vert K\vert\mathrel{\not\equiv 3}(\bmod\
4)$. It should be noted that semirings satisfying (2) are even
more plentiful: for example, finite distributive lattices as well
as the power semirings of finite semigroups (with the subset union
as addition and the subset product as multiplication) fall in this
class.
\end{Rmk}

\noindent\textbf{Acknowledgement.} The second author was supported
by Grant No.\ 144011 of the Ministry of Science and Technological
Development of the Republic of Serbia. The third author
acknowledges support from the Federal Education Agency of Russia,
grant 2.1.1/3537, and from the Russian Foundation for Basic
Research, grant 09-01-12142.

\begin{thebibliography}{99}

\bibitem{Alm95}
J.\,Almeida, \emph{Finite Semigroups and Universal Algebra}, World
Scientific, Singapore, 1995.

\bibitem{AM}
J.\,Ara\'ujo and J.\,D.\,Mitchell, \emph{An elementary proof that
every singular matrix is a product of idempotent matrices}, Amer.\
Math.\ Monthly \textbf{112} (2005), 641--645.

\bibitem{A1}
K.\,Auinger, \emph{Strict regular $*$-semigroups}, pp.190--204 in:
Proceedings of the Conference on Semigroups with Applications, J.
M. Howie, W. D. Munn and H.-J. Weinert (eds.), World Scientific,
Singapore, 1992.

\bibitem{BIG}
A. Ben-Israel and Th.\ Greville, \emph{Generalized Inverses:
Theory and Applications}, Springer-Verlag, Berlin--Heidelberg--New
York, 2003.

\bibitem{BuSa81}
S. Burris and H. P. Sankappanavar, \emph{A Course in Universal
Algebra}, Springer-Verlag, Berlin--Heidelberg--New York, 1981.

\bibitem{CP}
A.\,H.\,Clifford and G.\,B.\,Preston, \emph{The Algebraic Theory
of Semigroups I}, Amer.\ Math.\ Soc., Providence, 1961.

\bibitem{Dolinka}
I. Dolinka, \emph{On identities of finite involution semigroups}, Semigroup Forum \textbf{80} (2010), 105--120.

\bibitem{D}
M.\,P.\,Drazin, \emph{Regular semigroups with involution},
pp.29--46 in: Proceedings of the Symposium on Regular semigroups,
Northern Illinois University, De Kalb, 1979.

\bibitem{Erdos}
J.\,A.\,Erdos, \emph{On products of idempotent matrices}, Glasgow
Math.\ J. \textbf{8} (1967), 118--122.

\bibitem{GoMi78}
I.\,Z.\,Golubchik and A.\,V.\,Mikhalev, \emph{A note on varieties
of semiprime rings with semigroup identities}, J. Algebra
\textbf{54} (1978), 42--45.

\bibitem{KR}
K.\,H.\,Kim and F.\,Roush, \emph{On groups in varieties of
semigroups}, Semigroup Forum \textbf{ 16} (1978), 201--202.

\bibitem{kleiman}
E.\,I.\,Kleiman, \emph{Bases of identities of varieties of inverse
semigroups}, Sibirsk.\ Mat.\ Zh. \textbf{20} (1979), 760--777
[Russian; English transl.\ Sib.\ Math. J. \textbf{20}, 530--543].

\bibitem{LidlNiederreiter}
R.\,Lidl and H.\,Niederreiter, \emph{Finite Fields},
Addison-Wesley, Cambridge, 1997.

\bibitem{mks}
W.\,Magnus, A.\,Karras and D.\,Solitar, \emph{Combinatorial Group
Theory}, Wiley, New York--London--Singapore, 1966.

\bibitem{margolissapir}
S.\,W.\,Margolis and M.\,V.\,Sapir, \emph{Quasi-identities of
finite semigroups and symbolic dynamics}, Israel J. Math.
\textbf{92} (1995), 317--331.

\bibitem{Meyer}
C.\,D.\,Meyer, \emph{Matrix Analysis and Applied Linear Algebra},
SIAM, Philadelphia, 2000.

\bibitem{moore}
E.\,H.\,Moore, \emph{On the reciprocal of the general  algebraic
matrix}, Bull.\ Amer.\ Math.\ Soc. \textbf{26} (1920),  394--395.

\bibitem{Ne}
H.\,Neumann, \emph{Varieties of Groups}, Springer-Verlag,
Berlin--Heidelberg--New York, 1967.

\bibitem{oatespowell}
S.\,Oates and M.\,B.\, Powell, \emph{Identical relations in finite
groups}, J. Algebra  \textbf{1} (1964), 11--39.

\bibitem{P}
R.\,Penrose, \emph{A generalized inverse for matrices}, Proc.\
Cambridge Phil.\ Soc. \textbf{51} (1955), 406--413.

\bibitem{Per89}
P.\,Perkins, \emph{Finite axiomatizability for equational theories
of computable groupoids}, J. Symbolic Logic \textbf{54} (1989),
1018--1022.

\bibitem{Procesi}
C.\,Procesi, \emph{Lie Groups: an Approach through Invariants and
Representations}, Springer-Verlag, Berlin--Heidelberg--New York,
2006.

\bibitem{sapirinherently}
M.\,V.\,Sapir, \emph{Inherently non-finitely based finite
semigroups}, Mat.\ Sb. \textbf{133}, no.2 (1987), 154--166
[Russian; English transl.\ Math. USSR-Sb. \textbf{61} (1988),
155--166].

\bibitem{sapirburnside}
M.\,V.\,Sapir, \emph{Problems of Burnside type and the finite
basis property in varieties of semigroups}, Izv.\ Akad.\ Nauk
SSSR, Ser.\ Mat. \textbf{51} (1987), 319--340 [Russian; English
transl.\ Math.\ USSR-Izv. \textbf{30} (1987), 295--314].

\bibitem{S}
M.\,V.\,Sapir, \emph{On Cross semigroup varieties and related
questions}, Semigroup Forum \textbf{ 42} (1991), 345--364.

\bibitem{sapirinverse}
M.\,V.\,Sapir, \emph{Identities of finite inverse semigroups},
Internat.\ J. Algebra Comput. \textbf{3} (1993), 115--124.

\bibitem{Sapir}
M. V. Sapir, \emph{Combinatorics on words with applications},
IBP-Litp 1995/32: Rapport de Recherche Litp, Universit\'e Paris 7,
1995 (available online under\\
\url{http://www.math.vanderbilt.edu/~msapir/ftp/course/course.pdf}).

\bibitem{Serre}
J.-P. Serre, \emph{Cours d'Arithmetique}, Presses Universitaires
de France, Paris, 1980.

\bibitem{V}
M.\,V.\,Volkov, \emph{On finite basedness of semigroup varieties},
Mat.\  Zametki \textbf{ 45}, no.3 (1989),  12--23 [Russian;
English transl.\ Math.\ Notes \textbf{ 45} (1989), 187--194].

\bibitem{volkovjaponicae}
M.\,V.\,Volkov, \emph{The finite basis problem for finite
semigroups}, Sci.\ Math.\ Jpn. \textbf{53} (2001), 171--199.
\end{thebibliography}

\end{document}
