\documentclass[11pt]{article}

\usepackage{amsmath,amsthm,amssymb}
\usepackage{url}
\usepackage{longtable}
\usepackage{amsfonts}
\usepackage{eepic}
\usepackage{amsgen}
\usepackage{eucal}
\usepackage{times}
\usepackage{hhline}
\usepackage{pstricks,pst-node,pst-text,pst-3d}
\usepackage{color}

\DeclareMathOperator{\var}{\mathsf{var}}

\DeclareSymbolFont{rsfscript}{OMS}{rsfs}{m}{n} \DeclareSymbolFontAlphabet{\mathrsfs}{rsfscript}

\def\cal{\mathcal}
\def\Ac{{\cal A}}
\def\Bc{{\cal B}}
\def\Cc{{\cal C}}
\def\Dc{\mathrsfs{D}}
\def\Ec{{\cal E}}
\def\Fc{{\cal F}}
\def\Gc{{\cal G}}
\def\Hc{\mathrsfs{H}}
\def\Ic{{\cal I}}
\def\Jc{{\cal J}}
\def\Kc{{\cal K}}
\def\Lc{\mathrsfs{L}}
\def\Mc{{\cal M}}
\def\Nc{{\cal N}}
\def\Oc{{\cal O}}
\def\Pc{{\cal P}}
\def\Qc{{\cal Q}}
\def\Rc{\mathrsfs{R}}
\def\Sc{{\cal S}}
\def\Tc{{\cal T}}
\def\Uc{{\cal U}}
\def\Vc{\mathbf{V}}
\def\Wc{\mathbf{W}}
\def\Xc{{\cal X}}
\def\Yc{{\cal Y}}
\def\Zc{{\cal Z}}

\def\H{\mathrm H}
\def\P{\mathrm P}

\textwidth=18.71cm
\oddsidemargin=-1.5cm
\parindent=0pt

\begin{document}

\setlongtables
\begin{longtable}{|p{2.2cm}|p{1.8cm}|p{4.2cm}|p{4.2cm}|p{4.2cm}|}
\caption*{\textbf{Proof reading for ``Matrix identities involving multiplication and transposition'' by Auinger et al}}\\
\hline
\textbf{Location} & \textbf{Type} & \textbf{In the proofs} & \textbf{In the original} & \textbf{Should be} \\
\hhline{|=|=|=|=|=|}
\endfirsthead
\hline
\multicolumn{5}{|l|}{\slshape continued from previous page}\\
\hline
\textbf{Location} & \textbf{Type} & \textbf{In the proofs} & \textbf{In the original} & \textbf{Should be} \\
\hhline{|=|=|=|=|=|}
\endhead
\hline
\multicolumn{5}{|r|}{\slshape continued on next page}\\
\hline
\endfoot
\hline
\endlastfoot
P.1, footnote, line~+4 & Update & 21000 & 21000 & 21101\\
\hline
P.1, footnote, line~+5 & Update & Faculty of Mathematics and Mechanics, Ural State University &
Faculty of Mathematics and Mechanics, Ural State University & Institute of Mathematics and Computer Science, Ural Federal University\\
\hline
P.1, footnote, line~+6 & Update & 620083 & 620083 & 620000\\
\hline
P.2, line~+21 & Typo (our fault) & \dots may be {\red a} summarized \dots &
\dots may be {\red a} summarized \dots & \dots may be summarized \dots\\
\hline
P.2, Theorem, line~+1 & Editor's intervention & \rule{0pt}{1pt}{\red None} of {\red the} following sets of matrix identities admits
{\red a} finite identity basis: &
Each of following sets of matrix identities admits no finite identity basis: & We quite agree with moving the negation
into the subject but according to standards of English grammar, when the sense is plural (as indicated by a plural noun
or pronoun in the following prepositional phrase---``none of [plural entity]"), ``none'' is plural. So the phrase should be:

None of the following sets of matrix identities admit a finite identity basis: \\
\hline
P.3, line~+18 & Editor's intervention & \dots then so is $u^*$. & \dots then so is $(u)^*$. & As in the original
(we  \textbf{\red do not} accept the change)\\
\hline
P.3, line~+20 & Editor's intervention &  $u\mapsto u^*$. & $u\mapsto (u)^*$. & As in the original
(we \textbf{\red do not} accept the change)\\
\hline
P.3, line~$-$3 & Typo (our fault) & A variety is {\red is} said to be \dots &
A variety is {\red is} said to be \dots & A variety is said to be \dots\\
\hline
P.7, matrix $M_n(g)$, entry (4,4) & Editor's intervention & \rule{0pt}{1pt}{\red $\vdots$} (produced by \verb+\vdots+) &
$\ddots$ (produced by \verb+\ddots+) &
As in the original (we \textbf{\red do not} accept the change)\\
\hline
P.11, line~+5 & Editor's intervention & \dots to
${\red (1,2,\ldots,r,\ldots,1,2,\ldots,r)^t}$ {\red where} the block $1,2,\dots,r$ occurs
$r$ times.

& \dots to the transpose of the row
$(1,2,\ldots,r,\ldots,1,2,\ldots,r)$ in which the block $1,2,\dots,r$ occurs
$r$ times.
& We \textbf{\red do not} accept the change in the proposed form. The notation $(\dots)^t$
for the transpose is inconsistent with the notation elsewhere in the paper. We suggest:

\dots to the transpose of $(1,2,\ldots,r,\ldots,1,2,\ldots,r)$ where the block $1,2,\dots,r$ occurs
$r$ times.\\
\hline
P.15, lines $-$20 and $-$19 & Update & We say that $b$ \emph{strictly divides} $a$ and write
$a\mathrel{{<}_{\Rc}}b$ if $a=bs$ for some $s\in S$ but $b\ne a$ and $b\ne at$ for any $t\in S$. &
We say that $b$ \emph{strictly divides} $a$ and write $a\mathrel{{<}_{\Rc}}b$ if $a=bs$ for some $s\in S$
but $b\ne a$ and $b\ne at$ for any $t\in S$. & Remove the whole sentence\\
\hline
P.15, lines $-$18 and $-$17 & Update & $\Rc$ is an equivalence relation (known as the \emph{right Green relation}
in semigroup theory) and $<_{\Rc}$ is transitive and anti-reflexive. &
$\Rc$ is an equivalence relation (known as the \emph{right Green relation}
in semigroup theory) and $<_{\Rc}$ is transitive and anti-reflexive. &
$\Rc$ is an equivalence relation (known as the \emph{right Green relation}
in semigroup theory).

(Remove the part of the sentence after the clause in parentheses.)\\
\hline
P.15, lines $-$9, $-$8, and $-$7 & Update & Further let $h$ denote the length of the longest possible chain of the form
$$s_1\mathrel{{<}_{\Rc}}s_2\mathrel{{<}_{\Rc}}\cdots\mathrel{{<}_{\Rc}}s_k.$$ &
Further let $h$ denote the length of the longest possible chain of the form
$$s_1\mathrel{{<}_{\Rc}}s_2\mathrel{{<}_{\Rc}}\cdots\mathrel{{<}_{\Rc}}s_k.$$ &
Remove the whole sentence\\
\hline
P.15, line $-$6 & Update & Set $n=h+1$; Lemma~7 in~[37] shows \dots &
Set $n=h+1$; Lemma~7 in~[37] shows \dots  &
Set $n={\red |S|}+1$; Lemma~7 in~[37] {\red implies} \dots \\
\hline
P.18, line~+5 & Update & \dots admits an involution \dots &
\dots admits an involution \dots  & \dots admits a {\red ring} involution \dots\\
\hline
P.24, line~+7 & Typo (our fault) + Editor's intervention & \rule{0pt}{12pt}$b^{\ell_1}a^{k_1}\cdots b^{\ell_{t-1}}a^{k_{t-1}}b^{{\red k}_t}${\red,} &
\rule{0pt}{12pt}$b^{\ell_1}a^{k_1}\cdots b^{\ell_{t-1}}a^{k_{t-1}}b^{{\red k}_t}$ &
\rule{0pt}{1pt}$b^{\ell_1}a^{k_1}\cdots b^{\ell_{t-1}}a^{k_{t-1}}b^{\ell_t}$,
\\
\hline
P.29, line~+17 & Update & the name suggested in~\verb+\cite{Kim}+ &
the name suggested in~\verb+\cite{Kim}+ & the name suggested in~\verb+\cite{Schwarz}+\\
\hline
P.31, Acknow\-ledgements, line~$-$1 & Typo (our fault) & \dots grant{\red s} 10-01-00524. &
\dots grant{\red s} 10-01-00524. & \dots grant 10-01-00524.\\
\hline
P.32, item [27] & Update & \verb+\bibitem{Kim}+

Kim, K. H.: The semigroups of Hall relations. Semigroup Forum \textbf{9}, 253--260 (1974)
Zbl 0292.20061 MR 0376910&
\verb+\bibitem{Kim}+

Kim, K. H.: The semigroups of Hall relations. Semigroup Forum \textbf{9}, 253--260 (1974)&
Remove this item\\
\hline
P.33, item [45], line +2 & Typo (our fault) & (200{\red 6}) &
(200{\red 6})& (2007)\\
\hline
P.33, between items [51] and [52]& Update &\multicolumn{3}{p{12.6cm}|}{Insert new item:

{\tt $\backslash$bibitem\{Schwarz\}}

Schwarz, \v{S}.: The semigroup of fully indecomposable relations and
Hall relations. Czechoslovak Math. J. \textbf{23}, 151--163 (1973) Zbl 0261.20057  MR 0316612}\\
\hline
P.34, item [55], line +2 & Typo & Zb{\red k} 1074.20036 &
&
Zbl 1074.20036\\
\end{longtable}
\end{document}
