

% Modif. April 30, 2010
% Send comments to publ@impan.pl

\documentclass[12pt, twoside]{article}
\usepackage{amsmath,amsthm,amssymb}
\usepackage{times}
\usepackage{enumerate}

\pagestyle{myheadings}
\def\titlerunning#1{\gdef\titrun{#1}}
\makeatletter
\def\author#1{\gdef\autrun{\def\and{\unskip, }#1}\gdef\@author{#1}}
\def\address#1{{\def\and{\\\hspace*{18pt}}\renewcommand{\thefootnote}{}%
\footnote {#1}}%
\markboth{\autrun}{\titrun}}
\makeatother
\def\email#1{e-mail: #1}
\def\subjclass#1{{\renewcommand{\thefootnote}{}%
\footnote{\emph{Mathematics Subject Classification (2010):} #1}}}
\def\keywords#1{\par\medskip
\noindent\textbf{Keywords.} #1}


%% Numbered objects of "theorem" style (text italicized).
%% The optional parameters indicate that all objects are numbered together, and "by section".
%% However, you are welcome to use any other numbering system of your choice.

\newtheorem{thm}{Theorem}[section]
\newtheorem{cor}[thm]{Corollary}
\newtheorem{lem}[thm]{Lemma}
\newtheorem{prob}[thm]{Problem}


%% A numbered theorem with a fancy name:

\newtheorem{mainthm}[thm]{Main Theorem}

%% Numbered objects of "non-theorem" style (text roman):

\theoremstyle{definition}
\newtheorem{defin}[thm]{Definition}
\newtheorem{rem}[thm]{Remark}
\newtheorem{exa}[thm]{Example}

%% An unnumbered remark:

\newtheorem*{xrem}{Remark}


%% Equations numbered by section:

\numberwithin{equation}{section}


%%%%%%%%%%% For JEMS
\frenchspacing

\textwidth=15cm
\textheight=23cm
\parindent=16pt
\oddsidemargin=-0.5cm
\evensidemargin=-0.5cm
\topmargin=-0.5cm

%%%%%%%%%%%%%%%%%%%%%%%%%%%%%%%%%%%
%%%%%%%%%%%%%%%%%%%%%%%%%%%%%%%%%%%

%%%% Put your macros here:






%%%%%%%%%%%%%


\begin{document}

%%%%% To ease editing, add:

\baselineskip=17pt

%%%%%%%%%%%%%%%%

%% In the running head, give an abbreviation of the title. 
\titlerunning{JEMS template}

\title{A template for submissions to JEMS}

\author{Firstname1 Surname1
\and 
Firstname2 Surname2}

\date{}

\maketitle

\address{F1. Surname1: address1; \email{xx1@yy1}
\and
F2. Surname2: address2; \email{xx2@yy2}}


\subjclass{Primary XXXX; Secondary YYYY}

%%%%%%%%

\begin{abstract}
A template for submissions to JEMS. For the subject classification, use
the 2010 Mathematics Subject Classification available at www.ams.org/msc.

%% Keywords are optional
\keywords{Aaaa, bbbb, cccc}
\end{abstract}

\section{Introduction}
You can use this file as a template when submitting your paper to JEMS.

The format of the file is \textbf{not} the exact final printed format (the latter is scaled down), but it is convenient for editing purposes.

Put all acknowledgments (including those concerning grants) just before references.

\subsection{Theorems etc.}
The statements of theorems, lemmas etc. are set in
italics. In definitions, only the term being defined is italicized. Remarks
and examples are set in roman type. 



\begin{defin}
A system $S$ is said to be \emph{self-extensional} if 
$S \in B$.
\end{defin}

\begin{thm}[Maximum Principle, see also {\cite[Theorem 5]{Shchepin}}]
If (\ldots), then the following conditions are equivalent: 

\begin{enumerate}[\upshape (i)]
\item first item,
\item second item.
\end{enumerate}
\end{thm}

\begin{proof} Observe that
\begin{align}\label{E:1}
AAAAAAAAAA &= BBBBBBBBBBB\notag\\
           &\quad + CCCCCCCCCC\notag\\
           &= DDDDDDDDDDDDD.
\end{align}
Now apply induction on $n$ to \eqref{E:1}.
\end{proof}

%%%%
%%%% For many examples of codes of complicated multiline formulas, see
%%%%   http://journals.impan.pl  ("For authors")

If you want to put the end-of-proof sign after a (non-numbered) formula, use $\backslash$qedhere:

\begin{proof} This follows from
\begin{gather*}
BBBBBBBBBBBBBBBB=CCCCCCCCCCCCC,\\
DDDDDDD-EEEEEEEEEE=0.\qedhere
\end{gather*}
\end{proof}



\begin{prob}
Is AAAA true?
\end{prob}




\begin{xrem}
An unnumbered remark.
\end{xrem}


\begin{mainthm} 
Here comes the statement of a numbered theorem with a fancy name.
\end{mainthm}


\subsection{Figures}
Figures must be prepared as eps or pdf files.
All figures will be printed black and white; the colours will only appear in the online version.

\bigskip
\footnotesize
\noindent\textit{Acknowledgments.}
This research was partly supported by NSF (grant no. XXXX).

\begin{thebibliography}{SK}

%% Use the widest label as parameter.

%% Reference items may be numbered or have labels of your choice.
%% The author's surname PRECEDES the initial of the first name
%% The issue number is only given when the issues are paginated separately.
%% In book titles, first letters are capitalized.
%% Only journal volume numbers are boldfaced.

%%%%%%%%%%% To ease editing, use normal size:

\normalsize
\baselineskip=17pt

%%%%%%%%%%%%%

\bibitem[B]{Barlow} 
Barlow, M. T.: 
Diffusions on fractals.  
In: Lectures on Probability Theory and Statistics (Saint-Flour, 1995), 
Lecture Notes in Math. 1690, Springer, Berlin, 1--121 (1998)


\bibitem[G]{Gratzer} 
Gr\"atzer, G.:
More Math into \LaTeX.
4th ed., Springer, Berlin (2007)


\bibitem[SK]{SatoKashiwaraKawai}
Sato, M., Kashiwara, M., Kawai, M.: 
Linear differential equations of infinite order and theta functions.
Adv. Math. \textbf{47}, 300--325 (1983)


\bibitem[Sh]{Shchepin}
Shchepin, E. V.:
On mappings of the two-dimensional sphere.  
Uspekhi Mat. Nauk \textbf{58}, no.~2, 169--170 (2003) (in Russian); 
English transl.: Russian Math. Surveys \textbf{58}, 1218--1219 (2003)

\bibitem[S]{Smith} 
Smith, J.:
A new upper bound on the cross number.
arXiv:2056.7895.


\bibitem[V]{Verkaar}
Verkaar, M.:
Continuous local martingales and stochastic integration in Banach spaces.
PhD thesis, Univ. of Helsinki (2001)
\end{thebibliography}


\end{document}